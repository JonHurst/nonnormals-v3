\documentclass[a5paper,11pt,twoside]{book}
\usepackage[utf8]{inputenc}
\usepackage[T1]{fontenc}
\usepackage{lmodern}
\usepackage{textcomp}
\usepackage{microtype}
\usepackage[left=0.75in, bottom=1in]{geometry}
\usepackage{color}
\usepackage[color]{changebar}
\cbcolor{blue}
\usepackage{array}
\usepackage[pdfauthor={Jon Hurst},hidelinks]{hyperref}
\usepackage[normalem]{ulem}
\title{A320 Family Non-Normal-Notes\\~\\\large{Version 3.0}}
\author{Jon Hurst}
\date{}


\newcommand{\ac}[1]{{\scshape\MakeLowercase{#1}}}
\newcommand{\ecam}[2]{{\ac{\uline{#1} #2}}}
\newcommand{\cphrase}[1]{\ac{#1}}
\newcommand{\inlcite}[1]{{\ac{#1}}}
\newcommand{\multicite}[1]{%
  \nopagebreak
  \noindent{{\color{blue}\footnotesize[ \inlcite{#1} ]}}
}
\newcommand{\V}[1]{V\textsubscript{#1}}
\newcommand{\strong}[1]{\textbf{#1}}

\begin{document}
\frontmatter
\maketitle

\ifdefined\HCode
\else
\tableofcontents
\fi

\chapter{links to be moved}
%%%%%%%%%%%%%%%%
%labels to be moved once section is processed
\label{sec-dualhyd-gy}
%%%%%%%%%%%%%%%%

\mainmatter

\chapter{Operating techniques}
\section{Rejected Takeoff}
The decision to reject rests solely with \ac{CM}1. This decision is communicated
with the words ``Stop'' or “Go''. ``Stop'' implies that \ac{CM}1 is taking
control of the aircraft.

Below 100kt the \ac{RTO} is relatively risk free and a decision to stop should
be made for any \ac{ECAM} and most other problems.

Above 100kt the \ac{RTO} may be hazardous and stopping should only be considered
for loss of engine thrust, any fire warning, any uninhibited
\ac{ECAM}\footnote{There are five uninhibited amber \ac{ECAM} cautions that
require a high speed \ac{RTO}. Only two uninhibited \ac{ECAM}s are not on this
list: \ecam{ENG}{1(2) THR LEVER DISAGREE} if the \ac{FADEC} automatically
selects idle thrust and \ecam{FWS}{FWC 1+2 FAULT}. The first of these should
never happen due to \ac{FADEC} logic. The second generates a message on the
\ac{EWD} but no master caution (it is the computers that generate master
cautions that have failed). You could therefore modify this rule to: stop for
any \ac{ECAM} warning or caution except the caution-like \ecam{FWS}{FWC 1+2
  FAULT}.}, or anything which indicates the aircraft will be unsafe or unable to
fly.

If a stop is required, \ac{CM}1 calls ``Stop'' while simultaneously bringing the
thrust levers to idle, then to max reverse.

If the stop was commenced below 72kt the ground spoilers will not automatically
deploy and the autobrake will therefore not engage. Monitor automatic braking,
and if there is any doubt, apply manual braking as required. If normal braking
fails, announce ``Loss of braking'' and proceed with the loss of braking memory
items (see Section \ref{sec-loss-of-braking}).

\looseness=1
If the reason for the stop was an engine fire on the upwind side, consider
turning the aircraft to keep the fire away from the fuselage.

If there is any chance of requiring evacuation, bring the aircraft to a complete
halt, stow the reversers, apply the parking brake, and order ``Attention, crew
at stations'' on the \ac{PA}.

If evacuation will definitely not be required, once the aircraft's safety is
assured the \ac{RTO} can be discontinued and the runway cleared. In this case
make a \ac{PA} of ``Cabin crew, normal operations''.

\looseness=1
During this initial phase, \ac{CM2} confirms reverser deployment (``Reverse
green''), confirms deceleration (``Decel''), cancels any audio warnings, informs
\ac{ATC} and announces ``70 knots'' when appropriate. \ac{CM2} then locates the
emergency evacuation checklist.

Once the aircraft has stopped, \ac{CM1} takes the radios and asks \ac{CM2} to
carry out any required \ac{ECAM} actions. Whilst the \ac{ECAM} actions are being
completed, \ac{CM1} will build up a decision as to whether to evacuate. If an
evacuation is required, see Section \ref{sec-evacuation}. Otherwise order
``Cabin crew, normal operations''.

If the aircraft has come to a complete halt using autobrake \cphrase{MAX}, the brakes
can be released by disarming the spoilers.

\looseness=1
If, following an \ac{RTO}, a new takeoff is to be attempted, reset both
\ac{FD}s, set the \ac{FCU}, then restart \ac{SOP}s from the After Start
checklist. Carefully consider brake temperatures; temperature indications
continue to climb for some time after a significant braking event.

\multicite{EOMB~3.10, FCTM~PRO.AEP.MISC}


\section{Asymmetric takeoff}

Apply rudder conventionally to maintain runway track. At V\textsubscript{r}
rotate at a slightly reduced rate towards an initial pitch target of
12\textonehalf\textdegree\, then target speed \V{2} to \V{2}+15kt. Bank angle
should be limited to 15\textdegree{ }when more than 3kt below manoeuvring speed
for the current configuration.\footnote{This is a conservative rule of thumb. If
the \ac{FMGC} has correctly identified an engine out condition, the \ac{FD/AP}
will automatically limit bank angle according to a less conservative algorithm
(see \inlcite{FCOM~SYS.22.20.60.40})}

When the ground to flight mode transition is complete,\footnote{Introducing
\ac{TOGA} during the ground to flight mode transition (commences as the pitch
increases through 8\textdegree, complete after 5 seconds) results in a pitch up
moment at a time where the effect of stick pitch control is not wholly
predictable: the stick will need to be moved forward of neutral to counteract
the introduced pitch moment and then returned to neutral as flight mode blends
in. A slight pause before selecting \ac{TOGA} results in much more normal and
predictable handling.} select \ac{TOGA}.\footnote{\ac{FLX} may be used but this
tends to allow speed to decay unless pitch is reduced} Adjust and trim rudder
to maintain $\beta$ target; this will result in a small side-slip angle towards
the failed engine. Engage the autopilot once gear is up and rudder is trimmed.

%%  and request ``pull heading''. If the \ac{EOSID}
%% follows the track of the cleared \ac{SID}, \ac{NAV} may be used, but this is
%% very rare with easyJet \ac{EOSID}s.

\section{Low level failure handling}
\label{sec-failures-after-v1}

Handling of failures that occur on the takeoff roll or at very low level is
primarily a test of triage skills. Airbus provides support in three ways:

\begin{itemize}
\item \ac{ECAM} flight phase inhibitions filter out less serious failures until
  1500ft \ac{AAL} is attained on climb out or speed is below 80kt on the landing
  roll.
\item \ac{TOGA} thrust is made available for an extra 5 minutes (giving a total
  of 10 minutes) in emergency situations, which allows more time before
  acceleration and cleanup is required.
\item A recommendation is published at the start of the \ac{QRH}, that, apart
  from cancelling audio warnings, ``no action will be taken'' until an
  appropriate flight path is established and the aircraft is at least 400ft
  \ac{AGL}.\footnote{\ldots although it does go on to water this statement down
  by saying that, for some unspecified emergency situations, the 400ft part of
  the recommendation may be disregarded}
\end{itemize}

Whilst below 400ft, then, the focus should be on flying and monitoring, with
heightened awareness of the possibility of missing essential normal actions,
such as calling rotate or raising the gear due to the distraction of the
failure. It may help \ac{PF} assimilate the challenges of the flying task if
\ac{PM} states the title of the first displayed \ac{ECAM} procedure, but no
action to diagnose or contain the failure should be taken. A very quick and well
timed ``Mayday, Mayday, Mayday, standby'' and ``Attention Crew at Stations''
from \ac{PM} may also be useful to forestall external interruptions. When and if
possible, the autopilot should be engaged to reduce workload.

An extremely useful tool for dealing with low level failures is the \ac{EOSID},
as described in Section \ref{sec-eosid}. While the name suggests that this is
only to be used with engine failures, in reality it is a pre-planned safe flight
path that may be flown whenever there is doubt that available aircraft
performance is sufficient to fly the cleared \ac{SID} or go-around. Since both
pilots should already be aware of the details of the \ac{EOSID}, \ac{PM} can
simply declare their intention to fly it to generate a shared mental model.
Note that flying an \ac{EOSID} is higher workload than flying the planned
\ac{SID} in \ac{NAV} or flying visually; it is an option, not a requirement.

Another useful tool, often used in association with the \ac{EOSID}, is the
concept of ``high priority tasks''. These are defined as:

\begin{itemize}
\item For engine failure, the master switch of the affected engine has been
  turned off.

\item For engine fires, \emph{either} one squib has been fired and the fire
  warning has extinguished \emph{or} both squibs have been fired.
\end{itemize}

These definitions help with the triage process, allowing standardised
interleaving of the diagnosis and containment of the failure with the flying and
monitoring tasks. In particular, the level acceleration phase of the \ac{EOSID}
is generally delayed until these tasks are completed, with the phrase ``engine
is secure'' a de facto standard call to indicate that \ac{PF} might like to
consider interrupting the containment process.\footnote{The phrases ``Stop
\ac{ECAM}'' and ``Continue \ac{ECAM}'' are standard for interrupting
containment.}  Unfortunately, they are only defined for engine failures and
fires, so with other failures, the crew will have to make a judgement call as to
what constitutes ``high priority''.

\multicite{FCTM~PRO.AEP.ENG}

\section{EOSID}
\label{sec-eosid}

Before the divergence point (the last common point between the \ac{SID} and the
\ac{EOSID}), if the aircraft detects a loss of thrust the \ac{EOSID} will be
displayed as a temporary flight plan. In this case the temporary flight plan can
be inserted and \ac{NAV} mode used. Otherwise it will be necessary to pull
heading and manually follow either the yellow line or bring up a pre-prepared
secondary flight plan and follow the white line.

If beyond the divergence point, pull heading and make an \emph{immediate} turn
the shortest way onto the \ac{EOSID}. Airbus specifically recommends against
this (\inlcite{FCOM~AS.22.20.60}), but easyJet states it as policy
(\inlcite{EOMB 4.4.4}).

Electing to fly the \ac{EOSID} implies a level acceleration segment:

\begin{itemize}
\item Initially fly a \ac{TOGA} climb at the higher of \V{2} or current speed,
  up to a limit of \V{2}+15kt. If a \ac{FLEX} takeoff was carried out, a
  \ac{FLEX} climb is permissible. This climb is continued until all high
  priority tasks are complete (see Section \ref{sec-failures-after-v1}) and the aircraft is
  above single engine acceleration altitude (usually 1000ft \ac{QFE}, but may be
  higher if so specified by the take-off performance calculation). If the
  \ac{FMGS} has detected the engine out condition, the automatic mode change
  from \ac{SRS} to \ac{CLB} will be inhibited; if not, intervention with
  selected modes will be required to prevent untimely acceleration.

\item The next segment is a \ac{TOGA} level acceleration and clean up, either to
  \ac{Conf} 1 and S speed if an immediate \ac{VMC} return is desired or to
  \ac{Conf} 0 and green dot. Again \ac{FLEX} may be used if a \ac{FLEX} takeoff
  was carried out. Level acceleration is usually achieved by pushing \ac{V/S};
  if the \ac{FMGS} has detected the engine out condition, all preselected speeds
  entered in the \ac{MCDU} will have been deleted, so the managed target speed
  should automatically move to 250kt. The phrases ``Stop \ac{ECAM}'' and
  ``Continue \ac{ECAM}'' can be used to interrupt \ac{ECAM} procedures in order
  to initiate this segment.

\item The final segment is a \ac{MCT} climb segment to \ac{MSA}, either at S
  speed if in \ac{Conf} 1 or at green dot speed if in \ac{Conf} 0. This is
  usually achieved in open climb; if the \ac{FMGS} has detected the engine out
  condition, the managed target speed becomes dependent on flight phase, and in
  this case should automatically select green dot.
\end{itemize}

\ac{TOGA} may be used for a maximum of 10 minutes.

If an \ac{EOSID} is annotated as ``\cphrase{STD}'', then acceleration to green
dot should be completed prior to commencing the first turn. If
``\cphrase{NON-STD}'', the turn takes priority.

\multicite{EOMB~4.4.4, FCOM~DSC.22\_20.60.40}

\chapter{Miscellaneous}

\section{Emergency descent \emph{(memory item)}}
\label{sec-emer-descent}

If an emergency descent is required, the Captain should consider taking control
if not already \ac{PF}. \ac{PF} initiates the memory items by announcing
``Emergency Descent.''

Don oxygen masks and establish communication.

\ac{PF} then flies the emergency descent.  Descent with autopilot and autothrust
engaged is preferred. The configuration is thrust idle, full speed brake and
maximum appropriate speed, taking into account possible structural
damage.\footnote{According to Airbus, structural damage may be suspected if
there has been a ``loud bang'' or there is a high cabin vertical speed. When
limiting descent speed due to suspected structural damage, it is \ac{IAS} rather
than Mach that is relevant.} Target altitude is \ac{FL100} or \ac{MORA} if this
is higher. If speed is low, allow speed to increase before deploying full
speedbrake to prevent activation of the angle of attack protection. Landing gear
may be used, but speed must be below \V{LO} when it is extended and remain below
\V{LE}. If on an airway, consider turning 90\textdegree{ }to the left.

\ac{PM}'s only memory action is to turn the seatbelt signs on; their primary
task is to ensure that \ac{PF} has promptly and correctly initiated the descent.

Once the memory actions are complete and the aircraft is descending, \ac{PF}
should call for the Emergency Descent Checklist (``My radios, Emergency Descent
Checklist''). This will lead \ac{PF} to finesse the speed and altitude targets and
inform \ac{ATC} of the descent; \ac{PM} to set continuous ignition on the engines and set
7700 on the transponder. Both pilots then set their oxygen flows to the \ac{N}
position\footnote{There may be insufficient oxygen to cover the entire emergency
descent profile if the oxygen masks are left set to 100\%.} and, if cabin
altitude will exceed 14,000ft, \ac{PM} deploys the cabin oxygen masks. On easyJet
aircraft, the \ac{CIDS}/\ac{PRAM} will automatically play a suitable \ac{PA}, so it is not
necessary for the flight crew to carry out the \ecam{}{EMER DESCENT (PA)} action.

Once level, restore the aircraft to a normal configuration. When safe to do so,
advise cabin crew and passengers that it is safe to remove their masks. To
deactivate the mask microphone and switch off the oxygen flow, close the oxygen
mask stowage compartment and press the ``\cphrase{PRESS TO RESET}'' oxygen control slide.

\multicite{EOMB~3.80.2, QRH~AEP.MISC, FCOM~AEP.MISC, FCTM~AEP.MISC}


\section{Windshear \emph{(memory item)}}

\subsection{Takeoff roll}

Windshear encountered on the takeoff roll is \emph{only} detectable by
significant airspeed fluctuations. It is possible that these fluctuations may
cause \V{1} to occur significantly later in the takeoff roll then it should. In
this case it falls to the Captain to make an assessment of whether sufficient
runway remains to reject the takeoff, or whether getting airborne would be the
better option. If the takeoff is to be continued, call ``Windshear, \ac{TOGA''}
and apply \ac{TOGA} power.  Rotate at \V{r} or with sufficient runway
remaining\footnote{``Sufficient runway remaining'' is actually Boeing advice –
Airbus offers no guidance for the case where there is insufficient runway
available to stop nor to rotate at normal speeds.} and follow \ac{SRS}
orders. \ac{SRS} will maintain a minimum rate of climb, even if airspeed must be
sacrificed.

\subsection{Reactive}

The reactive windshear detection system is a function of the \ac{FAC}s. It only
operates when below 1300ft \ac{RA} with at least \ac{CONF} 1 selected. In the
takeoff phase, it is inhibited until 3 seconds after lift off and in the landing
phase it is inhibited below 50ft \ac{RA}.

A warning is indicated by a red \cphrase{WINDSHEAR} flag on the \ac{PFD} and a
``Windshear, Windshear, Windshear'' aural warning. Call ``Windshear, \ac{TOGA}''
and apply \ac{TOGA} power.

The autopilot can fly the escape manoeuvre as long as the required \ac{AOA} is
less than $\alpha$\textsubscript{prot}. If the autopilot is not engaged, follow
the \ac{SRS} orders on the \ac{FD}s. If the \ac{FD}s are not available,
initially pitch up to 17.5\textdegree, then increase as required.

Do not change configuration until out of the windshear. Once clear of the
windshear, clean up the aircraft: leveraging the go-around procedure is useful
for this.

In severe windshear, it is possible that Alpha Floor protection will
activate. As \ac{TOGA} will already be selected, this will have no immediate
effect. Once clear of the windshear, however, \ac{TOGA} lock will be
active. This, combined with the unusual aircraft configuration, leads to a
significant threat of overspeed. The most natural way to disengage \ac{TOGA}
lock is to disengage the autothrust using the instinctive disconnect \ac{PB} on
the thrust levers then use manual thrust until the situation has sufficiently
stabilised to re-engage the autothrust.

\multicite{FCOM~PRO.AEP.SURV}

\subsection{Predictive}

Below 2300ft \ac{AGL}, the weather radar scans a 5nm radius 60\textdegree\ arc
ahead of the aircraft for returns indicating potential windshear.

Alerts are categorised as \strong{advisory}, \strong{caution} or
\strong{warning}, in increasing order of severity. Severity is determined by
range, position and phase of flight. Alerts are only provided when between 50ft
and 1500ft, or on the ground when below 100kt.

All types of alert produce an indication of windshear position on the \ac{ND},
provided that the \ac{ND} range is set to 10nm. A message on the \ac{ND}
instructs the crew to change range to 10nm if not already set. A
\strong{caution} additionally gives an amber \cphrase{W/S AHEAD} message on
both \ac{PFD}s and an aural ``Monitor Radar Display'' warning. A
\strong{warning} additionally gives a red \cphrase{W/S AHEAD} message on the
\ac{PFD}s and either a ``Windshear Ahead, Windshear Ahead'' or ``Go Around,
Windshear Ahead'' aural message.

If a \strong{warning} occurs during the takeoff roll, reject the takeoff. If it
occurs during initial climb, call ``Windshear, \ac{TOGA}'', apply \ac{TOGA}
thrust and follow \ac{SRS} orders. If it occurs during approach, fly a normal
go-around. Configuration may be changed as long as the windshear is not entered.

If a \strong{caution} occurs during approach, use \ac{CONF} 3 to optimise
go-around climb gradient and consider increasing \V{APP}; up to a maximum of
\V{LS}+15 may be used.

If positive verification is made that no hazard exists and providing that the
reactive windshear is serviceable the crew may downgrade the \strong{warning} to
a \strong{caution.}

\multicite{FCTM PR.NP.SP.10.10}

\section{Unreliable airspeed \emph{(memory item)}}
\label{sec-unreliable-airspeed}

Unreliable airspeed indications may result from radome damage and/or
unserviceable probes or ports. Altitude indications may also be erroneous if
static probes are affected.

The \ac{FMGC}s normally reject erroneous \ac{ADR} data by isolating a single
source that has significant differences to the other two sources. It is possible
that a single remaining good source may be rejected if the other two sources are
erroneous in a sufficiently similar way. In this case, it falls to the pilots to
identify and turn off the erroneous sources to recover good data.

The first problem is recognition of a failure, since the aircraft systems may be
unable to warn of a problem. The primary method of doing this is correlation of
aircraft attitude and thrust to displayed performance. Correlation of radio
altimeter and \ac{GPIRS} derived data (available on \cphrase{GPS MONITOR} page)
may also aid identification. The stall warning (available in alternate or direct
law) is based on alpha probes, so will likely be valid. Other clues may include
fluctuations in readings, abnormal behaviour of the automatics, high speed
buffet or low aerodynamic noise.

If the aircraft flight path is in doubt, disconnect the automatics and fly the
following short term attitude and thrust settings to initiate a climb:

\bigskip
\begin{tabular}{|l|c|c|}
  \hline
  \textbf{Condition} & \textbf{Thrust} & \textbf{Pitch}\\\hline
  Below Thrust Reduction Altitude & \ac{TOGA} & 15\textdegree \\\hline
  Below \ac{FL}100 & \ac{CLB} & 10\textdegree \\\hline
  Above \ac{FL}100 & \ac{CLB} & 5\textdegree \\\hline
\end{tabular}
\bigskip

If configured \ac{CONF} Full, select \ac{CONF} 3, otherwise flap/slat
configuration should be maintained. The gear and speedbrake should be
retracted. If there is any doubt over the validity of altitude information, the
\ac{FPV} must be disregarded. If altitude information is definitely good, the
\ac{FPV} may be used.

It is important to understand that at this stage, while the pilot has identified
that airspeed is unreliable, the aircraft systems have not. Thus flight envelope
protections based on airspeed data from unreliable \ac{ADR}s may activate. This
may lead to pitch inputs from the flight computers that cannot be overridden
with the sidesticks. In this case, immediately switch off any two \ac{ADR}s;
this causes the flight computers to revert to Alternate Law with no protections,
and thus allows control of the aircraft to be regained.

Once the flight path is under control and a safe altitude is attained, the
aircraft should be transitioned into level flight. Refer to the \ac{QRH}
\ecam{NAV}{Unreliable Speed Indication} procedure to extract a ballpark thrust
setting, a reference attitude and a reference speed for the current
configuration, bearing in mind that an auto-retraction of the flap may have
occurred. Set the ballpark thrust setting and adjust pitch attitude to fly
level; if barometric altitude data is considered accurate use the \ac{VSI},
otherwise fly a constant \ac{GPS} altitude. The thrust should then be adjusted
until level flight is achieved with the reference attitude. Note that in the
radome damage case, the required N1 may be as much as 5\% greater than the
ballpark figure. Once stable, the speed will be equal to the reference speed.

If there is insufficient data available to fly level (e.g. \ac{GPS} data
unavailable and barometric data unreliable), fly the reference attitude with the
ballpark thrust setting. This will give approximately level flight at
approximately reference speed.

With the speed now known, the \ac{ADR}s can be checked to see if any are giving
accurate data. If at least one \ac{ADR} is reliable, turn off the faulty
\ac{ADR}s. \ac{GPS} and \ac{IRS} ground speeds may also be used for an
approximate cross check.

If all \ac{ADR}s are considered unreliable, turn off any two of them; one is
kept on to provide stall warning from the alpha probes. More recent aircraft
have backup speed/altitude scales based on \ac{AOA} probes and \ac{GPS}
altitudes which are activated when below \ac{FL}250 by turning off the third
\ac{ADR}. The \ac{QRH} \ecam{NAV}{ALL ADR OFF} procedure describes the use of
these scales, but it boils down to fly the green on the speed scale and
anticipate slightly reduced accuracy from the altitude scale.  For aircraft
without this functionality, tables are provided in the \ac{QRH}
\ecam{NAV}{Unreliable Speed Indication} procedure to enable all phases of flight
to be flown using just pitch and thrust settings. Acceleration and clean up are
carried out in level flight. \ac{CONF} 1 can be selected as soon as climb thrust
is selected, \ac{CONF} 0 once the appropriate S speed pitch attitude from the
table on the first page of the procedure is reached. Configuration for approach
is also carried out in level flight, stabilising in each configuration using the
technique described above. The approach is flown in \ac{CONF} 3 at an attitude
that should result in \V{LS}+10 when flying a 3\textdegree{ }glide. Landing
distance will be increased.

\multicite{QRH~AEP.NAV, FCOM~PRO.AEP.NAV, FCTM~PRO.AEP.NAV}

\section{Incapacitation}

Take control, using the stick priority button if necessary. Contact cabin crew
\ac{ASAP}. They should strap the incapacitated pilot to his seat, move the seat
back, then recline it. If there are two cabin crew available, the body can be
moved. Medical help should be sought from passengers, and the presence of any
type rated company pilots on board ascertained.

\multicite{FCTM~PRO.AEP.MISC}

\section{Forced Landing (inc. Ditching)}

There are two scenarios where off field forced landing or ditching would be
considered: either you have insufficient energy to reach a suitable airfield
(e.g actual or impending fuel exhaustion, catastrophic failure of both engines),
or you have insufficient time to do so (e.g. uncontained fire).

Ditching and off field forced landing without power are discussed in Section
\ref{sec-all-engine-failure}. Support for such landings is provided by the
\ecam{ENG}{ALL ENGINES FAILURE} \ac{ECAM} and the \ecam{ENG}{ALL ENG FAIL}
\ac{QRH} procedure.

Support for ditching and forced landing with power is provided by the \ac{QRH}
\ecam{MISC}{Ditching} and \ac{QRH} \ecam{MISC}{Forced Landing} procedures
respectively. Necessarily implicit in these checklists is the assumption that
the aircraft is fully serviceable, which is unlikely to be the case. There is
also an assumption that plenty of time is available for extensive preparation of
cabin and cockpit. It is highly likely that these checklists will need adapting
to the situation.

The fundamentals are the same with or without power. Ditching will be gear up
with a target pitch attitude of 11\textdegree{ }and minimal vertical speed, and
should be made parallel to the swell unless there are strong crosswinds, in
which case an into wind landing is preferred. Forced landing will be gear down
with the spoilers armed. The aircraft should be depressurised for the landing,
with the Ditching \ac{PB} pushed for the ditching case; in the without power
cases this is achieved using \ac{RAM AIR}, and this is the same for with power
forced landing. The ``with power'' ditching achieves depressurisation by turning
off all bleeds, which provides a slightly more watertight hull.

The main difference between the with and without power cases is that max
available slats and flaps are used in the former case, wheras \ac{CONF} 2 is
mandated for the latter. Approach speeds must also be high enough to prevent
\ac{RAT} stall (i.e. >140kt) if it is being relied upon. The combination of
these factors means that much lower landing speeds can be achieved if power is
available.

\multicite{QRH~AEP.MISC}

\section{Evacuation}
\label{sec-evacuation}
Evacuation should be carried out in accordance with the emergency evacuation
checklist. The easyJet procedure is for \ac{CM}1 to call for the checklist and
then send a Mayday message to \ac{ATC} before commencing the checklist.

The first two items confirm the \ac{RTO} actions of stopping the aircraft,
setting the parking brake and alerting the cabin crew. The next item confirms
\ac{ATC} has been alerted.

The next four items prepare the aircraft for evacuation. If manual cabin
pressure has been used, \ac{CM}2 checks cabin diff is zero, and if necessary
manually opens the outflow valve. \ac{CM}2 then shuts the engines down with
their master switches, and pushes all the fire buttons (including the
\ac{APU}). Confirmation is \emph{not} required before carrying out these
actions. In response to the next checklist item, ``Agents'', \ac{CM}1 decides if
any extinguishing agents should be discharged and instructs \ac{CM}2 to
discharge them as required. Engine agent 2 will not be available. Agents should
only be discharged if there are positive signs of fire.

Finally, order the evacuation. This is primarily done with the \ac{PA}
``Evacuate, unfasten your seat belts and get out'', with the evacuation alarm
being triggered as a backup.

\multicite{EOMB~3.80.1, FCOM~PRO.AER.MISC, FCTM~PRO.AER.MISC}


\section{Overweight landing}

A landing can be made at any weight, providing sufficient landing distance is
available. In general, automatic landings are only certified up to \ac{MLW}, but
the \ac{FCOM} specifies that, for the A319 only, autoland is available up to
69000kg in case of emergency.

The preferred landing configuration is \ac{CONF}~Full, but lower settings may be
used if required by \ac{QRH} or \ac{ECAM} procedures. \ac{QRH}
\ecam{MISC}{Overweight Landing} also specifies \ac{CONF}~3 if the aircraft
weight exceeds the \ac{CONF}~3 go around limit; this will only ever be a factor
for airfields with elevations above 1000ft.

Packs should be turned off to provide additional go around thrust.

If planned landing configuration is less than \ac{Conf}~Full, use \ac{Conf}~1+F
for go-around.

It is possible that S speed will be higher than \V{FE~next} for \ac{CONF}~2. In this
case, a speed below \V{FE~next} should be selected until \ac{CONF}~2 is achieved,
then managed speed can be re-engaged.

In the final stages of the approach, reduce speed to achieve \V{LS} at runway
threshold. Land as smoothly as possible, and apply max reverse as soon as the
main gear touches down. Maximum braking can be used after nosewheel
touchdown.

After landing, switch on the brake fans and monitor brake temperatures
carefully. If temperatures exceed 800\textdegree C, tyre deflation may occur.

\multicite{QRH~AER.MISC, FCOM~PRO.AER.MISC, FCTM~PRO.AER.MISC}

\section{Engine failure in cruise}

Engine out ceiling is highly dependent on weight; \ac{ISA} deviation also has a
modest effect. It will generally lie between \ac{FL}180 and \ac{FL}350.

The first action will be to select both thrust levers to \ac{MCT} so as to allow
the autothrust its full engine out range. If the \ac{N}1 gauges indicate a
thrust margin exists, then the aircraft is below engine out ceiling; descent may
be appropriate to increase the available thrust margin, but there is no
immediate threat. If, however, the \ac{N}1 gauges indicate that the autothrust
is commanding \ac{MCT}, and the speed is still decaying, then the aircraft is
above engine out ceiling and prompt execution of a drift down procedure is
required.

Drift down with autopilot engaged in \ac{OP} \ac{DES} is preferred. Engagement
of this vertical mode normally results in the autothrust commanding idle thrust,
which is not what is desired. Thus, having set the thrust lever to \ac{MCT}, the
autothrust is disconnected. The \ac{PROG} page provides a \ac{REC MAX EO} flight
level to use as an altitude target. If the speed decay is modest, it may be
possible to alert \ac{ATC} before initiating the descent, but in-service events
have shown that speed decay is often very rapid, requiring descent initiation to
be prioritised.

Once drift down has been initiated, a decision needs to be made about speed. If
obstacles are a concern, the lowest drift down rate and highest ceiling are
achieved at green dot. Airbus refers to drifting down at green dot as ``Obstacle
strategy''. Flying at green dot reduces the chance of the \ac{FADEC}s
automatically relighting the failed engine as the engine will be windmilling
more slowly. Therefore, if obstacles are not a concern, M.78/300kt is flown, a
speed that will always fall within the stabilized windmill engine relight
envelope; Airbus refers to this as ``Standard Strategy''.

If obstacles remain a problem, \ac{MCT} and green dot speed can be maintained to
give a shallow climbing profile. Once obstacles are no longer a problem, descend
to \ac{LRC} ceiling (use \ac{V/S} if <500 fpm descent rate), engage the autothrust
and continue at \ac{LRC} speed.

\multicite{FCTM~PRO.AEP.ENG.EFDC}
\section{Single engine circling}

It may not be possible to fly level in the standard circling configuration of
\ac{CONF} 3, gear down. This can be ascertained by checking the table in the
\ac{QRH}\ \ecam{misc}{One Engine Inoperative -- Circling Approach} procedure. If
affected, plan a \ac{conf} 3 landing and delay gear extension until level flight
is no longer required; anticipate a \cphrase{L/G NOT DOWN} \ac{ecam} warning
below 750ft (which can be silenced with the \ac{EMER CANC} pb) and a
\ac{GPWS} ``Too Low Gear'' aural alert below 500ft \ac{RA}.

\multicite{QRH~AEP.MISC}

\section{Bomb on board}

The primary aim is to get the aircraft on the ground and evacuated \ac{ASAP}.

The secondary aim is to prevent detonation of the device. This is achieved by
preventing further increases in cabin altitude through the use of manual
pressure control and by avoiding sharp manoeuvres and turbulence.

The tertiary aim is to minimise the effect of any explosion. This is achieved by
reducing the diff to 1 psi. The method is to set cabin vertical speed to zero
using manual pressurisation control, then descend to an altitude 2500ft above
cabin altitude. As further descent is required, cabin vertical speed should be
adjusted to maintain the 1 psi diff for as long as possible. Automatic pressure
control is then reinstated on approach. Low speeds reduce the damage from an
explosion but increase the risk of a timed explosion occurring whilst airborne;
a compromise needs to be found. The aircraft should be configured for landing as
early as possible to avoid an explosion damaging landing systems.

In the cabin, procedures are laid down for assessing the risks of moving the
device and for moving the device to the \ac{LRBL} at door 2R.

\multicite{QRH~AER.80, FCOM~PRO.AER.MISC}

\section{Stall recovery \emph{(memory item)}}

Aerofoil stall is always and only an angle of attack issue. It is not possible
to directly prove an unstalled condition from attitude and airspeed data. The
flight recorders from the December 2014 Air Asia accident recorded an angle of
attack of 40\textdegree\ (i.e. around 25\textdegree\ greater than critical
angle) with both pitch and roll zero and speeds up to 160kt. Importantly, it is
perfectly possible to be fully stalled in the short term unreliable airspeed
configurations described in Section
\ref{sec-unreliable-airspeed}. Identification of a fully stalled condition is
thus largely dependent on identifying a high and uncontrollable descent rate
that does not correlate with normal flight path expectations for the attitude
and thrust applied.

To recover from a fully stalled condition, the angle of attack of the aerofoils
must be reduced to below critical. The generic stall recovery is therefore
simply to pitch the nose down sufficiently to break the stall and level the
wings. In normal operations, the velocity vector of the aircraft is around
3\textdegree{ }below the centerline of the aircraft (i.e. an attitude of around
3\textdegree{ }is required to fly level). In a stalled condition, the velocity
vector may be 40\textdegree{ }or more below the centerline of the
aircraft. Thus the amount of pitch down required to recover a fully stalled
aircraft can be 30\textdegree{ }or more.

In two recent Airbus accidents involving stalls, the lack of physical cross
coupling of sidesticks was a major factor. If one pilot elects to hold full back
sidestick, the aircraft cannot be recovered by the other pilot unless the
takeover button is used. With all the alarms, it would be easy to miss ``Dual
Input'' warnings, so \emph{always} press the takeover button.

The aircraft's thrust vector helps to accelerate the aircraft during the
recovery, and increasing speed along the aircraft's centerline acts to reduce
the stalled angle of attack. Thus, while thrust is not a primary means of
recovery, it does help. Unfortunately, Airbus have determined that due to the
pitch couple associated with underslung engines, there may be insufficient
longitudinal control authority to pitch the aircraft sufficiently to recover
from a stall if \ac{TOGA} is selected. It may therefore be necessary to
initially \emph{reduce} thrust to allow the primary recovery technique to be
applied; this is extremely counterintuitive.

Once there are no longer any indications of the stall, smoothly recover from the
dive, adjust thrust, check speedbrakes retracted and if appropriate (clean and
below 20,000ft) deploy the slats by selecting \ac{CONF}~1. The load factor
associated with an overly aggressive pull out can induce a secondary stall; on
the flip side, once reattachment of the airflow occurs, drag rapidly diminishes
and exceedance of high speed airframe limitations becomes a threat. A balance
needs to be found.

If a stall warner sounds on takeoff it is likely to be spurious since you are
almost certainly in normal law. The procedure in this case is essentially to
initially assume unreliable airspeed and fly \ac{TOGA}, 15\textdegree , wings level
until it can be confirmed that the warning is spurious.

A stall warning may occur at high altitude to indicate that the aircraft is
reaching $\mathrm{\alpha_{buffet}}$. In this case simply reduce the back
pressure on the sidestick and/or reduce bank angle.

\multicite{FCOM~PRO.AER.MISC}


\section{Computer reset}

Abnormal computer behaviour can often be stopped by interrupting the power
supply of the affected computer. This can be done either with cockpit controls
or with circuit breakers. The general procedure is to interrupt the power
supply, wait 3 seconds (5 seconds if a \ac{CB} was used), restore the power,
then wait another three seconds for the reset to complete. \inlcite{QRH
  AER.RESET} details the specific procedures for a variety of systems.

On the ground, almost all computers can be reset. \ac{MOC} can usually supply a
reset procedure if nothing applicable is available in the \ac{QRH}. The
exceptions are the \ac{ECU} and \ac{EIU} while the associated engine is running
and the \ac{BSCU} when the aircraft is not stopped.

In flight, only the computers listed in the \ac{QRH} should be considered for
reset.

\multicite{QRH~AER.SYSTEM~RESET}

\section{Landing distance calculations}

Many failures result in a longer than normal landing distance. The \ac{QRH}
inflight performance section has tables for calculating \V{APP} and Reference
Landing Distances for single failures. These reflect the performance achievable
in a typical operational landing without margin. easyJet requires a factor of
1.15 to be applied to these distances.

The \ac{EFB} module provides both factored and unfactored landing distances, and
also can calculate for multiple failures.

The safety factor may be disregarded in exceptional circumstances.

\multicite{QRH~IFP, FCOM~PER.LDG, EOMB~4.14.2}

\section{Abnormal V Alpha Prot}

If two or more angle of attack vanes become frozen at the same angle during
climb, a Mach number will eventually be reached such that the erroneous angle of
attack data indicates an incipient stall. When this happens, Normal Law high
angle of attack protection will activate. The flight computers' attempt to
reduce angle of attack will not, however, be registered by the frozen vanes,
leading to a continuous nose down pitch rate which cannot be overridden with
sidestick inputs.

Indications of this condition are available from the
$\alpha$\textsubscript{prot} and $\alpha$\textsubscript{max} strips. If the
$\alpha$\textsubscript{max} strip (solid red) completely hides the
$\alpha$\textsubscript{prot} strip (black and amber) or the
$\alpha$\textsubscript{prot} strip moves rapidly by more than 30kt during flight
manoeuvres with \ac{AP} on and speed brakes retracted, frozen angle of attack
vanes should be suspected.

The solution is to force the flight computers into Alternate Law where the
protection does not apply. This is most conveniently done by turning off any two
\ac{ADR}s. Once in Alternate Law, the stall warning strip (red and black)
becomes available. Since stall warning data also comes from the angle of attack
vanes, erroneous presentation is likely.

\section{Overspeed Recovery}

In general the response to an overspeed should be to deploy the speedbrake and
monitor the thrust reduction actioned by the autothrust. Disconnection of the
autopilot will not normally be required. If autothrust is not in use, the thrust
levers will need to be manually retarded.

It is possible that the autopilot will automatically disengage and high speed
protection will activate, resulting in an automatic pitch up. In this case,
smoothly adjust pitch attitude as required.

At high altitude, there is a threat of over-correction caused by the lethargic
response of the speedbrake when commanded to stow. In the worst case, a descent
may be required to recover speed. This threat can be mitigated by promptly
cancelling the speedbrake as soon as the overspeed condition ceases.

\multicite{FCTM~PRO.AER.MISC}

\section{Volcanic Ash Encounter}

Volcanic ash clouds are usually extensive, so a 180\textdegree\ turn will
achieve the quickest exit.

Air quality may be affected, so crew oxygen masks should be donned with 100\%
oxygen to exclude fumes. Passenger oxygen may also need to be deployed.

Probes may become blocked with ash, so be prepared to carry out the unreliable
speed procedure.

Disconnect the autothrust to prevent excessive thrust variations.

To minimise the impact on the engines, if conditions permit thrust should be
reduced. Turn on all anti-ice and set pack flow to high in order to increase
bleed demand and thus increase engine stall margin. Wing anti-ice will need to
be turned off again before attempting relight in case of flameout.

If engine \ac{EGT} limits are exceeded, consider a precautionary engine shutdown
with restart once clear of volcanic ash. Engine acceleration may be very slow
during restart. Since compressor and turbine blades may have been eroded, avoid
sudden thrust changes.

Damage to the windshield may necessitate an autoland or landing with a sliding
window open.

\multicite{QRH~AEP.MISC, FCOM~PRO.AEP.MISC, FCTM~PRO.AEP.MISC}

\chapter{Air con and pressurisation}

\section{Cabin overpressure}

There is no \ac{ECAM} in the case of total loss of pressure control leading to an
overpressure, so apply the \ac{QRH} procedure. The basic procedure is to reduce air
inflow by turning off one of the packs and put the avionics ventilation system
in its smoke removal configuration so that it dumps cabin air overboard. The
$\Delta$P is monitored, and the remaining pack is turned off if it exceeds 9
psi. 10 minutes before landing, both packs are turned off and remain off, and
the avionics ventilation is returned to its normal configuration.

\multicite{QRH~AEP.CAB~PR, FCOM~PRO.AEP.CAB~PR}


\section{Excess cabin altitude}

An \ac{ECAM} warning of excess (>9550ft) cabin altitude should be relied upon, even
if not backed up by other indications.

The initial response should be to protect yourself by getting an oxygen mask
on. Initiate a descent; if above \ac{FL}160, this should be in accordance with
the Emergency Descent procedure (see Section \ref{sec-emer-descent}). Once the
descent is established and all relevant checklists are complete, check the
position of the outflow valve and, if it is not fully closed, use manual control
to close it.

\multicite{\uline{CAB PR}~EXCESS~CAB~ALT, FCOM~PRO.AEP.CAB~PR}


\section{Landing Elevation Fault}

If the landing field elevation is not available from the \ac{FMGS}, the landing
elevation must be manually selected. This is done by pulling out and turning the
\ac{LDG} \ac{ELEV} knob. The scale on the knob is only a rough indication; use the \ac{LDG}
\ac{ELEV} displayed on either the \ac{CRUISE} page or the \ac{CAB} \ac{PRESS} \ac{SD} page instead.

\multicite{\uline{CAB PR}~LDG~ELEV~FAULT, FCOM~PRO.AEP.CAB~PR}


\section{Pack fault}
\label{sec-pack-fault}

The \ecam{air}{PACK FAULT} \ac{ECAM} indicates that the pack flow control valve
position disagrees with the selected position or that the pack valve has closed
due to either compressor outlet overheat or pack outlet overheat.

The affected pack should be turned off.

A possible reason for this failure is loss of both channels of an Air
Conditioning System Controller (\ac{ACSC}). If this occurs, the associated hot
air trimming will also be lost (cockpit for \ac{ACSC} 1, cabin for \ac{ACSC} 2).

If there are simultaneous faults with both packs, ram air must be used. This
will necessitate depressurisation of the aircraft, so a descent to \ac{FL}100 (or \ac{MEA}
if higher) is required. If a \ac{PACK} button \ac{FAULT} light subsequently extinguishes,
an attempt should be made to reinstate that pack.

\multicite{\uline{AIR}~PACK~1(2)(1+2)~FAULT, FCOM~PRO.AEP.AIR}


\section{Pack overheat}

The associated pack flow control valve closes automatically in the event of a
pack overheating (outlet temp~>260\textdegree C or outlet temp >230\textdegree
C four times in one flight). The remaining pack will automatically go to high
flow, and is capable of supplying all of the air conditioning requirement. This
system's automatic response is backed up by turning off the pack. The \ac{FAULT}
light in the \ac{PACK} button remains illuminated whilst the overheat condition
exists. The pack can be turned back on once it has cooled.

\multicite{\uline{AIR}~PACK~1(2)~OVHT, FCOM~PRO.AEP.AIR}


\section{Pack off}

A warning is generated if a functional pack is selected off in a phase of flight
when it would be expected to be on. This is usually the result of neglecting to
re-instate the packs after a packs off takeoff. Unless there is a reason not to,
turn the affected pack(s) on.

\multicite{\uline{AIR}~PACK~1(2)~OFF, FCOM~PRO.AEP.AIR}


\section{Pack regulator faults}

A regulator fault is defined as a failure of one of four devices: the bypass
valve, the ram air inlet, the compressor outlet temperature sensor or the flow
control valve. The \ac{ECAM} bleed page can be used to determine which device is at
fault.

Regardless of the device at fault, the ramification is the same; the pack will
continue to operate but there may be a degradation in temperature regulation. If
temperatures become uncomfortable, consideration should be given to turning off
the affected pack.

\multicite{\uline{AIR}~PACK~1(2)~REGUL~FAULT, FCOM~PRO.AEP.AIR}


\section{ACSC single lane failure}

Each \ac{ACSC} has two fully redundant ``lanes'', so loss of a single ``lane'' results in
loss of redundancy only.

\multicite{\uline{AIR}~COND~CTL~1(2)~A(B)~FAULT, FCOM~PRO.AEP.AIR}


\section{Duct overheat}

A duct overheat is defined as a duct reaching 88\textdegree C or a duct reaching
80\textdegree C four times in one flight. If this occurs, the hot air pressure
regulating valve and trim air valves close automatically and the \ac{FAULT} light
illuminates in the \ac{HOT} \ac{AIR} button. This light will extinguish when the
temperature drops to 70\textdegree C.

Once the duct has cooled, an attempt can be made to recover the hot air system
by cycling the \ac{HOT} \ac{AIR} button.

If recovery is not possible, basic temperature regulation will continue to be
provided by the packs.

\multicite{\uline{COND}~FWD~CAB/AFT~CAB/CKPT~DUCT~OVHT, FCOM~PRO.AEP.COND}


\section{Hot air fault}

If the hot air pressure regulating valve is not in its commanded position, the
effects will depend on its actual position.

If it is closed when commanded open, the packs will provide basic temperature
regulation.

More serious is if it has been commanded closed in response to a duct overheat
and it remains open. Manual control may be effective, but if it is not the
only option is to turn off both packs and proceed as per Section
\ref{sec-pack-fault}.

\multicite{\uline{COND}~HOT~AIR~FAULT, FCOM~PRO.AEP.COND}


\section{Trim air faults}

Either a fault with one of the trim air valves or an overpressure downstream of
the hot air valve. An associated message indicates which condition exists.

Failure of a trim valve leads to loss of optimised temperature regulation for
the corresponding zone; basic temperature regulation is still available.

The \ac{TRIM} \ac{AIR} \ac{HIGH} \ac{PR} message may be disregarded if triggered when all the trim
air valves are closed. This occurs during the first 30 seconds after the packs
are selected on and in flight if all zone heating demands are
fulfilled.\footnote{The \ac{FCOM} is not very informative regarding response to
overpressure when this does not apply. However the \ac{MEL} operating procedures for
dispatch with this condition indicate that turning the \ac{HOT} \ac{AIR} pb-sw off is
probably a good idea.}

\multicite{\uline{COND}~TRIM~AIR~SYS~FAULT, FCOM~PRO.AEP.COND}


\section{Cabin fan faults}

If both cabin fans fail, their flow should be replaced by increasing the pack
flow to \ac{HI}.

\multicite{\uline{COND}~L~+~R~CAB~FAN~FAULT, FCOM~PRO.AEP.COND}


\section{Lavatory and galley fan faults}

The cabin zone temperature sensors are normally ventilated by air extracted by
these fans. Loss of the fans therefore leads to loss of accurate zone
temperature indication.

On older aircraft, temperature control reverts to maintenance of a fixed cabin
zone inlet duct temperature of 15\textdegree C.

On newer aircraft the temperature controls for the cabin revert to controlling
temperature in the ducts. If \ac{ACSC} 2 has also failed, the duct temperatures are
maintained at the same level as the cockpit duct temperature, and may therefore
be controlled with the cockpit temperature selector.

\multicite{\uline{COND}~LAV~+~GALLEY~FAN~FAULT, FCOM~PRO.AEP.COND}



\section{Pressure controller faults}
\label{sec-pressure-controller}

Loss of a single cabin pressure controller leads to loss of redundancy only.

If both pressure controllers are lost, use manual control. The outflow valve
reacts slowly in manual mode, and it may be 10 seconds before positive control
of the outflow valve can be verified. It may also react too slowly to prevent a
temporary depressurisation.

To activate manual pressurisation control, press the \ac{MODE SEL} button. This
allows the \ac{MAN V/S CTL} toggle switch to directly control the outflow
valve. Moving the toggle to \ac{DN} closes the outflow valve causing the cabin
altitude to descend, whilst moving the toggle to \ac{UP} opens the outflow valve
causing the cabin altitude to climb. The target climb and descent rates are
500fpm and 300fpm, these being displayed on the status page for easy reference.

A table of \ac{FL} versus \ac{CAB ALT TGT} is also provided on the status page;
no guidance is given for the interpretation of this table. The final action of
the procedure is to fully open the outflow valve when reaching 2500ft \ac{AGL}
in preparation for an unpressurised landing, so to avoid large pressurisation
changes during this action, the final cabin altitude target needs to be
aerodrome elevation plus 2500ft. This gives an indication of how \ac{CAB ALT
  TGT} should be interpreted: it is the lowest cabin altitude that still results
in a safe $\Delta$P at a given \ac{FL}. A cabin altitude greater then \ac{CAB
  ALT TGT} is always acceptable\footnote{A reasonable maximum cabin altitude is
8800ft, which is when the \ac{CAB ALTITUDE} advisory triggers.} and, moreover,
for the final stages of the approach, it is necessary. The method is therefore
to avoid cabin altitudes below \ac{CAB ALT TGT} for your current \ac{FL} while
ensuring that a cabin altitude of aerodrome elevation plus 2500ft will be
achieved by the time you need to fully open the outflow valve.

Ensure cabin diff pressure is zero before attempting to open the doors.

\multicite{\uline{CAB~PR}~SYS~1(2)(1+2)~FAULT, FCOM~PRO.AEP.CAB~PR}


\section{Low diff pressure}

High rates of descent may lead to the aircraft descending through the cabin
altitude when more than 3000ft above the landing altitude. An \ac{ECAM} warning
indicates that this situation is projected to occur within the next
1\textonehalf minutes. If the rate of descent of the aircraft is not reduced,
the pressure controllers will have to resort to high rates of change of cabin
altitude, which may cause passenger discomfort. The aircraft's vertical speed
should be reduced unless there is a pressing reason not to.

\multicite{\uline{CAB~PR}~LO~DIFF~PR, FCOM~PRO.AEP.CAB~PR}


\section{Outflow valve closed on ground}

If the outflow valve fails to automatically open on the ground, manual control
should be attempted. If that doesn't work, depressurise the aircraft by turning
off both packs.

\multicite{\uline{CAB~PR}~OFV~NOT~OPEN, FCOM~PRO.AEP.CAB~PR}



\section{Open safety valve}

There are safety valves for both cabin overpressure and negative differential
pressure; the associated \ac{ECAM} message does not distinguish between the two.

If diff pressure is above 8psi, it is the overpressure valve that has
opened. Attempt manual pressurisation control and if that fails, reduce aircraft
altitude.

If diff pressure is below zero, it is the negative differential valve. Reduce
aircraft vertical speed or expect high cabin rates.

\multicite{\uline{CAB~PR}~SAFETY~VALVE~OPEN, FCOM~PRO.AEP.CAB~PR}

\chapter{Avionics Ventilation}

\section{Blower fault}

Defined as low blowing pressure or duct overheat. Unless there is a \ac{DC} \ac{ESS} Bus
fault, the blower fan should be set to \ac{OVRD}. This puts the avionics ventilation
into closed configuration and adds cooling air from the air conditioning
system.%<!-- {TODO:investigate --> <!-- involvement of DC ESS BUS fault} -->

\multicite{\uline{VENT}~BLOWER~FAULT, FCOM~PRO.AEP.VENT}


\section{Extract fault}

Defined as low extract pressure. The extract fan should be put in \ac{OVRD}. This
puts the avionics ventilation into closed configuration and adds cooling air
from the air conditioning system.

\multicite{\uline{VENT}~EXTRACT~FAULT, FCOM~PRO.AEP.VENT}


\section{Skin valve fault}

Defined as one of three faults: the inlet valve is not fully closed in flight;
the extract valve is fully open in flight; or the extract valve did not
automatically close on application of take-off power. The \ac{ECAM} \ac{Cab
  Press} page will differentiate.

If the fault is with the inlet valve, no action is required since it
incorporates a non-return valve.

If the extract valve is affected, the system should be put into smoke
configuration; this sends additional close signals to the extract valve. If the
extract valve still remains open, the \ac{ECAM} directs the crew to depressurise the
aircraft. The rationale for this seemingly extreme reaction to a relatively
minor issue is that the \ac{ECAM} can only really occur immediately after the
take-off inhibit ceases at 1500ft \ac{AAL}. The extract valve is normally held closed
by the pressurisation and its motor is not sufficiently powerful to overcome
this. Thus the extract valve can only be open in flight if it never closed. With
the extract valve open, it will likely not be possible to complete the flight
since the additional hole will make it impossible to properly pressurise the
aircraft at cruise altitude, and the pressurised air rushing through the open
outflow valve will cause it to be unpleasantly noisy in the cockpit. This makes
depressurising the aircraft and returning for engineering attention the obvious
solution.\footnote{The \ac{ECAM} procedure associated with this failure is due to be
modified in an upcoming \ac{FWC} update.}

\multicite{\uline{VENT}~SKIN~VALVE~FAULT, FCOM~PRO.AEP.VENT}


\section{Avionics ventilation system fault}

Defined as either a valve not in its commanded position or the Avionics
Equipment Ventilation Controller (\ac{AEVC}) being either unpowered or failing its
power-up test. The system will automatically default to a safe configuration
similar to smoke configuration. No crew action is required.

\multicite{\uline{VENT}~AVNCS~SYS~FAULT, FCOM~PRO.AEP.VENT}

\chapter{Electrical}

\section{Emergency configuration}
\label{sec-emerg-elec}

Attempt to restore normal power by recycling the main generators. If that fails,
try again after splitting the systems with the \ac{BUS TIE} button.

If normal power cannot be restored, ensure that the emergency generator is on
line (deploy the \ac{RAT} manually if required) and maintain speed >140kt to avoid
\ac{RAT} stall. Cycling \ac{FAC}1 will recover rudder trim. Once 45 seconds have elapsed
and when below \ac{FL}250, the \ac{APU} can be started.

So much equipment is lost in the emergency configuration that \ac{QRH}
\ecam{ELEC}{emer config sys remaining} provides a table of \emph{surviving}
equipment. Notable losses are:

\begin{itemize}
\item All the fuel pumps, requiring Gravity Fuel Feeding procedures (see Section
  \ref{sec-gravity-fuel}) and making center tank fuel unusable.

\item The anti-skid, three fifths of the spoilers and the reversers. Combined
  with the higher landing speeds required to prevent \ac{RAT} stall this results
  in significantly increased landing distances.

\item Alternate Law with reduced protections (see Section
  \ref{sec-alternate-law}). Mechanical yaw becomes Alternate Law yaw with
  \ac{FAC}1 reset. Anticipate Direct Law at gear extension.

\item Anti-icing for probes supplying \ac{ADR}2 and \ac{ADR}3. Only \ac{CM}1
  instruments should be considered reliable in icing conditions.

\item Nose Wheel Steering (see Section \ref{sec-nose-wheel-steering}).
\end{itemize}

The \ac{QRH} \ecam{ELEC}{elec emer config summary} should be applied once
\ac{ECAM} actions are complete.

\multicite{\uline{ELEC}~EMER~CONFIG, QRH~AEP.ELEC, FCOM~PRO.AEP.ELEC}


\section{Battery only}

Battery power is available for approximately 30 mins.\footnote{This information was
part of Airbus \ac{CBT} training. There is no figure available in the \ac{FCOM}.}

\ac{QRH} \ecam{ELEC}{emer config sys remaining} provides details of remaining
equipment. Mostly this is the same as for emergency electrical configuration (see
Section \ref{sec-emerg-elec}). Notable additional losses are:

\begin{itemize}
\item The remaining \ac{ND}
\item The remaining \ac{FMGC} and \ac{MCDU}
\item The remaining \ac{DME} and transponder
\item The remaining \ac{FAC}
\item Wing anti-ice
\item \ac{CM}1 \ac{AOA} anti-ice
\item Passenger oxygen masks
\end{itemize}

An attempt should be made to bring the emergency generator on line with the
\cphrase{EMER ELEC PWR MAN ON} button.

\multicite{\uline{ELEC}~ESS~BUSES~ON~BAT, QRH~AEP.ELEC, FCOM~PRO.AEP.ELEC}


\section{IDG low oil pressure/ high oil temperature}
\label{sec-idg}

The \ac{IDG} should be disconnected. Assuming the associated engine is running,
press the \ac{IDG} button until the \ac{GEN} \ac{FAULT} light comes on. Do not
press the button for more than \strong{3 seconds}.

The \ac{APU} generator should be used if available.

\multicite{\uline{ELEC}~IDG~1(2)~OIL~LO~PR/OVHT, FCOM~PRO.AEP.ELEC}


\section{Generator fault}

Try to reset the generator by turning it off, then, after a short pause, turning
it on again. If unsuccessful, turn it back off.

If an engine driven generator cannot be recovered, the \ac{APU} generator should be
used if available.

Single generator operation leads to shedding of the galley. Loss of an engine
driven generator leads to loss of \cphrase{CAT 3 DUAL} capability.

\multicite{\uline{ELEC}(APU)~GEN~(1)(2)~FAULT, FCOM~PRO.AEP.ELEC}


\section{Battery fault}

The affected battery contactor opens automatically. \ac{APU} battery start is
unavailable with a single battery.

\multicite{\uline{ELEC}~BAT~1(2)~FAULT, FCOM~PRO.AEP.ELEC}


\section{AC Bus 1 fault}

Some or all of the equipment on \ac{AC} bus 1 becomes unavailable, including
\ac{TR}1: \ac{DC} bus 1 is powered from \ac{DC} bus 2 via the battery bus.

Power must be re-routed to the essential \ac{AC} bus via \ac{AC} bus 2. This is
automatic on some aircraft. Manual re-routing is achieved with the \ac{AC ESS
  FEED} button. Once essential \ac{AC} is powered, the essential \ac{TR} powers
the \ac{DC} essential bus.

Notable lost equipment includes the blue hydraulic system, \ac{RA}1 (and hence
\cphrase{Cat 3} capability), half the fuel pumps, the nose wheel steering, the
avionics blower fan and \ac{CM}1 windshield heat.

\multicite{\uline{ELEC}~AC~BUS~1~FAULT, FCOM~PRO.AEP.ELEC}


\section{AC Bus 2 fault}

Some or all of the equipment on \ac{AC} bus 2 becomes unavailable, including
\ac{TR}2: \ac{DC} bus 2 is powered from \ac{DC} bus 1 via the battery bus.

The majority of this equipment has a redundant backup, the loss of the \ac{FO}’s
\ac{PFD} and \ac{ND} and a downgrade to \cphrase{Cat 1} being the major
issues. Landing distances are unchanged.

\multicite{\uline{ELEC}~AC~BUS~2~FAULT, FCOM~PRO.AEP.ELEC}


\section{AC Ess Bus fault}
\label{sec-ac-ess}

It may be possible to recover the bus by transferring its power source to \ac{AC} bus
2 with the \ac{AC ESS FEED} button. If this is unsuccessful, some or all of the
equipment on the \ac{AC} essential bus will be lost.

The majority of this equipment has a redundant backup, with the loss of the
Captain’s \ac{PFD} and \ac{ND} and a downgrade to \cphrase{Cat 1} being the major
issues. Landing distances are unchanged.

It is worth noting that loss of \ac{AC} essential bus implies loss of passenger
oxygen masks. The \ac{MEL} allows dispatch without operative passenger oxygen
masks provided that operating altitude is limited to 10,000ft. Where possible, a
descent to this altitude would seem appropriate.

Where it is not possible to immediately descend to 10,000ft, a compromise level
whereby a descent to a safe altitude can be achieved without masks needs to be
chosen.  The main form of guidance on altitude hypoxia comes in the form of
“Time of useful consciousness” tables. Working on the principal that if you
remain conscious you definitely remain alive, 25,000ft would seem to be a
reasonable compromise. This gives you 2 to 3 minutes of useful consciousness to
dive to 18,000ft, where you would then have 30 minutes to clear any
terrain.\footnote{These tables are obviously not designed to be used in this way
– the exposure to hypoxia in the descent will likely impact the \ac{TOUC} at
18,000ft, and we are really more concerned with survivability than useful
consciousness – but they can at least give a feeling for the parameters
involved.}

\multicite{\uline{ELEC}~AC~ESS~BUS~FAULT, FCOM~PRO.AEP.ELEC}


\section{AC Essential Shed Bus lost}

Some or all of the equipment on the \ac{AC} \ac{ESS} \ac{SHED} bus is lost. The
major issue is the loss of the passenger oxygen masks (see discussion in Section
\ref{sec-ac-ess}). Landing distances are unchanged.

\multicite{\uline{ELEC}~AC~ESS~BUS~SHED, FCOM~PRO.AEP.ELEC}


\section{DC Bus 1 fault}

Some or all of the equipment on \ac{DC} bus 1 is lost. Most of the equipment loss
causes loss of redundancy only. Landing distances are unchanged.

\multicite{\uline{ELEC}~DC~BUS~1~FAULT, FCOM~PRO.AEP.ELEC}


\section{DC Bus 2 fault}

Some or all of the equipment on \ac{DC} bus 2 is lost. Notable items are:

\begin{itemize}
\item 3 spoilers per side and one reverser. Landing distances increase by
  $\sim$35\%.
\item Autobrake.
\item \ac{FO}'s static probe: select \ac{ADR}3 to \ac{FO}'s side.
\item \ac{FO}'s window heat, wipers and rain repellent.
\end{itemize}

The other lost systems either have redundant backups or are non-essential. It
should, however, be noted that the only flight computers remaining are
\ac{ELAC}1, \ac{SEC}1 and \ac{FAC}1, so not much redundancy remains.

\multicite{\uline{ELEC}~DC~BUS~2~FAULT, FCOM~PRO.AEP.ELEC}


\section{DC Essential Bus fault}

The major headache associated with \ac{DC} essential bus failure is a
significant loss of redundancy in communications systems.

\ac{ACP}1 and \ac{ACP}2 are lost, along with \ac{VHF}1. This allows two-way
communication to be recovered by one pilot using \ac{ACP}3 (selected via the
\ac{AUDIO SWTG} rotary selector) with \ac{VHF}2 or \ac{VHF}3. Since speaker 1 is
also lost, having \ac{CM}2 handle the radios with speaker 2 at high volume is
the only method of both pilots having awareness of \ac{ATC} communications.

On some airframes,\footnote{\ac{MSN}s 2184–2402 and \ac{MSN}s 2471–3122 at time
of writing.}\ this loss of communications is exacerbated by a design flaw: the
audio cards for cockpit mikes and headsets are \emph{all} powered from the
\ac{DC} essential bus. It may still be possible to receive transmissions with a
combination of \ac{VHF}2/3, \ac{ACP}3 on \ac{FO} and speaker 2, but transmission
is limited to morse code on the transmit switch.\footnote{I am only inferring
this solution from the list of equipment lost; there is no specific
documentation to indicate that it will work.}

Other notable lost equipment includes:

\begin{itemize}
\item Reverser 2 and the blue hydraulic system, leading to modestly increased
  landing distances.
\item The \ac{HP} fuel shutoff valves. This requires that the engines are shut
  down with the Engine Fire pushbutton switches.
\item Wing anti-ice.
\item Auto-thrust. This is notable because the \ac{ECAM} status page incorrectly
  reports that \cphrase{Cat 3 single} is available, when the actual landing capability is
  \cphrase{Cat 2}.
\item \ac{GPWS}.
\end{itemize}

\multicite{\uline{ELEC}~DC~ESS~BUS~FAULT, FCOM~PRO.AEP.ELEC}


\section{DC Essential shed}

The only major issue is the loss of wing anti-ice. Therefore, avoid icing
conditions, and apply landing distance procedure if ice accretes.

\multicite{\uline{ELEC}~DC~ESS~BUS~SHED, FCOM~PRO.AEP.ELEC}


\section{Loss of DC Bus 1 and DC Bus 2}

Some or all of the systems supplied by \ac{DC} bus 1 and \ac{DC} bus 2 are lost.

The implications are the same as for \ac{DC} emergency configuration (see
Section \ref{sec-dc-emer-conf}) except that the \ac{RAT} will not need to be
deployed since \ac{ESS TR} will be supplied from \ac{AC} bus 1 instead of the
emergency generator.

\multicite{\uline{ELEC}~DC~BUS~1+2 FAULT, FCOM~PRO.AEP.ELEC}


\section{Generator overload}

Shed some load by switching off the galleys.

\multicite{\uline{ELEC}~GEN~1(2)~OVERLOAD, \uline{ELEC}~APU~GEN~OVERLOAD,\\
  FCOM~PRO.AEP.ELEC}


\section{Loss of TRs}

\looseness=1 No systems are lost as a result of failure of a single \ac{TR}. If
the fault is with \ac{TR}1 or \ac{TR}2, \ac{DC ESS} will be supplied by the
\ac{ESS TR} via the \ac{AC} essential bus; in this case only \cphrase{Cat 3
  single} will be available.

\looseness=1
Dual failures involving the \ac{ESS TR} are similar to single failures, except
redundancy is further compromised. If, however, both \ac{TR}1 and \ac{TR}2 fail
then \ac{DC} bus 1, \ac{DC} bus 2 \emph{and} the \ac{DC} battery bus will be
lost. The \ac{DC} essential bus will remain powered by the \ac{ESS TR}.

The \ac{FCOM} is not very forthcoming regarding this failure; there is only a
description in the \ac{DSC} section and nothing specific in the \ac{PRO}
section. The situation is, however, very similar to the “\ac{ECAM} complete”
phase of \ac{DC} emergency configuration (see Section \ref{sec-dc-emer-conf}),
the only difference being that \ac{AC} bus 1 rather than the emergency generator
is providing the power to the \ac{ESS TR}, and hence you don’t need to worry
about \ac{RAT} deployment.

\multicite{\uline{ELEC}~TR~1(2), \uline{ELEC}~ESS~TR~FAULT, FCOM~PRO.AEP.ELEC}


\section{Battery bus fault}

Some or all of the equipment on the battery bus is lost. The only notable losses
are \ac{APU} fire detection and \ac{APU} battery start.

\multicite{\uline{ELEC}~DC~BAT~BUS~FAULT, FCOM~PRO.AEP.ELEC}


\section{DC Emergency configuration}
\label{sec-dc-emer-conf}
Defined as the loss of \ac{DC} busses 1 and 2, the \ac{DC} essential bus and the
battery bus. Recovery assumes that the \ac{DC} essential bus can be fully
restored by deploying the \ac{RAT} with the \ac{EMER ELEC PWR} button.

Equipment powered or controlled through \ac{DC} busses 1 and 2 and the battery
bus is therefore lost.

Loss of the equipment associated with the battery bus is fairly benign: mainly
\ac{APU} battery start is unavailable due to loss of the \ac{APU ECB} and
\ac{APU} fire detection.

A lot of equipment is lost with the loss of \ac{DC} busses 1 and 2, but it is
worth remembering that all three affected busses are also lost in emergency
electrical config. Thus, you will have, at minimum, all the equipment listed in
\ac{QRH} \ecam{ELEC}{emer config sys remaining}. The main items of note are:

\begin{itemize}

\item A long runway is required. Minimum \V{APP} is 140kt to prevent \ac{RAT}
  stall. Antiskid, reversers and 60\% of the spoilers are lost. Loss of
  nosewheel steering adds to the difficulty. Braking is from the \ac{ABCU}, so only
  manual braking is available.\footnote{easyJet aircraft automatically modulate
  to 1000psi, but the sim may not.}

  \item Flight computer redundancy is significantly reduced, with only
    \ac{ELAC}1, \ac{SEC}1 and \ac{FAC}1 available. This is, however, sufficient
    to keep Normal Law, so a \ac{CONF} Full landing and, indeed, \cphrase{Cat 3
      Single} autoland (using \ac{AP}1 and \ac{FMGS}1) are available.

  \item Pressurised fuel is available from the \#1 wing tank pumps, but center
    tank fuel is unusable.

  \item Redundancy in the pressurisation system is seriously compromised. Manual
    pressure control and \ac{CPC}2 are lost, so you are reliant on \ac{CPC}1 for
    control. Pack 2, \ac{BMC}2 and cross bleed control are all lost, so you are
    reliant on \ac{ENG}1 bleed and pack 1 for supply. Ram air remains available, so
    if required a slow depressurisation through turning off pack 1 followed by a
    depressurised landing with ram air is achievable.

  \item Communications are limited to \ac{VHF}1, controlled by \ac{RMP}1. All
    communications will be lost below 100kt on the landing roll when the \ac{DC}
    essential bus is depowered.

  \item Redundancy in the fire detection and suppression systems is
    compromised. The engines each retain one detection loop and one fire
    bottle. \ac{APU} fire detection and cargo fire extinguishing are lost.

  \item Heating for all static ports is lost, so be alert for unreliable
    airspeed and altitude.
\end{itemize}

\multicite{\uline{ELEC}~DC~EMER~CONFIG, FCOM~PRO.AEP.ELEC}


\section{Static inverter fault}

Normal operations are not affected.

\multicite{\uline{ELEC}~STAT~INV~FAULT, FCOM~PRO.AEP.ELEC}


\section{Generator 1 line off}

Pressing the \ac{GEN} 1 \ac{LINE} button on the emergency electrical panel has much the
same effect as pressing the \ac{GEN}1 button on the main electrical panel, with the
difference that \ac{GEN}1 continues to supply its associated fuel pumps. It is
primarily used for the smoke drill. If it’s not meant to be off, turn it on.

\multicite{\uline{ELEC}~EMER~GEN~1~LINE~OFF, FCOM~PRO.AEP.ELEC}


\section{Tripped circuit breakers}

It is generally not recommended to reset circuit breakers in flight. It is,
however, acceptable to attempt a single reset if it is judged necessary for the
safe continuation of the flight.

On the ground, any circuit breakers other than those for the fuel pumps may be
reset as long as the action is coordinated with \ac{MOC}.

An \ac{ECAM} warning will be triggered if a green circuit breaker trips. If the
circuit breaker is left tripped, additional tripped circuit breakers on the same
panel will not be detected.

\multicite{\uline{C/B}~TRIPPED, FCOM~PRO.AEP.ELEC}


\chapter{Flight Controls}


\section{Flaps and/or slats fault/locked}
\label{sec-stuck-slats/flaps}

The most pressing concern following a flap or slat problem is to select a safe
initial speed that will avoid overspeeding the stuck device. To establish a
conservative \V{max}:

\begin{itemize}
\item If the failure occurred while \emph{deploying} surfaces, use the placarded
  \V{FE} associated with the \emph{new} flap lever position. The \V{max}
  displayed on the \ac{PFD} will be this \V{FE}.\footnote{The \V{max} displayed
  on the \ac{PFD} is a function of flap lever position. Conversely, the
  \ac{ecam} \ac{overspeed} warning is a function of actual flap and slat
  positions.}
\item If the failure occurred while \emph{retracting} surfaces, use the
  placarded \V{FE} associated with the \emph{previous} flap lever
  position. Consider returning the flap lever to this position to get this
  \V{FE} displayed on the \ac{PFD}.
\end{itemize}

For minimum speed, the \V{LS} displayed on the \ac{PFD} is calculated from
actual flap and slat position and can be trusted. If \V{LS} is unavailable, fly
\V{max}~$-$~5kt.

Unless there is an obvious reason not to (e.g. wing tip brake on, alignment
fault or fault due to dual hydraulic failure), the flap lever can be recycled.

If normal operation cannot be restored, there are two major issues that must be
quickly addressed:

\begin{itemize}
\item Fuel burn will be dramatically higher when flying with a locked
  device. With slats extended, fuel burn will increase by 60\%. With flaps
  extended it will increase by 80\%. With \emph{both} slats and flaps extended,
  fuel burn will double. These figures are available in \cphrase{QRH~OPS}
\item Landing distances are significantly increased, in the worst case by a
  factor of 2.2
\end{itemize}

It may be that the combination of these factors requires a prompt diversion
decision.

The flap and slat systems are largely independent, so the flap lever will
continue to move the slats if the flaps are locked and vice versa. In general,
\ac{conf} 3 should be selected for landing. There are two exceptions. If flaps
are locked at >3, \ac{conf} Full should be used. If both slats and flaps are
locked at 0, \ac{conf} 1 should be used so that the \ac{AP/FD} go-around is
armed.

Configurations and \V{REF} increments are available in the \ac{QRH} or from the
\ac{EFB}. If a flapless and slatless landing is required, the threshold speed
may be below \V{LS}. This is necessary as the landing speeds in this
configuration are very close to tyre limit speeds.

Generally, the deployment method is to select a speed 5kt below \V{FE~NEXT},
then select the next configuration as the speed reduces \emph{through}
\V{FE~NEXT}.\footnote{\V{FE~NEXT} is a function of flap lever position, so it
takes no account of the failure. Since it is likely that for any given
configuration there are fewer high lift surfaces than normal deployed, the next
configuration is selected as soon as possible, even at the risk of turbulence
causing a slight overspeed. If \V{FE~NEXT} is not available, placarded speeds
can be used instead.}

In the case where \V{LS}>\V{FE~NEXT}, prioritise \V{LS}: fly \V{LS}, select the
next configuration, then track \V{LS} as it reduces with the extension of the
lift device. Use of autothrust with selected speed is generally recommended for
all phases of the approach, but in this case it will need to be temporarily
disconnected.

The autopilot may be used down to 500ft \ac{AAL}, but since it is not tuned for
the abnormal configuration it must be closely monitored.

Anticipate an unusual visual picture for landing.

For the go-around, initially maintain flap/slat configuration. A speed 10kt
lower than max operating speed should be flown. If it is the slats that are
jammed or if the flaps are jammed at 0, \ac{conf}~0 can be selected for transit
to a diversion airfield. To clean up, accelerate to \V{Max} $-$ 10kt before
selecting each new configuration. Transit should be at \V{Max} $-$ 10kt unless
clean.

Other issues include the possible loss of the automatic operation of the center
tank pumps (which is sequenced to the slats) and possible reversion to Alternate
Law.

It is also worth noting that failure of the slat channels of both \ac{SFCC}s
appears to result in the loss of characteristic speed display on both
\ac{PFD}s. This is not mentioned in the \ac{FCOM} but occurs in the sim.

The upshot of this is that neither \V{LS} nor \V{SW} are available at all, since
they are not displayed and there is no way to calculate them. This is of
particular concern when trying to deploy \ac{conf}~2 on the approach: \ac{conf}
1 does not extend flaps, so the aircraft will still be clean, yet it must be
slowed to the \V{FE} for \ac{conf}~2. It is highly likely that the stall warner
will activate during this transition, and if not anticipated, the subsequent
recovery will overspeed the flaps.

The solution is to brief that speed will be reduced very slowly and if the stall
warning occurs the speed will be maintained whilst allowing the deployment of
the flaps to recover the stall margin.

\multicite{\uline{F/CTL}~FLAPS(SLATS)~FAULT(LOCKED), QRH AEP.F/CTL,\\
  FCOM~PRO.AEP.F/CTL}


\section{Direct Law}
\label{sec-direct-law}

In Direct Law, deflection of the control surfaces is a linear function of
deflection of the side-stick and trimming must be done manually. The controls
are very sensitive at high speeds.

Use of manual thrust is recommended as power changes will result in pitch
changes. Similarly, use of the speed brake will result in nose up pitch changes
so it should be used with care.

Protections are unavailable, so speed is limited to 320kt/0.77M and great care
must be taken when flying \ac{GPWS} or windshear manoeuvres.

Direct Law landings are \ac{conf} 3; landing distances, in the absence of other
pertinent failures, are comparable to normal \ac{conf} 3 landings.

The major handling difficulty with Direct Law is the go-around. There is no
compensation for the large pitch moment introduced by selecting \ac{TOGA} power
on the under-slung engines, the thrust levers are non-linear, thrust onset is
laggy and non-linear and the use of the manual pitch trim wheel will be
unfamiliar. Apply power smoothly and progressively and anticipate a requirement
for unusual side-stick inputs.

Direct Law works with or without yaw dampers. The aircraft is \emph{always}
convergent in dutch roll, so use lateral control, not rudder, if dutch roll is
experienced.

\multicite{\uline{F/CTL}~DIRECT~LAW, FCOM~PRO.AEP.F/CTL}


\section{Alternate Law}
\label{sec-alternate-law}

\looseness=1
In Alternate Law, pitch is as in Normal Law, but roll is as in Direct Law. Load
factor protection is retained, but other protections are either replaced with
static stability or are lost, depending on the nature of the failure. Stall
warnings and overspeed warnings become active.

The main effects are that speed is limited to 320kt and stall warnings must be
respected when carrying out \ac{EGPWS} manoeuvres.

The autopilot may be available.

Expect Direct Law after landing gear extension (see Section
\ref{sec-direct-law}), and hence increased approach speeds and landing distances
due to the associated \ac{CONF} 3 landing.

\multicite{\uline{F/CTL}~ALTN~LAW, FCOM~PRO.AEP.F/CTL}


\section{Elevator faults}

If a single elevator fails, the \ac{SEC}s use the remaining elevator to provide
pitch control in Alternate Law (see Section \ref{sec-alternate-law}). In
addition, speed brake should not be used and the autopilots are unserviceable.

If both elevators fail, the only available mechanism for pitch control is manual
pitch trim, so pitch reverts to mechanical back up and roll reverts to Direct
Law. For the approach fly a long final, initiating the descent from at least
5000ft \ac{AAL}. Do not try to flare using trim and do not remove power until
after touchdown. From 1000ft \ac{AAL}, try to keep power changes to within 2\%
N1. In the event of a go-around, power \emph{must} be applied very slowly if
control is not to be lost.\footnote{This is Boeing advice -- Airbus does not
provide guidance for the flare or go-around technique when elevators are
frozen.}

\multicite{\uline{F/CTL}~L(R)(L+R)~ELEV~FAULT, FCOM~PRO.AEP.F/CTL}


\section{Stabilizer jam}

Manual pitch trim is a mechanical connection to the stabilizer actuator. It may
be possible to use manual pitch trim when the \ac{ELAC}s have detected a
stabilizer jam, although it may be heavier than normal. If it is usable, trim
for neutral elevators.

The flight controls will revert to Alternate Law. If the stabilizer could not be
moved, gear extension should be delayed until \ac{CONF} 3 and \V{APP} are
achieved so that the elevators are properly trimmed.

If the jam is caused by the mechanical connection, it is possible that the \ac{ELAC}s will not
detect the problem. The procedure in this case is similar, but Normal Law will remain.

\multicite{\uline{F/CTL}~STABILIZER~JAM,
QRH~AEP.F/CTL, FCOM~PRO.AEP.F/CTL}


\section{Aileron faults}

The lateral aircraft handling is not adversely affected even if both ailerons
fail, as the systems compensate by using the spoilers. Fuel consumption will,
however, increase by approximately 6\%.

\multicite{\uline{F/CTL}~L(R)~AIL~FAULT, FCOM~PRO.AEP.F/CTL}


\section{Spoiler faults}

The effect of a spoiler fault depends on whether the spoiler fails retracted or
extended.

If the spoiler fails in the retracted position, handling should not be adversely
affected. A \ac{CONF} 3 landing may reduce any buffeting that is
encountered. Speed brake should not be used if spoilers 3 and 4 are
affected. The loss of ground spoilers will significantly increase landing
distances.

Airbus have identified a failure scenario that leads to high pressure hydraulic
fluid reaching the extend chamber of a spoiler actuator via a failed
o-ring. This has the effect of a spoiler failing in the fully extended
position. In this case, the autopilot does not necessarily have sufficient
authority to control the aircraft, and it should be disconnected. Fuel burn will
increase significantly; \ac{FMGC} fuel predictions do not account for the
failure and should be disregarded. Green dot speed will minimize this increased
fuel burn, but may not be viable if there is excessive buffet -- attempt to find
a compromise speed. Landing will be \ac{conf} 3; \V{APP} and \cphrase{LDG DIST}
factors are available in the \ac{QRH}.

\multicite{\uline{F/CTL}~(GND)~SPLR~(1+2)(3+4)~FAULT, FCOM~PRO.AEP.F/CTL}


\section{Rudder Jam}

The main indication of jammed rudder is undue and adverse pedal movement during
rolling manoeuvres caused by the yaw damper orders being fed back to the pedals
when they are no longer sent to the rudder.

Crosswinds from the side that the rudder is deflected should be avoided, and a
cross wind limit of 15kt applies. Control on the ground will require
differential braking until the steering handle can be used (below 70kt), so
landing distances are increased. Do not use autobrake.

\multicite{\uline{F/CTL}~RUDDER~JAM, QRH AEP.F/CTL, FCOM~PRO.AEP.F/CTL}


\section{ELAC fault}
\label{sec-elac}

In normal operations, \ac{ELAC}1 controls the ailerons and \ac{ELAC}2 controls
the elevators and stabiliser. Failure of a single \ac{ELAC} will result in
failover to the remaining computer. Provided no uncommanded manoeuvres occurred,
an attempt can be made to reset the failed \ac{ELAC}.

Failure of both \ac{ELAC}s leads to loss of ailerons and hence Alternate
Law. One of the \ac{SEC}s will take over control of the elevators and
stabiliser. Again, an attempt can be made to reset the computers.

If the fault is designated a pitch fault, only the pitch function of the
associated \ac{ELAC} is lost.

\multicite{\uline{F/CTL} ELAC~1(2)~FAULT,
FCOM~PRO.AEP.F/CTL}


\section{SEC fault}

Each \ac{SEC} controls either 1 or 2 spoilers per wing. \ac{SEC}1 and \ac{SEC}2
also provide back up for the \ac{ELAC}s (see Section \ref{sec-elac}). Loss of a
\ac{SEC} leads to loss of its associated spoilers. \ac{SEC}1 provides spoiler
position to the \ac{FAC}s. If speedbrakes are deployed with \ac{SEC}1 failed and
\ac{SEC}3 operative, spoiler 2 will deploy without a corresponding increase in
\V{LS}. Therefore, do not use speedbrake if \ac{SEC}1 is affected.

Pairs of \ac{SEC}s also provide the signal for reverse thrust lever angle to the
reversers and spoiler deployment to the autobrake. A dual \ac{SEC} failure will
therefore lead to a loss of a reverser and loss of autobraking.

If all \ac{SEC}s are lost, in addition to the above, the controls revert to
Alternate Law. Due to the normal routing of data from the \ac{LGCIU}s to the
\ac{ELAC}s being via the \ac{SEC}s, Direct Law will occur at selection of
\ac{CONF} 2 rather than at gear extension.\footnote{The autopilot is available
with loss of all \ac{SEC}s. If the autopilot is engaged when \ac{CONF}~2 is
selected, Alternate Law will be retained; the reversion to Direct Law will occur
at autopilot disconnection.}

An attempt should be made to reset the affected \ac{SEC}(s).

\multicite{\uline{F/CTL}~SEC~1(2)(3)~FAULT, FCOM~PRO.AEP.F/CTL}


\section{SFCC faults}

Each \ac{SFCC} has fully independent slat and flap channels. A failure of a
channel in a single controller will lead to slow operation of the associated
surfaces, although this is barely discernible in practice. In addition, the flap
channel of \ac{SFCC}1 provides input to the idle control part of the \ac{FADEC}s
and to the \ac{EGPWS}.

Failure of both flap channels or failure of both slat channels is covered in
Section \ref{sec-stuck-slats/flaps}.

\multicite{\uline{F/CTL}~FLAP(SLAT)~SYS~1(2)~FAULT,
FCOM~PRO.AEP.F/CTL}


\section{FCDC faults}

The two \ac{FCDC}s are redundant, so a single failure has no immediate effect.

If both \ac{FCDC}s fail, the \ac{ELAC}s and \ac{SEC}s can no longer supply data
to the \ac{EIS}. The major effect of this is that \ac{F/CTL} \ac{ECAM} warnings
are no longer generated. The warning lights on the overhead panel continue to
give valid information and should be monitored.

The aircraft remains in Normal Law with all protections, but protection
indications (bank and pitch limits, \V{$\alpha$-prot} and \V{$\alpha$‑max}) are
not shown and the stall warning system becomes active.

\multicite{\uline{F/CTL}~FCDC~1(2)(1+2)~FAULT, FCOM~PRO.AEP.F/CTL}


\section{Wingtip brake fault}

The wingtip brakes activate in case of asymmetry, mechanism overspeed,
symmetrical runaway or uncommanded movements. This protection is lost.

\multicite{\uline{F/CTL}~FLAP(SLAT)~TIP~BRK~FAULT, FCOM~PRO.AEP.F/CTL}


\section{Flap attach sensor failure}

The flap attach sensor detects excessive differential movement between the inner
and outer flaps which would indicate failure of a flap attachment. This
protection is lost.

\multicite{\uline{F/CTL}~FLAP~ATTACH~SENSOR, FCOM~PRO.AEP.F/CTL}


\section{Flight control servo faults}

All flight controls have redundant servos. In the case of an elevator servo
fault, a restriction to not use speedbrake above \V{MO}/M\textsubscript{MO}
applies.

\multicite{\uline{F/CTL}~AIL(ELEV)~SERVO~FAULT, FCOM~PRO.AEP.F/CTL}


\section{Speed brake disagree}

This indicates that the spoiler positions do not correspond with the speedbrake
lever position. This may be as a result of automatic retraction ($\alpha$-floor
activation or speed brakes deployed when \ac{conf} Full selected) or as a result
of spoiler malfunction. In both cases retract the speedbrake lever and in the
case of spoiler malfunction consider the speedbrakes unserviceable.

\multicite{\uline{F/CTL}~SPD~BRK~DISAGREE, FCOM~PRO.AEP.F/CTL}


\section{Speed brake fault}

This indicates a failure of the speedbrake lever transducers rather than a
problem with the spoilers. Ground spoiler activation may be expected on
selection of reverse, so providing reversers are used, landing distances should
not be affected.

\multicite{\uline{F/CTL}~SPD~BRK~(2)(3+4)~FAULT, FCOM~PRO.AEP.F/CTL}


\section{Stiff sidestick/ rudder pedals}

This may affect both sidesticks at the same time, but not the rudder pedals or
it may affect the rudder pedals and one sidestick. Control forces will remain
moderate and the aircraft remains responsive. Confirm autopilot disengagement
and consider transferring control if one of the sidesticks is unaffected.

\multicite{QRH~AEP.F/CTL, FCOM~PRO.AEP.F/CTL}


\section{Unannunciated sidestick faults}

It is possible for a failed sidestick transducer to cause uncommanded control
inputs. If no fault is detected, the result is that the aircraft behaves as if
that input had actually been made. The autopilot will disconnect and any attempt
to control the aircraft with the failed sidestick will fail.

The aircraft should be recovered with the other sidestick using the takeover
button. Keeping this button pressed for 40 seconds will lock out the failed
sidestick, and the autopilot can then be re-engaged. The autopilot should not be
disconnected in the normal manner as pressing \emph{either} takeover button will
re-introduce the failed sidestick and the uncommanded input; use the FCU
instead.


\chapter{Fire}

\section{Smoke and fumes}

The presence of smoke and fumes may or may not generate \ac{ECAM} warnings. If a
smoke related \ac{ECAM} warning occurs, run the \ac{ECAM} before turning to the
\ac{QRH}.\footnote{For the \cphrase{avionics smoke} \ac{ecam}, the \ac{FCTM}
gives you the option of skipping the \ac{ECAM} and referring directly to the
\ac{QRH}, since the \ac{QRH} contains all of the relevant actions. However,
treating this \ac{ecam} as ``special'' just complicates matters, since you will
then have to remember which one was ``special''.}

The \cphrase{smoke/fumes/avncs smoke} \ac{QRH} procedure should be applied
when:

\begin{itemize}
\item Smoke is detected and the crew suspect the avionics, air conditioning or
  cabin equipment as the source.
\item An \cphrase{Avionics Smoke} \ac{ECAM} requests it.
\item Orange peel or pine needle smells are detected in the flight deck. These
  are due to rain repellent leaks; the former is toxic, the latter non-toxic.
  \end{itemize}

While the smoke aspect of these procedures is fairly unambiguous, the fumes
aspect has led to both under- and over-reaction in the simulator. Training
department guidance is that if a ``smell'' is detected in the cabin and it is
having \emph{no physiological effects}, no immediate action is required. If
there \emph{are} physiological effects, the \cphrase{SMOKE/FUMES/AVNCS SMOKE}
\ac{QRH} procedure should be actioned. However, only the first line,
\cphrase{LAND ASAP}, is actioned since there is no perceptible \emph{smoke} and
the rest of the checklist is below the line ``if perceptible smoke apply
immediately''. Thus after cross-confirming \cphrase{LAND ASAP}, immediately move
on to the \cphrase{REMOVAL OF SMOKE/FUMES} \ac{QRH} checklist.\footnote{It
should be noted that confirmation of the correctness of this advice is currently
being sought from Airbus, and that in a number of in-service events the source
of the smell has been positively identified as one of the packs.}


The \ac{QRH} \cphrase{SMOKE/FUMES/AVNCS SMOKE} checklist attempts to isolate the
source of the smoke. It is possible that it may become impossible to carry out
this checklist due to smoke density. In this case, interrupt the checklist and
carry out the smoke removal drill (see Section \ref{sec-smoke-removal}). It is
also possible that the situation may deteriorate to a level that an immediate
forced landing becomes the preferable option. In general, unless the source of
the smoke is obvious and extinguishable, a diversion should be initiated
immediately. The smoke removal drill is most effective and adaptable at lower
levels, so a descent to 10,000ft or \ac{MSA} is also a priority.

The first priority is to protect yourself, so get an oxygen mask on. The mask
must be set to 100\% oxygen to exclude fumes; at minimum dispatch oxygen levels
this will provide as little as 15 minutes of protection. Pushing the ``Emergency
pressure selector'' knob will provide a few seconds of overpressure, which can
be used to clear any smoke trapped in the mask as it was donned.

Likely sources of smoke are the avionics, the cabin fans and the galleys. Smoke
from these sources can be contained with simple and reversible actions which can
be initiated immediately: put the avionics ventilation into smoke removal mode
by selecting both blower and extract fans to \ac{OVRD}, turn off the cabin fans
and turn off the galleys.

Where the source is immediately obvious, accessible and extinguishable, isolate
the faulty equipment. Otherwise the \ac{QRH} provides separate drills for
suspected air conditioning smoke, suspected cabin equipment smoke or suspected
avionics/electrical smoke. In addition the avionics/electrical smoke drill
includes undetermined and continuing smoke sources.

Suspect air conditioning smoke if it initially comes out of the ventilation
outlets. Several \ac{ECAM} warnings are also likely to occur as sensors detect
the smoke in other areas. The displayed \ac{ECAM} procedures must be
applied. Following an engine or \ac{APU} failure, smoke may initially enter the
air conditioning system but should dissipate quickly once the failure is
contained. The air conditioning drill starts by turning the \ac{APU} bleed off
in case this is the source. The packs are then turned off one at a time to
determine if the source of the smoke is a pack.

The cabin equipment smoke drill involves selecting the commercial button off and searching
for faulty cabin equipment.

Suspect avionics smoke if the only triggered \ac{ECAM} is \cphrase{AVIONICS
  SMOKE}. If an item of electrical equipment fails immediately prior to the
appearance of the smoke, that equipment should be suspected as the source. The
avionics/electrical drill (which includes the undetermined source drill) reduces
the amount of electrically powered equipment to a minimum by adopting a slightly
modified emergency electrical configuration.\footnote{The \cphrase{EMER ELEC GEN
  1 LINE} button rather than the \ac{GEN} 1 button is used to disconnect
generator 1 which disconnects generator 1 from the electrical system but allows
it to directly supply one fuel pump in each wing tank.} The resulting \ac{ECAM}
\emph{may}\footnote{Which \ac{ECAM} procedure is displayed is dependent on
whether an \cphrase{AVIONICS SMOKE ECAM} has been triggered prior to the
adoption of emergency electrical configuration.} contain instructions to reset
the generators; these instructions should be disregarded, although the rest of
the \ac{ECAM} procedure must be actioned. The intention is to restore power 3
minutes before landing or at 2000ft aal. Since you will not be able to restore
the two \ac{IR}s that were depowered, the landing will be in Direct Law and
hence \ac{CONF} 3.\footnote{\ac{QRH} \ecam{ELEC}{ELEC EMER CONFIG Sys Remaining}
indicates that by selecting the \cphrase{ATT~HDG} selector to \cphrase{CAPT~3} it
may be possible to retain \ac{IR}3 and hence have sufficient equipment for a Cat
3 Single landing once power is restored. This has not yet been confirmed by
Airbus.}  This is not mentioned in \ac{QRH}, and is only mentioned on the
\ac{ECAM} once gear is extended.

\multicite{\uline{AVIONICS}~SMOKE, QRH~AEP.SMOKE, FCOM~PRO.AEP.SMOKE}


\section{Smoke/fumes removal}
\label{sec-smoke-removal}

Smoke removal procedures initially use the pressurisation system to draw smoke
and fumes overboard by increasing the cabin altitude. If there are no fuel
vapours present, the packs are used to drive the smoke overboard. Otherwise it
is driven overboard by residual pressure.

The final target configuration is packs off, outflow valve fully open and ram
air on. As this depressurises the aircraft, it can only be achieved at lower
levels (preferably \ac{FL}100). If in emergency configuration, turning the
\ac{APU} master switch on connects the batteries for a maximum of 3 minutes and
allows manual control of the \ac{DC} powered outflow valve motor. Once at a
suitable level and below 200kt, as a last resort \ac{PNF}'s cockpit window can
be opened.

\multicite{QRH~AEP.SMOKE, FCOM~PRO.AEP.SMOKE}


\section{Engine fire}

The basic sequence is to bring the thrust lever of the affected engine to idle,
turn off its engine master, push its fire button, wait 10 seconds then deploy
its first fire bottle. If the fire is not extinguished after 30 seconds,
indicated by the fire button remaining lit, deploy the second bottle.

This sequence is modified on the ground in that both fire bottles are fired
immediately. Cross confirmation is not required for master switches or fire
buttons when operated on the ground. The emergency evacuation procedure is then
applied if required (see Section \ref{sec-evacuation}).

\multicite{\uline{ENG}~1(2)~FIRE, FCOM~PRO.AEP.ENG}


\section{Lithium Battery Fire}

If there are flames, they should be attacked with a halon extinguisher. This
will necessitate \ac{PF} donning a crew oxygen mask and \ac{PNF} donning the
smoke hood.

If there are no flames, or once the flames have been extinguished, the cabin
crew should remove the device from the cockpit and store it in a lined container
filled with water. If the device cannot be removed, water or non-alcoholic
liquid should be poured on the device, and it should be continuously monitored
for re-ignition.

Note that these procedures assume that you are dealing with lithium ion
batteries (i.e. rechargeable batteries found in laptops, tablets, phones etc.)
where the amount of water reactive lithium metal is actually fairly low. Once
the flames have been knocked down, the focus is on cooling to prevent thermal
runaway in adjacent cells. Counter-intuitively, it is vital that ice is
\emph{not} used as this acts as a thermal insulator and will likely cause
adjacent cells to explode. For the same reason, smothering with anything that
might thermally insulate the battery pack (e.g. a fire bag) is probably a bad
idea.

If smoke becomes the biggest threat, see Section \ref{sec-smoke-removal}. If the
situation becomes unmanageable, consider an immediate landing.

\multicite{QRH~AEP.SMOKE, FCOM~PRO.AEP.SMOKE, FAA videos (youtube)}


\chapter{Fuel}

\section{Fuel system differences}

The easyJet Airbus fleet has evolved to include a number of variations of the
fuel system. The two major variations, from the point of view of non-normals,
are the replacement of electrical center tank pumps with jet transfer pumps on
some airframes and the simplified tank system of the A321 \ac{NEO} where the
normal outer and inner cells are replaced with a single wing tank. In addition,
some airframes are fitted with a center tank fuel inerting system, but since
there are no cockpit controls for this, it doesn't affect the non-normal
procedures.

The philosophy of center tank jet transfer pumps is very different from that of
the electrical pumps. With the electric pumps, fuel is supplied directly from
the center tank to the engines, and any transfer between center tank and wing
tank is a side effect of the fuel return system. With the jet transfer pumps,
fuel is only ever transferred from the center tank to wing tanks: it is always
the wing tank pumps that supply pressurised fuel to the engines. The jet
transfer pumps themselves are also powered by the wing tank pumps: if there is a
need to shut down a jet transfer pump, that is achieved by turning off the
associated wing pumps. There also appears to be some scope for gravity feeding
from the center tank when jet pumps are fitted (although 2T of center tank fuel
will still be unusable),\footnote{This information only appears in the
\ecam{FUEL}{CTR L+R XFR FAULT} checklist. I have requested further information
and will update once I get a response.} something that is not possible with the
electrical pumps. This leads to some subtle airframe dependent differences in
non-normal fuel procedures.

The simplification of the wing tanks on the A321 \ac{NEO} has less impact on non-normal
procedures: it is mainly a case of changing \ac{ECAM} titles appropriately.


\section{Fuel leak}

Whenever a non-normal fuel event occurs, the possibility that the underlying
cause of the event is a fuel leak should be considered. Only when a fuel leak
has been categorically ruled out should the cross-feed valve be opened.

The primary method used to detect fuel leaks is a regular check that actual fuel
remaining corresponds to expected fuel remaining and that fuel used plus fuel
remaining corresponds to fuel at engine start. The latter parameter is monitored
on some aircraft and may trigger an \ac{ECAM} warning. Other indications of a
leak include fuel imbalance or excessive fuel flow from an engine. It is also
possible that a fuel leak may be detected visually or by a smell of fuel in the
cabin.

If a leak can be confirmed to be coming from an engine or pylon, either visually
or as indicated by excessive fuel flow, the affected engine must be shut
down. In this case, cross-feeding is allowable. Otherwise, the cross-feed must
be kept closed.

If the leak cannot be confirmed to be originating from an engine or pylon, an
attempt should be made to identify the source of the leak by monitoring the
inner tank (or, for A321 \ac{NEO}, wing tank) depletion rates with the crossfeed
valve closed and the center tank pumps off.

If depletion rates are similar, a leak from the center tank or from the \ac{APU}
feeding line should be suspected. If there is a smell of fuel in the cabin, it
is likely that the \ac{APU} feeding line is at fault and the \ac{APU} should be
turned off. Fuel from the center tank should be used once one of the inner/wing
tanks has <3000kg.\footnote{The logic here is strange. An unofficial explanation
of the requirement for <3000kg in the inner tank was given to me: some of the
fuel lines from the center tank run through the wing tanks, so fuel from a
center tank leak may end up transferring to the wing tanks and with full wing
tanks will be lost overboard. As for the \ac{APU} feeding line leak, I would
expect the left tank to decrease faster than the right in this case; my guess is
that the expectation is that an \ac{APU} feeding line leak will be detected as a
smell in the cabin and the leak will be too small to become apparent as an
imbalance.}

If, after 30 minutes, one tank has been depleted by 300kg more than the other,
the location of leak is narrowed down to the engine or the wing on the more
depleted side. To confirm which it is, shut down the engine.  If the leak then
stops, an engine leak is confirmed and the cross feed can be used. If not, a
leak from the wing is most likely. In this case, an engine restart should be
considered.

The handling of the center tank pumps in the presence of a fuel leak is
dependent on whether the aircraft is fitted with electrical center tank pumps or
with jet transfer pumps. When electrical center tank pumps are on, no fuel is
transferred between the center tank and the wing tank unless the engine
associated with that wing tank is running. If an engine \emph{is} running with
electrical center tank pumps on, surplus fuel is returned from the engine to the
associated wing tank, and thus a fairly modest rate of transfer occurs. The jet
transfer pumps, on the other hand, transfer fuel directly from center tank to
wing tank at a high rate regardless of whether the associated engine is
running. It is therefore important not to run a jet transfer pump if you suspect
its associated wing tank has a leak since significant extra fuel loss would
likely occur.

In an emergency, a landing may be carried out with maximum fuel imbalance.

Do not use thrust reversers.

\multicite{\uline{FUEL}~F~USED/FOB~DISAGREE, QRH~AEP.FUEL, FCOM~PRO.AEP.FUEL}


\section{Fuel imbalance}

All fuel balancing must be carried out in accordance with \ac{QRH}
\ecam{FUEL}{Fuel Imbalance}, paying particular attention to the possibility of a
fuel leak. Any action should be delayed until sufficient time has passed for a
fuel leak to become apparent. The \ac{FCOM} adds a note not found in the
\ac{QRH} that ``there is no requirement to correct an imbalance until the
\ac{ECAM} fuel advisory limit is displayed'', an event that occurs when one
inner tank holds >1500kg more than the other. The limitations for fuel imbalance
in \ac{FCOM}~\cphrase{LIM.FUEL}, however, show that the fuel advisory does not
necessarily indicate that a limitation is likely to be breached. In particular,
when the outer tanks are balanced and the heavier inner tank contains
$\leq$2250kg, there are no imbalance limitations. Furthermore, the aircraft
handling is not significantly impaired even at maximum imbalance.

To balance the fuel, open the cross-feed valve and turn the lighter side pumps
and the center tank pumps off.

\multicite{QRH~AEP.FUEL, FCOM~PRO.AEP.FUEL, FCOM~LIM.FUEL}


\section{Gravity fuel feeding}
\label{sec-gravity-fuel}

Turn on ignition and avoid negative G. The ceiling at which fuel can be reliably
gravity fed depends on whether the fuel has had time to deaerate, this being a
function of achieved altitude and time at that altitude. The algorithm used to
calculate gravity feed ceiling is airframe dependent and is provided in the
\ac{QRH}. Once calculated, descend to the gravity feed ceiling; it may be as low
as \ac{FL}150, so terrain must be considered.

It is also possible to gravity cross feed by side slipping the aircraft with the
cross feed valve open. The section of the \ac{QRH} describing this procedure has
recently (June 2017) been cleaned up by moving an explanatory note into a branch
title, but this has had the effect of changing the apparent
intention. Previously, gravity cross feeding was indicated when single engine
and required ``for aircraft handling'' (i.e. it could generally be disregarded),
whereas with the update the procedure is indicated whenever gravity feeding on a
single engine. Clarification has been requested from Airbus via the easyJet
Technical Manager, and these notes will be updated once a response is received.

\multicite{QRH~AEP.FUEL, FCOM~PRO.AEP.FUEL}


\section{Wing tank pump low pressure}

Failed pumps should be turned off.

Failure of a single pump in either tank results in reduced redundancy only.

Failure of both pumps in a given tank means that the fuel in that tank is only
available by gravity feeding. Pressurized fuel may be available from the center
tank (use manual mode if necessary) or by cross-feeding. A descent to gravity
feed ceiling may be required (see Section \ref{sec-gravity-fuel}).

\multicite{\uline{FUEL}~L(R)~TK~PUMP~1(2)(1+2)~LO~PR, FCOM~PRO.AEP.FUEL}


\section{Center tank pump low pressure}

Failed pumps should be turned off.

Failure of a single center tank pump results in a loss of redundancy. The
crossfeed should be opened until the center tank fuel has been exhausted so that
the remaining pump can supply both engines.

Failure of both center tank pumps makes the fuel in the center tank unusable.

\multicite{\uline{FUEL}~CTR~TK~PUMP(S)(1(2))~LO~PR, FCOM~PRO.AEP.FUEL}


\section{Center tank transfer pump faults}

Since the motive power for jet transfer pumps comes from their associated wing
tank pumps, their main failure mode is failure of their associated center
transfer valve.

If a valve fails in a not fully closed position, fuel may continue to be
transferred to the wing tank even when it is full, and will thus
overflow. Selecting the transfer pump off may be effective; otherwise the jet
transfer pump's motive power is removed by switching off its associated wing
tank pumps, with the wing tank pumps from the unaffected side supplying both
engines until the center tank is empty. If both sides fail in this way, all that
can be done is to turn the center tank transfer pumps off; it may not be
possible to prevent fuel overflow.

If a valve fails in a not fully open position, fuel may not be transferred
between center tank and wing tank on the affected side. Manual control of the
transfer pumps may be effective; if not, feeding both engines with the wing tank
pumps of the unaffected side until the center tank is empty is once again the
solution. If both sides fail in this way and manual control is not effective, 2T
of center tank fuel are unusable; the rest is available by gravity.

\multicite{\uline{FUEL}~CTR~L(R)(L+R)~XFR~FAULT, FCOM~PRO.AEP.FUEL}


\section{Auto feed/transfer fault}

Whenever center tank fuel is being used, fuel is transferred from the center
tank to the wing tanks. In the case of jet transfer pumps this is a direct
transfer; in the case of electrical pumps it is via the fuel return
system. Center tank fuel can thus only be used when there is space in the wing
tanks to receive this transferred fuel. The pumps therefore automatically cycle
on and off, starting when there is space for 500kg in the wing tanks and
stopping when the wing tanks are full, until such time as the center tank fuel
is exhausted. In addition, electrical center tank pumps are inhibited whenever
the slats are extended.\footnote{As always with Airbus, there is an
exception. If there is fuel in the center tank, each electrical center tank pump
will operate for two minutes after its associated engine is started, regardless
of slat selection, to pre-pressurise the center tank fuel lines.}

Malfunction of the automatic cycling of the center tank pumps (electrical or jet
transfer) is identified by the presence of more than 250kg of fuel in the center
tank when there is less than 5000kg in one of the wing tanks. Malfunction of
automatic control of electrical pumps is also indicated when they continue to
run when slats are extended or the center tank is empty.

If the automatic control has malfunctioned, the cycling of the center tank pumps
must be managed manually. For both types of pump, they should be switched on
whenever one of the wing tanks has less than 5000kg fuel and center tank fuel
remains, and switched off when one of the wing tanks is full or the center tank
fuel is exhausted. In the case of electrical pumps, they must also be switched
off whenever the slats are extended.

\multicite{\uline{FUEL}~AUTO~FEED(TRANSFER)~FAULT, FCOM~PRO.AEP.FUEL}


\section{Wing tank overflow}

When center tank fuel transfers to a wing tank, either directly in the case of
transfer pumps of via the fuel return system in the case of electrical pumps,
and that wing tank has no space to accommodate it, that fuel will overflow.

To stop the overflow, the fuel transfer must cease. In the case of electrical
center tank pumps, this is just a matter of switching the offending center tank
pump off. With jet transfer pumps, switching the pump off may also be effective;
if not, remove motive power of the offending pump by switching off both of its
associated wing pumps -- pressurised fuel will be available via crossfeed.

\multicite{\uline{FUEL}~L(R)~WING~TK~OVERFLOW, FCOM~PRO.AEP.FUEL}


\section{Low fuel level}

The \ac{ECAM} is triggered at approximately 750kg. This warning is generated by
sensors that are independent of the \ac{FQI} system. The warning may be spurious
if the \ac{ECAM} is triggered just before the wing cell transfer valves open. If
center tank fuel remains, it should be used by selecting the center tank pumps
to manual mode. If there is a fuel imbalance and a fuel leak can be ruled out,
crossfeed fuel as required.

If both tanks are low level, about 30 minutes of flying time remain.

If any change to the current clearance will lead to landing with less than
minimum reserve fuel, declare ``minimum fuel'' to \ac{ATC}. This is just a heads
up to \ac{ATC}, not a declaration of an emergency situation. If it is calculated
that less than minimum fuel will remain after landing, declare a Mayday.

\multicite{\uline{FUEL}(R)(L+R)~WING~TK~LO~LVL, FCOM~PRO.AEP.FUEL, EOMA~8.3.8.2}


\section{Outer tank transfer valve faults}

Imbalances caused by the outer tank transfer valves on one side failing to
sequence correctly, either by failing to open or by opening out of sequence,
will not exceed limits, provided that the total fuel in each wing is the same
(see \ac{FCOM}~\cphrase{LIM.FUEL}).

Of more concern is the sudden loss of usable fuel when transfer valves fail to
open on schedule. If only one side is affected, 700kg of fuel will become
unusable. If both sides are affected, 1400kg of fuel will become unusable. This
will happen when fuel levels are already fairly low, as the valves are triggered
when an inner tank reaches 750kg, potentially leaving you with just 1500kg of
fuel when you were expecting to have 2900kg.

\multicite{\uline{FUEL}~L(R)~XFR~VALVE~OPEN,
  \uline{FUEL}~L(R)~XFR~VALVE~CLOSED,\\
  FCOM~PRO.AEP.FUEL}


\section{Cross-feed valve fault}

If the valve has failed open, fuel balance can be maintained through selective
use of fuel pumps. If it has failed closed, crossfeeding is unavailable.

\multicite{\uline{FUEL}~FEED~VALVE~FAULT, FCOM~PRO.AEP.FUEL}


\section{Low fuel temperature}

\ac{ECAM} is triggered at approx -43\textdegree C for A319/A320 and
-44\textdegree C for A321. If on the ground, delay takeoff until temperatures
are within limits. If in flight, descending or increasing speed should be
considered.

\multicite{\uline{FUEL}~L(R)~OUTER(INNER)~TK~LO~TEMP, FCOM~PRO.AEP.FUEL}


\section{High fuel temperature}

This \ac{ECAM} is known to be triggered spuriously by interference from
communication equipment. The procedure should only be applied if the message has
not disappeared within 2 minutes.

The \ac{ECAM} temperature triggers on the ground are 55\textdegree C for an outer
cell and 45\textdegree C for an inner cell or A321 wing tank. In the air they are
60\textdegree C for an outer cell and 54\textdegree C for an inner cell or A321
wing tank.

The temperature of fuel returning to the tanks is primarily a function of
\ac{IDG} cooling requirement. The immediate action, therefore is to turn the
galley off to reduce the \ac{IDG} load.

On the ground, the engine on the affected side must be shut down if an outer
cell reaches 60\textdegree C, an inner cell reaches 54\textdegree C or, for the
A321, the wing tank reaches 55\textdegree C. An expeditious taxi may, therefore,
be advantageous.

In the air, if only one side is affected, fuel flow can be increased so that
less hot fuel is returned to the tanks. If the temperature gets too high
(>65\textdegree C outer or >57\textdegree C inner/wing), \ac{IDG} disconnection
will be required (see Section \ref{sec-idg}).

\multicite{\uline{FUEL}~L(R)~OUTER(INNER)(WING)~TK~HI~TEMP, FCOM~PRO.AEP.FUEL}

\chapter{Landing gear}

\section{Loss of braking \emph{(memory item)}}
\label{sec-loss-of-braking}

If it is simply an autobrake failure, just brake manually. Otherwise, apply max
reverse and attempt to use the alternate brake system. To do this, release the
brake pedals and turn off the \cphrase{ASKID \& NW STRG} switch. If the
alternate system also appears to have failed, short successive applications of
the parking brake may be used. Use of the parking brake in this way risks tire
burst and lateral control difficulties (due brake onset asymmetry) so delay
until low speed if at all possible.

\multicite{FCOM~PRO.AEP.BRAKES}


\section{Residual braking procedure}

Residual brake pressure must be checked after landing gear extension as there is
no \ac{ECAM} warning. A brief brake pressure indication is expected as the
alternate system self tests after the gear is downlocked, but pressure should
quickly return to zero. If the triple indicator shows residual pressure after
this test, try to zero it by pressing the brake pedals several times. If a
landing must be made with residual pressure in the alternate braking system, use
autobrake \ac{MED} or immediate manual braking to prioritise the normal
system. Anticipate brake asymmetry at touchdown.

\multicite{QRH~AEP.BRAKES, FCOM~PRO.AEP.BRAKES}


\section{Gravity extension}
\label{sec-gravity-gear}

Gravity extension is achieved by turning the \cphrase{GRAVITY GEAR EXTN}
handcrank clockwise three times until a mechanical stop is reached. Once the
gear is down, the \ac{LG} lever should be set to down to extinguish the
\ac{UNLK} lights and remove the \cphrase{LG CTL} message from the \ac{WHEEL}
page.

Availability of landing gear indications depends on the nature of the failure
that resulted in the requirement for gravity extension. \cphrase{LDG GEAR}
control panel indications may still be available if \ac{LGCIU}1 is otherwise
unserviceable, providing that it is electrically supplied.

Gear doors may show amber on the \ac{WHEEL} page after gravity extension. There
may also be spurious \ecam{L/G}{LGCIU 2 FAULT} or \ecam{BRAKES}{SYS 1(2) FAULT}
\ac{ECAM} warnings.

\multicite{QRH~AEP.L/G, FCOM~PRO.AEP.L/G}


\section{Asymmetric braking}
\label{sec-asymmetric-braking}

Defined as all brakes on one gear released (indicated by amber brake release
indicators on both wheels of one main gear on the \cphrase{WHEEL SD} page).
%<!-- {TODO: check  this} -->%
When the remaining brakes are applied, the aircraft will tend to swing
towards them. This tendency must be countered with rudder, hence the braking
must be progressive and co-ordinated with available rudder authority.
Crosswinds from the side of the available brakes will re-inforce the swing, so
anything greater than 10kt from that side should be avoided.

If a reverser is inoperative on the same side as the inoperative brakes, do not
use the remaining reverser since it would also re-inforce the swing.

Landing distances will increase significantly.

\multicite{QRH~AEP.BRAKES, FCOM~PRO.AEP.BRAKES}


\section{Landing with abnormal landing gear}

A landing should be carried out on a hard surface runway using any available
landing gear. Foaming of the runway is recommended. Manual braking should be
used. Reverse thrust should not be used as it will cause ground spoiler
extension. The \cphrase{GRVTY GEAR EXTN} handcrank should be turned back to
normal to allow the landing gear down actuators to be pressurised and thus
reduce the chance of gear collapse.

If the nose gear is not available, move the \ac{CG} aft by moving passengers to
the rear of the aircraft. Use elevator to keep the nose off the runway, but
lower the nose onto the runway before elevator control is lost. Braking must be
progressive and balanced against available elevator authority. The engines
should be shut down with the \cphrase{ENG MASTER} switches prior to nose impact.

If one main gear is not available, consider crossfeeding to remove the fuel from
the wing with the unserviceable gear. The anti-skid system cannot operate with a
single main gear extended and must be switched off to avoid permanent brake
release. The ground spoilers should not be armed in order to maintain the
maximum possible roll authority. The engines should be shut down at
touchdown. After touchdown, use roll control to keep the unsupported wing from
touching down for as long as possible.

If both main gear are unavailable, the engines should be shut down in the
flare. Pitch attitude at touchdown must be >6\textdegree .

All doors and slides are available for evacuation in any of the normal gear up
attitudes.

\multicite{QRH~AEP.L/G, FCOM~PRO.AEP.L/G}


\section{Flight with landing gear extended}
\label{sec-flight-with-gear-extended}

Flight into expected icing conditions is not approved. Gear down ditching has
not been demonstrated. \ac{FMGC} predictions will be erroneous -- selected speed
should be used for all phases except approach. \ac{CLB} and \ac{DES} modes
should not be used. Altitude alerting will not be available. Any failure that
normally causes a degradation to Alternate Law will instead cause a degradation
to Direct Law.

The dual engine failure scenario is modified to reflect the gear limiting
speed. Assisted start should be preferred. If the \ac{APU} is not available,
gear limit speeds should be disregarded to achieve a windmill start. Flight
controls will be in Direct Law; manual pitch trim should be available, even when
not annunciated on the \ac{PFD}.

Performance in all phases will be affected. In particular, approach climb
limiting weights for go-around\footnote{See \inlcite{FCOM PRO.NOR.SUP.L/G}}
must be reduced by 14\%. Fuel burn will increase (approximate factor is
2.3). Engine out ceiling and take-off performance are also impacted.

\multicite{FCOM~PRO.NOR.SUP.L/G}


\section{Gear shock absorber fault}

A shock absorber did not extend when airborne or did not compress on landing. If
airborne the gear cannot be retracted. Respect the gear extended limit speed of
280kt and see Section \ref{sec-flight-with-gear-extended}.

\multicite{\uline{L/G}~SHOCK~ABSORBER~FAULT, FCOM~PRO.AEP.L/G}


\section{Gear not uplocked}

Landing gear retraction sequence has not completed within 30 seconds. If the gear doors
have closed, the gear will rest on the doors so avoid excess g loads. If the doors have not
closed, recycle the gear. If this does not work, select the gear down and see
Section \ref{sec-flight-with-gear-extended}.

\multicite{\uline{L/G}~GEAR~NOT~UPLOCKED, FCOM~PRO.AEP.L/G}


\section{Gear not downlocked}

If the landing gear extension sequence has not completed within 30 seconds,
retract the gear, wait until it has fully stowed, and then redeploy it. Recent
studies show that if the gear does not immediately deploy successfully following
reselection, it may deploy normally within the next two minutes as hydraulic
pressure continues to act on the gear and doors throughout this time. If still
unsuccessful after two minutes, attempt to deploy the gear by gravity (see
Section \ref{sec-gravity-gear}).

\multicite{\uline{L/G}~GEAR~NOT~DOWNLOCKED, FCOM~PRO.AEP.L/G}


\section{Gear doors not closed}

A gear door is not uplocked. Recycle the gear. If the doors cannot be closed,
speed is limited to 250kt/M0.6.

\multicite{\uline{L/G}~DOORS~NOT~CLOSED, FCOM~PRO.AEP.L/G}


\section{Uplock fault}

An uplock is engaged when the corresponding gear is downlocked. As the uplock
will not move to accept the gear the gear must be left down. See Section
\ref{sec-flight-with-gear-extended}.

\multicite{\uline{L/G}~GEAR~UPLOCK~FAULT, FCOM~PRO.AEP.L/G}


\section{LGCIU disagreement}

The \ac{LGCIU}s disagree on the position of the gear. In the absence of other
\ac{ECAM} warnings, the gear position can be assumed to agree with the gear
lever position.

\multicite{\uline{L/G}~SYS~DISAGREE, FCOM~PRO.AEP.L/G}


\section{LGCIU fault}

The \ac{FADEC}s use \ac{LGCIU} input to determine idle mode. If a \ac{LGCIU} is
determined to be faulty, the system failsafes to approach idle mode, and
modulated idle and reverse idle (and hence reversers) will not be available.

The \ac{GPWS} uses \ac{LGCIU}1 to determine landing gear position. If this
\ac{LGCIU} is faulty, the \ac{GPWS} will need to be inhibited to prevent
spurious warnings.

If both \ac{LGCIU}s are lost, normal landing gear control and indicating systems
are lost. The gear must be gravity extended (see Section
\ref{sec-gravity-gear}). Additionally, the autopilots and autothrust are lost
(Normal Law remains available) and wing anti-ice is limited to 30s of heating
(i.e. the ground test), the only indication of which is a \cphrase{no Anti-Ice}
message on the \cphrase{BLEED SD} page.

\multicite{\uline{L/G}~LGCIU~1(2)~FAULT, FCOM~PRO.AEP.L/G}


\section{Gear not down}

Indicates that the landing gear is not downlocked when radio altitude is below
750ft rad alt and \ac{N}1 and flap setting indicate that the aircraft is on
approach. If rad alt data is not available, it indicates gear is not down when
\ac{conf} 3 or \ac{conf} Full is selected. In some cases the warning may be
cancelled with the emergency cancel pushbutton.

\multicite{\uline{L/G}~GEAR~NOT~DOWN, FCOM~PRO.AEP.L/G}


\section{Park brake on}

The parking brake is set when the thrust levers are set to \ac{FLX} or
\ac{TOGA}. Check the position of the brake handle position and for pressure
indications on the brake triple gauge.

\multicite{\uline{CONFIG}~PARK~BRK~ON, FCOM~PRO.AEP.CONFIG}


\section{Nosewheel steering fault}
\label{sec-nose-wheel-steering}

Nosewheel steering is unavailable so differential braking must be used to steer
the aircraft. The nosewheel may not be aligned if the \ecam{L/G}{shock absorber
  fault} \ac{ECAM} is also displayed, in which case delay nosewheel touch down
as long as possible. Cat 3 dual will not be available.

\multicite{\uline{WHEEL}~N/W~STRG~FAULT, FCOM~PRO.AEP.WHEEL}


\section{Antiskid nosewheel steering off}
\label{sec-askid-off}

The \cphrase{A/SKID \& NW STRG} switch is off. The \ac{ABCU} controls braking
through the alternate braking system. Antiskid is not available so landing
distance will increase significantly. Autobrake and nosewheel steering will also
not be available.

\multicite{\uline{BRAKES}~ANTI~SKID/NWS~OFF, FCOM~PRO.AEP.BRAKES}


\section{Antiskid nosewheel steering fault}

Either:

\begin{itemize}
\item both BSCU channels have failed or
\item the normal brake system has been lost and the yellow hydraulic pressure is
low.
\end{itemize}

Effects are as for Section \ref{sec-askid-off}, although, if yellow hydraulic
pressure is low, braking will be accumulator only.

\multicite{\uline{BRAKES}~A/SKID~NWS~FAULT, FCOM~PRO.AEP.BRAKES}


\section{Brake system fault}

A fault has been detected in one channel of the \ac{BSCU}. Loss of redundancy
only.

\multicite{\uline{BRAKES}~SYS~1(2)~FAULT, FCOM~PRO.AEP.BRAKES}


\section{Brakes hot}

At least one brake temperature is >300\textdegree C. Check Section
\ref{sec-brake-temp-limits} if the temperature is excessive or the brake
temperatures are not reasonably even.

Temperature must be <300\textdegree C for takeoff to prevent ignition of any
hydraulic fluid that leaks onto the brake. Use brake fans as necessary to bring
the temperature down in time for the next takeoff. The brake fans also cool the
temperature sensor, so assume the real brake temperature is twice that indicated
if they have recently been used.

% TODO: Check whether we use chocks and release the parking brake.

If the warning appears in flight, providing that performance permits, the
landing gear should be extended to allow the brakes to cool.

\multicite{\uline{BRAKES}~HOT, FCOM~PRO.AEP.BRAKES}


\section{Auto brake fault}

A failure was detected when the autobrake was armed. Brake manually.

\multicite{\uline{BRAKES}~AUTO~BRK~FAULT, FCOM~PRO.AEP.BRAKES}


\section{Hydraulic selector valve fault}

This \ac{ECAM} message may indicate two completely different conditions:

\begin{itemize}
\item The normal brake selector valve has failed in the open position. The
  normal servo valves (downstream of the selector valve) will have continuous
  full pressure at their inlets, but, as long as anti-skid is operative, will
  control brake pressure and anti-skid normally.
\item The steering selector valve has failed in the open position. This means
  that the steering will remain pressurised as long as there is pressure in the
  yellow hydraulic system. This has obvious implications if towing is attempted,
  but will also mean that the nosewheel will go to maximum deflection if the
  \cphrase{A/SKID \& N/W STRG} switch is selected off or the \ac{BSCU} is reset.

\end{itemize}

\multicite{\uline{WHEEL}~HYD~SEL~FAULT, FCOM~PRO.ABN.32}


\section{Brake system failures}

Loss of the alternate braking system results in loss of redundancy only.

If the normal brake system is lost, alternate braking and anti-skid are
available. Landing distance increases slightly.

Loss of both normal and alternate brake systems leaves the parking brake as the
only remaining braking option. See Section \ref{sec-loss-of-braking} for method.

\multicite{\uline{BRAKES}~NORM~BRK~FAULT, \uline{BRAKES}~ALTN~BRK~FAULT,\\
  FCOM~PRO.AEP.BRAKES}


\section{Brake accumulator low pressure}

Braking is not available unless either the green or yellow hydraulic systems are
pressurised.

If the engines are shut down, attempt to recharge the accumulator
using the yellow system electrical pump.

When parking the aircraft, use chocks.

\multicite{\uline{BRAKES}~BRK~Y~ACCU~LO~PR, FCOM~PRO.AEP.BRAKES}


\section{Released brakes, normal system}
\label{sec-released-normal-brakes}

If normal braking is active and at least one engine is running, the \ac{BSCU}
self tests when it receives a ``gear downlocked'' signal from either of the
\ac{LGCIU}s. The \ecam{BRAKES}{RELEASED} \ac{ECAM} is provided if at least one
set of brakes on a main wheel is incorrectly released during this test. The
failed brake is shown by an amber release symbol on the \ac{WHEEL} page. Loss of
a brake leads to increased landing distances. If both brakes on the same gear
are released, see Section \ref{sec-asymmetric-braking}.

\multicite{\uline{BRAKES}~RELEASED, FCOM~PRO.AEP.BRAKES}


\section{Released brakes, alternate system}

The \ac{ABCU} self tests the brakes in a similar manner to the \ac{BSCU} (see
Section \ref{sec-released-normal-brakes}). If this test is failed, normal
braking can be expected as long as the normal braking system is active. If the
alternate braking system is active, braking will be asymmetric (see Section
\ref{sec-asymmetric-braking}) because released brakes occur in pairs with the
alternate braking system.

\multicite{\uline{BRAKES}~ALTN~L(R)~RELEASED,
FCOM~PRO.AEP.BRAKES}


\section{Brake temperature limitations}
\label{sec-brake-temp-limits}

Maintenance is required if:

\begin{itemize}
\item One brake temp is >600°C and the other brake on the same gear is 150°C
  less.
\item One brake temp is <60°C and the other brake on the same gear is 150°C
  more.
\item The average temp of one gear is 200°C more than the average temp of the
  other.
\item Any brake temp exceeds 900°C.
\item A fuse plug has melted.
\end{itemize}

\multicite{EOMB~2.3.21}


\chapter{Power plant}

\section{All engine failure}
\label{sec-all-engine-failure}

The entire easyJet fleet has, thankfully, now been upgraded such that the
various \ecam{ENG}{DUAL FAILURE} checklists are no longer applicable. Where
plenty of time is available, the response to failure of both engines is now
supported by the \ecam{ENG}{ALL ENGINES FAILURE} \ac{ECAM} and the
\ecam{ENG}{ALL ENG FAIL} \ac{QRH} procedure. For Hudson-like events, the
\ac{QRH} \ecam{MISC}{EMER LANDING ALL ENG FAILURE} checklist (also available on
the back of the normal checklist) should be used. Note that the \ac{QRH}
\ecam{MISC} {Ditching} and \ac{QRH} \ecam{MISC}{Forced Landing} checklists are
for engines operative landings and are therefore not applicable; engine
inoperative ditching and forced landing are incluced in the \ecam{ENG}{ALL ENG
  FAIL} \ac{QRH} procedure.

The \ecam{ENG}{ALL ENGINES FAILURE} \ac{ECAM} actions ensure that emergency
electrical power is online and that the aircraft is optimally set up for an
immediate windmill relight (300kt/M0.77 for \ac{CEO}, 270kt/M0.77 for \ac{NEO},
thrust levers idle). \ac{APU} start is suggested if below \ac{FL}250, although
this may be spurious if fuel is exhausted or the \ac{APU} is otherwise
unavailable. For the \ac{CEO}s, a \ac{FAC}1 reset is also actioned in order to
recover \ac{PFD} characteristic speeds and rudder trim; this does not appear to
be necessary for the \ac{NEO}s. The \ac{ECAM} then suggests a diversion and
hands off to the \ac{QRH} procedure. The \ac{QRH} includes the \ac{ECAM}
actions, so it can be used directly if the \ac{ECAM} is unavailable.

Due to lack of engine bleeds, a slow depressurisation will likely be
occuring. Since it would be easy to miss excess cabin altitude warnings, donning
an oxygen mask may be a sensible precaution. \cphrase{RAM AIR} can be used once
below \ac{FL}100 with differential pressure <1psi.

For the diversion, as a rough rule of thumb, from normal cruise levels any
airfield within 80nm should be reachable with sufficient height remaining to
position for a glide approach. You really want at least a 3000m runway, although
if you can get the \ac{APU} supporting normal electrics and the yellow
electrical hydraulic pump it may be possible to make do with less. Take account
of descent winds, airport elevation and available runway directions when
selecting an airport.

There are two main bifurcations in the procedures, dependent on whether there is
any chance of restarting an engine in the first case, and whether an emergency
landing will be made on water or land in the second.

Where relight is feasible, parallel windmill start attempts may be attempted
once below \ac{FL}270\footnote{Note that this is different from the old
\ecam{ENG}{DUAL FAILURE} checklist, where windmill start attempts were tried
immediately, even if outside the relight envelope.} (or below \ac{FL}250 on some
airframes), and sequential starter assisted start attempts may be made once
below \ac{FL}200, provided that the \ac{APU} bleed is available. The windmill
start attempts consist of selecting \ac{IGN} on the \cphrase{ENG MODE sel},
turning both engine masters off for 30 seconds to ventilate the combustion
chambers, then turning them both back on, repeating the cycle if
unsuccessful. Starter assisted start is much like the normal engine start
procedure: turn both masters off for 30 seconds, ensure \cphrase{ENG MODE sel}
\ac{IGN}, pneumatic pressure is available to the starter and wing anti-ice is
off, then turn one of the masters on. If the engine fails to start, turn that
master off, and try the other one, ensuring 30 seconds of combustion chamber
ventilation between each attempt. When making starter assisted start attempts,
speed should be reduced to green dot to achieve maximum glide range; if windmill
starts are required below \ac{FL}200 (e.g. due lack of \ac{APU}), suitable
speeds can be found in the \ecam{ENG}{RELIGHT} \ac{QRH} checklist.

Where relight is not a possibility, speed should be reduced to green dot to
maximise glide range and available time. At windmill relight speeds, available
range is 2nm per 1000ft, wheras at green dot it is 2\textonehalf nm per
1000ft. If available, the \ac{APU} should still be started at \ac{FL}250 in
order to provide normal electrics and pressurisation.

The Green and Yellow hydraulic failure aspect of the dual engine failure is
interesting in that the checklists make no attempt to bring the Yellow Electric
pump online once electrical power is available from the \ac{APU}. The rationale
for this has three parts: firstly, there are branches of the failure where the
yellow electric pump will not be available, such as if the \ac{APU} was inop,
and accounting for these cases in the checklists would over-complicate them;
secondly, the engine driven pumps continue to provide hydraulic pressure for
some time due to windmilling; and lastly, the \ac{ECAM} will eventually
recognise the dual hydraulic failure and request the pump be turned on, and the
checklist does encourage you to clear the \ac{ECAM} alerts and \ac{STATUS} if
sufficient time is available.

The problem with this is that you will likely have
given up on the \ac{ECAM} by the time it makes this suggestion. In general,
then, if you recognise that the yellow electric pump is available, turn the
\ac{PTU} off (see Section \ref{sec-dualhyd-gy}) and turn the yellow pump on. You
will, of course, still need to gravity extend the gear, as the green system will
not be recovered, but with blue from the \ac{RAT} and yellow from the electric
pump, your stopping ability is greatly enhanced.

The recommended configurations are \ac{CONF} 2, gear up for ditching and
\ac{CONF} 2, gear down for forced landing. \V{APP} is available in each of the
checklists; it will always be at least 150kt to give a 10kt margin against
\ac{RAT} stall. The gear is available with gravity extension. If the yellow
hydraulics have not been reinstated with the electric pump the stabilizer will
be frozen once engine driven hydraulics are lost and elevator trimming will
cease with transition to Direct Law at gear extension. Therefore, it may be
advantageous to delay gear extension until \ac{CONF} 2 and \V{APP} are reached
in this case. A pitch attitude of 11\textdegree\ with minimal vertical speed is
suggested for ditching.

If an airfield can be reached, arrange to be inbound on the runway centerline at
4nm and 2400ft \ac{aal} (giving a 6\textdegree\ glide to the threshold) with
\ac{CONF} 1, S speed and gear up. To help achieve this, for a clean aircraft,
the following rules of thumb apply:

\begin{itemize}
\item A standard one minute leg holding pattern loses 8000ft and an orbit loses
  4000ft. Thus for every 15 seconds outbound in a holding pattern, approximately
  1000ft is lost.
\item Wings level, 400ft is lost per nm.
\end{itemize}

For the segment inbound from 4nm, macro adjustment of glide path is available
through the timing of gear and slat deployment (\ac{CONF} 1, S speed, gear up
gives about a 4\textonehalf\textdegree\ glide; \ac{CONF} 3, 150kt, gear down
gives about a 7\textonehalf\textdegree\ glide), then micro adjustment is
available from temporarily increasing speed above \V{APP}. If necessary,
disregard slat limiting speeds. It is better to land fast then long.

\multicite{\uline{ENG}~DUAL~FAILURE, QRH~AEP.ENG, FCOM~PRO.AEP.ENG,
  FCTM~PRO.AEP.ENG}


\section{Single Engine failure}

Defined as a rapid decrease in \ac{EGT}, \ac{N}2 and \ac{FF}, followed by a
decrease in \ac{N}1. The crew must determine whether the engine has been damaged
or whether a simple flame-out has occurred. Indications of damage are loud
noises, significantly increased vibration or buffeting, repeated or
uncontrollable engine stalls or abnormal post-failure indications
(e.g. hydraulic fluid loss, zero \ac{N}1 or \ac{N}2 etc.).

Firstly, the ignitors are turned on to protect the remaining engine and to
confirm an immediate relight attempt. The thrust lever of the failed engine is
then moved to idle (\ac{PF} moves the lever after confirmation from
\ac{PNF}). If the \ac{FADEC} hasn't relit the failed engine within 30 seconds of
the failure, it is shut down with the master switch. If damage is believed to
have occurred, the associated fire button is pushed and, after 10 seconds, agent
1 discharged.

If it is believed that the engine is undamaged, a relight can be considered. The
relight procedure is fairly long and highly unlikely to be successful; do not
delay diversion and landing by attempting a relight. Also note that a relight
attempt will erase \ac{FADEC} troubleshooting data.

If there is vibration and/or buffeting, attempt to find an airspeed and altitude
combination that minimizes the symptoms.

Refer to Section \ref{sec-single-eng-ops} if unable to relight the engine.

\multicite{\uline{ENG 1(2)} FAIL,
FCOM~PRO.AEP.ENG}


\section{Single engine operation}
\label{sec-single-eng-ops}

The most pressing issue is that a single engine bleed cannot support wing
anti-ice and two packs. With the crossbleed valve selector in the normal
\ac{AUTO} position, the crossbleed valve is effectively synchronised to the
\ac{APU} bleed valve\footnote{The exception is that the crossbleed won't open if
a bleed air duct leak is detected except during engine start.} and thus will
most probably be closed; wing anti-ice, if it is in use, will be operating
asymmetrically. If a fire button has been pushed, its associated side of the
pneumatic system will be locked out and thus the only option is to turn the wing
anti-ice off. \inlcite{PRO.NOR.SUP.AW} \cphrase{Minimum Speed with Ice
  Accretion} provides mitigation of icing in the event of inoperative wing
anti-ice. If both sides of the cross bleed system are available, the cross bleed
valve can be manually opened at a cost of 1200ft to the single engine gross
ceiling. With the cross bleed valve open, wing anti-ice is available, but one of
the packs must be turned off\footnote{It will need to be pack 1 in Emergency
Electrical config; otherwise it will generally be the pack on the dead engine
side.} whenever it is used.

The remaining engine must be safeguarded. To this end, continuous ignition
should be selected.

A fuel imbalance may develop. Fuel imbalance limitations are detailed in
\ac{FCOM} \cphrase{LIM.FUEL}.  If the outer tanks are balanced, once the fuller
inner tank contains less than 2250kg, fuel balance will never be limiting. Since
this first occurs with approximately 5900kg of fuel remaining, fuel balancing
due to balance limitations will generally not be required. Fuel may, however,
still need to be crossfed to prevent fuel starvation of the remaining
engine. Balance this concern against feeding your live engine the same fuel that
was feeding your failed engine when it stopped working.

\ac{TCAS} should be selected to \ac{TA} to avoid unflyable climb \ac{RA}s.

If a reverser is unlocked with associated buffet, speed should be limited to
240kt. See Section \ref{sec-reverser-unlocked} for more details of this
scenario.

If the remaining engine is operated at maximum power with the aircraft at low
speed (e.g. responding to windshear) it is possible that directional control may
be lost before the flight computer protections apply. Be cautious about reducing
speed below \V{LS} on one engine.

The main systems lost are the generator, bleed and hydraulic pump associated
with the engine. Other systems may be lost depending on the reason for the
shutdown. The \ac{APU} can be used to replace the lost generator and, providing
the left side of the pneumatic system is available and isolated (i.e. cross
bleed valve closed), provide pressurisation through pack 1, thus giving
additional margin for the go-around. The \ac{BMC}s automatically close the
engine bleeds when the \ac{APU} bleed valve is opened, so it is not necessary to
manually turn them off to achieve this additional go-around margin. Note,
however, that the \ac{APU} cannot support wing anti-ice.

Approach and landing will be fairly normal. The main provisos are:

\begin{itemize}
\item Full flap should only be selected once descending on the glidepath; if a level
off is required, the landing should be \ac{CONF} 3 [\ac{QRH} \ecam{ENG}{OEI -- Straight in
approach}]
\item Only Cat 3 Single is available due to the loss of the ability to split the
  electrical system.[\ac{QRH}~\ac{OPS}]
\item On A319s, the autopilot cannot fly \cphrase{FINAL APP}, \cphrase{NAV/VS}
  or \cphrase{NAV/FPA} approaches. All modes are available for manual flight
  with flight directors. [\ac{FCOM}~\cphrase{LIM.AFS.GEN}]
\item If flying manually, consider using manual thrust to better anticipate the
  rudder inputs required by thrust changes. Also consider setting rudder trim to
  zero at a late stage of the approach.[\ac{FCTM}~\cphrase{PRO.AEP.ENG}]
\end{itemize}

\multicite{\uline{ENG}~1(2)~SHUT~DOWN, FCOM~PRO.AEP.ENG}


\section{Engine relight in flight}

A graph showing the in flight relight envelope is provided in section
\cphrase{AEP.ENG} of the \ac{QRH}. The ceiling is 27000 ft. Automatic start is
recommended, but crew action is required in case of abnormal start.

To prepare for the start, ensure the affected engine master switch is turned off
and the affected thrust lever is at idle. Select ignition on the engine mode
selector and open the cross bleed. If it is anticipated starter assist may be
required, ensure wing anti ice is selected off.

To begin the start sequence, select the affected master switch on. The
\ac{FADEC} will determine whether starter assist is required and will open the
start valve as needed. Both ignitors are energised as soon as the master switch
is turned on, and the \ac{HP} fuel valve opens at 15\% \ac{N}2. Closure of the
start valve and de-energisation of the ignitors occurs at 50\% \ac{N}2 as
normal. Light off must occur within 30 seconds of fuel flow initiation. If
uncertain about successful relight, move the thrust lever to check for engine
response. The \ecam{ENG}{START FAULT} and \ecam{ENG}{STALL} \ac{ECAM}s may be
disregarded if all other parameters are normal.

\multicite{QRH~AEP.ENG, FCOM~PRO.AEP.ENG}


\section{Engine stall}

A stall is indicated by abnormal engine noise, flame from the engine exhaust
(and possibly inlet in extreme cases), fluctuating performance parameters,
sluggish thrust lever response, high \ac{EGT} and/ or rapid \ac{EGT} rise when
the thrust lever is advanced.

A variety of \ac{FADEC}s are fitted within the easyJet fleet. The earlier
\ac{FADEC}s do not trigger an \ac{ECAM} warning if \ac{N}2 is above idle,
whereas the later \ac{FADEC}s are more capable. The \ac{FCTM} warns that all
\ac{FADEC}s may fail to detect engine stalls in some cases. Crew must therefore
be ready to diagnose engine stalls on the basis of the above symptoms and apply
the \ac{QRH} procedures where necessary.

If an engine stall occurs on the ground, shut the engine down.

When an engine stall occurs in flight, the response is airframe specific. For
the earlier \ac{FADEC}s, if an \ac{ECAM} is triggered, the engine is simply shut
down.  In all other cases (no \ac{ECAM} triggered on earlier \ac{FADEC}s; later
\ac{FADEC}s) an attempt is made to contain the stall without shutting down the
engine. The affected thrust lever is retarded to idle and the engine parameters
checked. If the engine parameters remain abnormal, the engine is shut down. If,
however, the parameters return to normal, stall margin is increased by turning
on anti-icing\footnote{The new \ac{QRH} procedure for the \ac{NEO} requires wing
anti-ice be turned on whereas that for the \ac{CEO} requires engine anti-ice be
turned on, but both then go on to mention that stall margin is increased by
turning on both, albeit at a cost of increased \ac{EGT}. The old \ac{QRH}
procedure did turn on all relevant anti-icing. I have requested more information
and will update this section when I get a response.} and the thrust levers are
slowly advanced. If the stall recurs, the engine can be operated at low thrust
settings, otherwise it can be operated normally.

\multicite{\uline{ENG~1(2)}~STALL, QRH~AEP.ENG, FCOM~PRO.AEP.ENG}


\section{Engine tailpipe fire}

An internal engine fire may be encountered during engine start or shutdown. It
will either be seen by ground crew or may be indicated by \ac{EGT} failing to
decrease after the master switch is selected off.

Start by getting the engine to a known state by ensuring the \cphrase{man start} button is
selected off and the affected engine master is selected off.

The concept is to blow the fire out by dry cranking the engine. It is therefore
essential that the fire button is \emph{not} pressed, as this will remove
external power from the \ac{FADEC} and prevent dry cranking. Firstly, a source
of bleed air must be available to power the starter. Possibilities, in order of
preference, are the \ac{APU}, the opposite engine or a ground air cart. If using
the opposite engine, the source engine bleed must be on, the target engine bleed
should be off, the cross bleed should be opened and thrust increased to provide
30 psi of pressure.
%<!-- {\ac{TODO}: This is just the crossbleed start procedure
% from \ac{QRH}~\ac{SI}.150 -- --> <!-- check that there are no differences} -->
If using ground air, both engine bleeds should be off and the cross bleed
opened. Once high pressure air is available, select the engine mode selector to
crank and select the \cphrase{man start} button to on. Once the fire is extinguished,
select the \cphrase{man start} button off and the engine mode selector to normal.

As a last resort, external fire suppression agents may be used. They are,
however, highly corrosive and the engine will be badly damaged.

\multicite{QRH~AEP.ENG, FCOM~PRO.AEP.ENG}


\section{High engine vibration}

The \ac{ECAM} \ac{VIB} advisory (\ac{N}1$\geq$6 units, \ac{N}2$\geq$4.3 units)
is simply an indication that engine parameters should be monitored more
closely. High \ac{VIB} indications alone do not require the engine to be shut
down.

High engine vibration combined with burning smells may be due to contact of
compressor blade tips with associated abradable seals.

If in icing conditions, high engine vibration may be due to fan blade or spinner
icing. The \ac{QRH} provides a drill to shed this ice, after which normal
operations can be resumed.

If icing is not suspected and if flight conditions permit, reduce thrust so that
vibrations are below the advisory level. Shut down the engine after landing for
taxiing if vibrations above the advisory level have been experienced.

\multicite{QRH~AEP.ENG, FCOM~PRO.AEP.ENG}

\section{Low oil pressure}

The sensors for the gauge on the \ac{ECAM} \ac{ENG} page and the \ac{ECAM}
warning are different. If there is a discrepancy between the two, a faulty
transducer is the most likely cause and the engine can continue to be operated
normally. If both sources agree, the engine should be shut down by retarding its
thrust lever and selecting its master switch off and the after shutdown
procedure applied (see Section \ref{sec-single-eng-ops}).

\multicite{\uline{ENG~1(2)}~OIL~LO~PR, FCOM~PRO.AEP.ENG}


\section{High oil temperature}

It may be possible to reduce oil temperature by increasing engine fuel flow.

If oil temperature exceeds 155\textdegree C or exceeds 140\textdegree C for 15
minutes, the engine must be shut down.

\multicite{\uline{ENG~1(2)}~OIL~HI~TEMP, FCOM~PRO.AEP.ENG}


\section{Oil filter clog}

If a warning occurs during a cold engine start with oil temperature
<40\textdegree C, the warning may be considered spurious. The oil filter
features a bypass mechanism, so there is no immediate problem.

\multicite{\uline{ENG~1(2)}~OIL~FILTER~CLOG, FCOM~PRO.AEP.ENG}


\section{Fuel filter clog}

No immediate crew action required. I assume there is some sort of bypass mechanism, but
this isn't apparent from the \ac{FCOM}.

\multicite{\uline{ENG~1(2)}~FUEL~FILTER~CLOG, FCOM~PRO.AEP.ENG}


\section{Uncommanded reverser pressurisation}

There are two valves that prevent pressure reaching the thrust reverser
actuators at an inopportune moment, plus a third that commands direction of
movement. The most upstream of these, controlled by the \ac{SEC}s, prevents any
hydraulic pressure reaching the Hydraulic Control Unit (\ac{HCU}) when the
thrust levers are not in the reverse quadrant. If this protection is lost, the
correct operation of the \ac{HCU} should keep the doors properly stowed. An
\ac{HCU} malfunction, however, could result in an in-flight reverser
deployment. If flight conditions permit, idle thrust should be selected on the
affected engine.

It is unclear from the \ac{FCOM} whether the \ac{ECAM} indicates pressure has
reached the directional solenoid valve and hence that the reverser door jacks
are pressurised, albeit in the closed direction, although the existence of the
\ecam{ENG}{REV ISOL FAULT ECAM} indicates that this is probably the case.

\multicite{\uline{ENG~1(2)}~REV~PRESSURIZED, FCOM~PRO.AEP.ENG}


\section{Reverser unlocked in flight}
\label{sec-reverser-unlocked}

If one or more reverser doors are detected as not stowed in flight, the
associated \ac{FADEC} will automatically command idle on the affected
engine. This should be backed up by setting the thrust lever to idle.

A warning without associated buffet is likely to be spurious. In this case limit
speed to 300kt/M.78, keep the engine running at idle and expect to make a normal
single engine approach and landing.

If there is buffet, shut the engine down and limit speed to 240kt. Full rudder
trim may be required. The \ac{ECAM} will provide one of two approach procedures
depending on how many doors are detected as not stowed:

\begin{itemize}
\item If all 4 doors are not stowed on \ac{CEO} or the reverser is deployed on
  \ac{NEO}, it will be a \ac{conf} 1 landing, with approach speed \V{REF}~+~55kt
  slowing to \V{REF}~+~40kt below 800ft. Gear should only be deployed once
  landing is assured.
\item Otherwise, it will be a flap 3 landing at \V{REF}~+~10kt for \ac{CEO} or
  \V{REF}~+~15kt for \ac{NEO}.
\end{itemize}

\multicite{\uline{ENG~1(2)}~REVERSE~UNLOCKED, FCOM~PRO.AEP.ENG}


\section{EIU fault}

The Engine Interface Unit (\ac{EIU}) receives data from the engine start system,
the auto-thrust system, the \ac{LGCIU}s, the air conditioning controller and the
engine anti ice system and feeds it to its related \ac{FADEC}. Thus loss of the
\ac{EIU} leads to loss of auto-thrust, reverser, idle control (defaults to
approach idle) and start for the affected engine. If engine anti ice is used,
the ignitors must be manually selected.

If an engine fails whilst its associated \ac{EIU} is inoperative, the usual
\ac{ECAM} messages will not be generated. The failure can still be diagnosed
from the system pages and an appropriate drill can be actioned from the
\ac{FCOM}.

\multicite{\uline{ENG~1(2)}~EIU~FAULT, FCOM~PRO.AEP.ENG}


\section{N1/N2/EGT overlimit}

If the overlimit is moderate, the associated thrust lever can be retarded until
the overlimit ceases, and the flight may be continued normally.

If the overlimit is excessive, the engine should generally be shut down. If
there are over-riding factors precluding a shut down, the engine may be run at
minimum required thrust.

\multicite{\uline{ENG~1(2)}~N1/N2/EGT~OVERLIMIT, FCOM~PRO.AEP.ENG}


\section{N1/N2/EGT/FF discrepancy}

The system can detect a discrepancy between actual and displayed values of
\ac{N}1, \ac{N}2, \ac{EGT} and fuel flow. This is indicated by an amber
\ac{CHECK} beneath the affected parameter. Attempt to recover normal indications
by switching from \ac{DMC}1 to \ac{DMC}3. If this fails, values can be inferred
from the opposite engine.

\multicite{\uline{ENG~1(2)} N1(N2)(EGT)(FF)~DISCREPANCY, FCOM~PRO.AEP.ENG}


\section{Start valve fault}

If a start valve fails open, remove bleed sources supplying the faulty valve. If
on the ground, turn off the \cphrase{MAN START} button if used, and shut the
engine down with its master switch.

If the start valve fails closed, it may be that insufficient pressure is
reaching it. Try opening the cross bleed and turning on the \ac{APU} bleed.

On the ground, a start may still be possible with manual operation of the start
valve.

\multicite{\uline{ENG~1(2)}~START~VALVE~FAULT, FCOM~PRO.AEP.ENG}


\section{Start faults}

Start faults include ignition faults (no light off within 18 seconds of ignition
start), engine stalls, EGT overlimit (>725\textdegree C) and starter time
exceedance (2 mins max).

On the ground, nearly all starts are auto starts. In this case the \ac{FADEC}
will automatically abort as needed. It will then automatically carry out the
required dry crank phase and make further attempts. Once the \ac{FADEC} gives
up, an \ac{ECAM} message will instruct the crew to turn off the relevant engine
master. If the fault was a stall due to low pressure, consider another automatic
start using cross bleed air.

If a manual start is attempted, the crew must monitor the relevant parameters
(the \ac{FADEC}s will provide some passive monitoring) and, if necessary, abort
the start by turning the engine master and man start button off. The crew must
then carry out a 30 second dry crank phase manually. Note that this is not
mentioned in the relevant supplementary procedure, nor are the relevant lines
displayed on the \ac{ECAM}. It is probably worth having \ac{FCOM}
\inlcite{PRO.AEP.ENG} handy when carrying out manual starts.

Following an aborted start in flight, the engine master should be turned off for
30 seconds to drain the engine. A further start attempt can then be made.

If the electrical power supply is interrupted during a start (indicated by loss
of \ac{ECAM} \ac{DU}s) turn the master switch off, then perform a 30 second dry
crank.

If a fuel leak from the engine drain mast is reported, run the engine at idle
for 5 minutes. If the leak disappears within this time the aircraft may dispatch
without maintenance action.

\multicite{\uline{ENG~1(2)}~START~FAULT, FCOM~PRO.AEP.ENG, EOMB~2.3.8.1}


\section{Ignition faults}

Each engine has two ignitors. If both fail on a single engine, avoid heavy rain,
turbulence and, as far as possible, icing conditions.

\multicite{\uline{ENG~1(2)}~IGN~FAULT, FCOM~PRO.AEP.ENG}


\section{Thrust lever angle sensor faults}

Each thrust lever has two thrust lever angle (\ac{TLA}) sensors.

Failure of one sensor only leads to a loss of redundancy; the proviso is that it
must have failed in a way that the system can positively detect.

More difficult is when the sensors are in disagreement. In this case, the
\ac{FADEC} makes the assumption that one of the sensors is accurate and provides
a default thrust setting based on this assumption:

\begin{itemize}
\item On the ground, if neither sensor is in a take-off position, idle power is
  commanded. If one sensor is in take-off position and the other is above idle,
  take-off thrust is commanded. This leaves the completely conflicted case of
  one sensor at take-off and the other at idle or below; the \ac{FADEC} selects
  idle power as the best compromise.
\item In flight, once above thrust reduction altitude the \ac{FADEC} will assume
  that the largest \ac{TLA}, limited to \ac{CLB}, is correct. The autothrust can
  then manage the thrust between idle and this position. For approach (slats
  extended), as long as both \ac{TLA}s indicate less than \ac{MCT}, thrust is
  commanded to idle.
\end{itemize}

If both \ac{TLA} sensors fail, the \ac{FADEC} again goes for sensible
defaults. On the ground, idle thrust is set. In flight, if the thrust was
\ac{TO} or \ac{FLEX} at the time of failure, this setting will be maintained
until slat retraction, whereupon \ac{CLB} will be selected. If the thrust was
between \ac{IDLE} and \ac{MCT}, \ac{CLB} will be selected immediately. As soon
as slats are deployed, \ac{IDLE} is commanded; this remains the case even for
go-around. Autothrust will manage thrust between \ac{IDLE} and \ac{CLB} whenever
\ac{CLB} is assumed.

\multicite{\uline{ENG~1(2)}~THR~LEVER~DISAGREE,
  \uline{ENG~1(2)}~THR~LEVER~FAULT, \\ \uline{ENG~1(2)}~ONE~TLA~FAULT,
  FCOM~PRO.AEP.ENG}


\section{FADEC faults}

The \ac{FADEC}s have two redundant channels; loss of a single channel does not
generally require crew action. Single channel \ac{FADEC} faults during start may
be considered spurious on successful application of the reset procedure detailed
in \ac{FCOM}~\inlcite{PRO.AEP.ENG}.

If both channels of a \ac{FADEC} are lost, the thrust lever should be set to
idle. Engine indications will be lost. If all other parameters are normal (check
all \ac{ECAM} system pages), the engine can be left running. Otherwise, shut it
down.

If a \ac{FADEC} overheats, reducing engine power may reduce temperature in the
\ac{ECU} area sufficiently to prevent shutdown. If on the ground the engine must
be shut down and the \ac{FADEC} depowered.

\multicite{\uline{ENG~1(2)}~FADEC~A(B)~FAULT, \uline{ENG~1(2)}~FADEC~FAULT,\\
  \uline{ENG~1(2)}~FADEC~HI~TEMP, FCOM~PRO.AEP.ENG}


\chapter{Navigation}

\section{EGPWS alerts \emph{(memory item)}}

\ac{EGPWS} alerts can be categorised into warnings and cautions. A warning is
any alert with the instruction ``Pull up'' or ``Avoid'' attached. All other
alerts are cautions. A warning may be downgraded to a caution if flying in
daylight \ac{VMC} and positive visual verification is made that no hazard
exists, or if an applicable nuisance warning notice is promulgated in Company
documentation [\inlcite{EOMA~8.3.6}].

The response to a ``Pull up'' type warning is to call ``Pull up, \ac{TOGA}'',
disconnect the autopilot and simultaneously roll the wings level, apply full
backstick and set \ac{TOGA} power. The speedbrake should then be checked
retracted. Once the flight path is safe and the warning stops, accelerate and
clean up as required. Note that it is highly likely that the autothrust
\cphrase{ALPHA FLOOR} protection will have engaged and thus the autothrust will
need to be disengaged to cancel \cphrase{TOGA LK} mode.

An ``Avoid'' warning indicates that a vertical manoeuvre alone is insufficient
to prevent collision, and lateral avoiding action must also be taken. The
response is essentially the same except that instead of rolling wings level, a
turn must be initiated. The direction of the turn is at the discretion of the
pilot, with the terrain or obstacle that is the source of the warning being
displayed in red and black crosshatch on the \ac{ND}.

The response to a caution is to correct the flight path or aircraft
configuration as necessary. A configuration warning will almost always require a
go around.

\multicite{FCOM~PRO.AEP.SURV}


\section{TCAS warnings \emph{(memory item)}}

\ac{TCAS} warnings may be either traffic advisories (``Traffic, Traffic'') or
resolution advisories (anything else).

From 28th January 2017, new easyJet deliveries are fitted with the new \ac{AP}/\ac{FD}
\ac{TCAS} mode. When this mode is fitted, the autopilot is capable of autonomously
flying the \ac{TCAS} escape manoeuvre. \ac{PF} simply calls ``\ac{TCAS} blue'' then calls the
\ac{FMA}s and monitors the autopilot as it flies the manoeuvre. If flying manually
with flight directors and autothrust on, the flight directors will give standard
guidance to fly the manoeuvre. If flying fully manually, the flight directors
will pop up and the autothrust will engage, although it may be necessary to set
the thrust levers to the climb gate.

If \ac{AP}/\ac{FD} \ac{TCAS} mode is not installed or not available, the first
response to either advisory is to call ``\ac{TCAS}, I have control'' to
unequivocally establish who will be carrying out any manoeuvres. If it is a
resolution advisory, the autopilot should be disconnected and \emph{both} flight
directors turned off.\footnote{If one \ac{FD} is left engaged, the autothrust
will not revert to speed mode, possibly resulting in speed decay and engagement
of normal law protections.} The autothrust remains engaged and reverts to speed
mode. A vertical manoeuvre should then be flown to keep the \ac{V/S} needle out
of the red areas shown on the \ac{V/S} scale. \ac{ATC} should then be notified
(e.g ``Radar, Easy 123 -- \cphrase{TCAS RA}''). When clear of conflict, return
to assigned level and re-engage the automatics (\ac{ATC} phraseology: ``Radar,
Easy 123 -- clear of conflict, returning to \ac{FL}XXX'').

If a climb resolution advisory occurs on final approach, a go around must be flown.

\multicite{FCOM~PRO.AEP.SURV, CAP413~1.7}


\section{RNAV downgrades}

\ac{RNAV} operations fall into three main categories:

\begin{itemize}
\item \ac{RNAV} approach (usually \ac{RNP} 0.3)
\item \ac{RNP}-1 (aka \ac{PRNAV})
\item \ac{RNP}-5 (aka \ac{BRNAV})
\end{itemize}

The equipment that must be serviceable is listed in \inlcite{EOMB~2.3.18} for
\ac{RNAV} approach and \inlcite{FCOM PRO.SPO.51} for \ac{RNP}
\ac{SID}/\ac{STAR}.

The following messages indicate loss of \ac{RNAV} capability:

\begin{itemize}
\item \cphrase{NAV ACCUR DOWNGRAD} (\ac{MCDU} and \ac{ND}) on both
  sides\footnote{If \cphrase{NAV ACCUR DOWNGRAD} occurs on one side only, the
  procedure may be continued using the unaffected \ac{FMGC}.}
\item \cphrase{FMS1/FMS2 POS DIFF} (\ac{MCDU})
\item \ecam{NAV}{FM/GPS POS DISAGREE} (\ac{ECAM})
\item \cphrase{CHECK IRS 1(2)(3)/FM POSITION} (\ac{MCDU})\footnote{This is
missing from the \ac{RNP}-1 list in \ac{EOMB} but is listed in the \ac{FCOM}. It
is not listed as a go around criteria for \ac{RNAV} approach, but continuing
would seem somewhat brave…}
\end{itemize}

For \ac{RNAV} approaches, a go-around is mandated for any of these messages or
if \cphrase{GPS PRIMARY LOST} is annunciated on both \ac{ND}s\footnote{If
\cphrase{GPS PRIMARY LOST} is annunciated on only one \ac{ND}, the approach may
be continued using the unaffected \ac{FMGC}. There is also conflict between
\ac{EOMA} and \ac{EOMB} as to whether \cphrase{GPS PRIMARY} is required at all
for \cphrase{RNAV(VOR/DME)} or \cphrase{RNAV(DME/DME} etc. -- I've gone with the
most restrictive here.}.

In an \ac{RNP}-1 (\ac{PRNAV}) environment, an \ac{RNAV} downgrade may leave the
aircraft unsure of position and below \ac{MSA}. The initial response is to
notify \ac{ATC} with the phrase ``Unable \ac{RNAV} due equipment'' and request
reclearance. An immediate climb above \ac{MSA} should be considered if a
suitable alternative navigation method (e.g. radar vectors) is not available.

Some \ac{RNP}-1 procedures specify additional downgrade criteria such as a
requirement for dual \ac{RNAV} systems or \ac{GPS}. If \ac{GPS} is not
specifically mandated as an additional restriction, an \ac{RNP}-1 procedure may
still be flown without \cphrase{GPS PRIMARY}: set \ac{RNP} to 1, check
\cphrase{NAV ACCURACY} is \cphrase{HIGH} and carry out a raw data check prior to
commencement (see Section \ref{sec-pos-disagree}).

Downgrades in an \ac{RNP}-5 (\ac{BRNAV}) environment are less critical as the
aircraft will be above \ac{MSA}. The \ac{IRS}s provides \ac{RNP}-5 required
accuracy for two hours from last full alignment regardless of \ac{MCDU}
\ac{ENP}, and it is acceptable to carry out a raw data check (see \inlcite{EOMB
  2.3.15}) to confirm that \ac{RNP}-5 capability is maintained. If loss of
\ac{RNP}-5 capability is confirmed, inform \ac{ATC} and continue with
conventional navigation.

\multicite{FCOM~PRO.SPO.51, EOM~A.8.3.3.5, EOMB~2.3.18.3}


\section{RA faults}

A single \ac{RA} fault results in degradation of approach capability to Cat 2.

Loss of both \ac{RA}s will lead to Direct Law at landing gear extension and a
loss of \cphrase{ILS APPR} mode capability. Therefore, landing will be \ac{CONF}
3 with associated corrections and the approach should be flown in \ac{LOC} and
\ac{FPA}. The \ac{FCTM} recommends that the final stages of the approach are
flown raw data as the autopilot gains are not being updated and autopilot
performance is likely to be unsatisfactory. Height callouts are not available.

\multicite{\uline{NAV}~RA~1(2)~FAULT, FCOM~PRO.AEP.NAV, FCTM~AEP.NAV}


\section{ADR faults}

A single \ac{ADR} fault simply requires switching to the hot spare and turning
the affected unit off. Loss of \ac{ADR}1 will lead to the loss of the extended
functions of the \ac{EGPWS}. Loss of \ac{ADR}2 will lead to both baro reference
channels being driven by the same \ac{FCU} channel,%
%<!-- {TODO: Find out details  of this} -->
so the baro refs should be checked.

Loss of two \ac{ADR}s will lead to Alternate Law with associated speed
restrictions and landing configuration considerations. Air data switching is
used as necessary, and the affected \ac{ADR}s are turned off. \ac{ATC} switching
may be required to restore transponder. If \ac{ADR}1 and \ac{ADR}3 are lost, the
landing gear safety valve is controlled closed, so the gear must be gravity
extended and cannot subsequently be retracted. This is not mentioned by the
\ac{ECAM} -- the gear will simply fail to extend normally.

If all three \ac{ADR}s are lost, the result is airframe dependent. Some of the
fleet now have a \ecam{NAV}{ADR 1+2+3 FAULT} \ac{ECAM} and an appropriate
procedure utilising the Backup Speed Scale, completing with the \ac{QRH}
\ecam{NAV}{ALL ADR OFF} paper procedure. For older airframes the \ac{ECAM}
displayed will be for Dual \ac{ADR} failure and must be ignored since it will
request meaningless air data and \ac{ATC} switching. Instead revert to standby
instruments (the standby \ac{ASI} and Altimeter have direct pressure feeds from
the the standby pitot and static ports) and refer to \ac{QRH} \ecam{NAV}{ADR
  1+2+3 FAULT}. Interestingly, when Backup Speed Scale is available, the
\ac{ECAM} advises that the standby instrument indications may be unreliable\ldots

Triple \ac{ADR} failure has a few additional ramifications. Of note is loss of
automatic cabin pressure control (see Section \ref{sec-pressure-controller} for
manual pressure control methodology), Alternate Law and gravity gear
extension. Of lesser concern are loss of stall warning\footnote{Most of the
easyJet fleet has now been modified so that the stall warning is not lost in the
event of triple \ac{ADR} failure.}, rudder travel limiter frozen until slat
extension and loss of auto flap retraction.

\multicite{\uline{NAV}~ADR~1(2)(3)(1+2)(1+3)(2+3)~FAULT, \\ QRH~AEP.NAV,
  FCOM~PRO.AEP.NAV}


\section{ADR disagree}

The \ac{ECAM} message indicates that, following an \ac{ADR} fault or rejection,
there is a speed or angle of attack disagreement between the two remaining
\ac{ADR}s. This will cause a degradation to Alternate Law. If there is a speed
disagreement, apply the Unreliable Speed procedure (see Section
\ref{sec-unreliable-airspeed}). If the speed does not disagree, an \ac{AOA}
sensor is providing incorrect data and there is a risk of spurious stall
warnings.

\multicite{\uline{NAV}~ADR~DISAGREE, FCOM~PRO.AEP.NAV}


\section{IR faults}

In case of simultaneous loss of the \ac{ADR} and \ac{IR} associated with an
\ac{ADIRU}, apply the \cphrase{ADR FAULT} procedure first.

A single \ac{IR} fault will simply require \ac{ATT}/\ac{HDG} switching. This may
lead to loss of the extended functions of the \ac{EGPWS} and/or loss of
\ac{TCAS}. It may be possible to recover the \ac{IR} in \ac{ATT} mode (see
Section \ref{sec-att-mode}).

A dual \ac{IR} fault will lead to loss of \ac{PFD} indications on at least one
side so use \ac{ATT}/\ac{HDG} switching to recover. It will also lead to
Alternate Law and associated speed restrictions and landing configuration
considerations.

\multicite{\uline{NAV}~IR~1(2)(3)(1+2)(1+3)(2+3)~FAULT, FCOM~PRO.AEP.NAV}


\section{IR disagree}

Following rejection or failure of an \ac{IR}, there is disagreement between the
two remaining \ac{IR}s. Normal and alternate laws are lost, but alternate law
with reduced protections can be recovered by isolating the faulty \ac{IR} (use
standby horizon to cross-check) and resetting the \ac{ELAC}s.

\multicite{\uline{NAV}~IR~DISAGREE, FCOM~PRO.AEP.NAV}


\section{IR alignment in ATT mode}
\label{sec-att-mode}

If \ac{IR} alignment is lost, it may be possible to recover attitude and heading
information by switching the \ac{ADIRU} selector to \ac{ATT} and maintaining
level constant speed flight for 30 seconds. The magnetic heading will need to be
entered, the exact method being dependent on the \ac{ADIRS} \ac{CDU} fitted.

\multicite{IR Alignment IN ATT Mode, QRH~AEP.NAV, FCOM~PRO.AEP.NAV}


\section{FM/GPS position disagree}
\label{sec-pos-disagree}
This can be disregarded if on an \ac{ILS} or \ac{LOC} approach. On an overlay
approach, revert to raw data. On an \ac{RNAV} approach, go around unless visual.

In other flight phases, manually tune a \ac{VOR} and check against either the
needle and \ac{DME} on the \ac{ND} or the \cphrase{BRG/DIST TO} field on the
\ac{PROG} page. If the error is greater than 3nm in the cruise or greater than
1nm for approach, raw data navigation and \ac{AP}/\ac{FD} selected lateral and
vertical modes should be used.

\multicite{\uline{NAV} FM/GPS POS DISAGREE, QRH~AEP.NAV, FCOM~PRO.AEP.NAV}

\chapter{Auto-flight}

\section{FAC faults}

Failure of a single \ac{FAC} results in loss of redundancy and hence loss of Cat
3 Dual. In particular, a single \ac{FAC} provides all the characteristic speeds
(\V{SW}, \V{LS}, \V{FE}, \V{FE~next}, \V{LE}, \V{MO}/M\textsubscript{MO}, Green
dot, S speed and F speed). It may be worth cross-checking against
\ac{QRH}~\inlcite{OPS.OD Operating Speeds}.

If both \ac{FAC}s are lost the rudder travel limit system, rudder trim control,
yaw damper and \ac{PFD} characteristic speeds are lost and Alternate Law with
mechanical yaw control becomes active. Recovery of full rudder authority at flap
extension should be anticipated, but use rudder with care above 160kt.

\multicite{\uline{AUTO FLT}~FAC~1(2)(1+2)~FAULT, FCOM~PRO.AEP.AUTO~FLT}


\section{Yaw damper faults}

A single failure leads to loss of redundancy, and hence loss of Cat 3 Dual. On
some airframes a reset of the affected \ac{FAC} can be attempted.

With a dual failure a reset of the \ac{FAC}s should be attempted. If the yaw
damper is not recovered, the flight controls revert to alternate law (see
Section \ref{sec-alternate-law}). Unless the failure occurred below alert
height, only Cat 1 will be available.

\multicite{\uline{AUTO FLT}~YAW~DAMPER~1(2)(SYS), FCOM~PRO.AEP.AUTO~FLT}


\section{Rudder trim faults}

Loss of a rudder trim from a single \ac{FAC} leads to loss of redundancy and
hence loss of Cat 3 Dual.

If complete loss of rudder trim occurs, an attempt should be made to reset the
\ac{FAC}s. If this is not successful, only Cat 1 is available.

\multicite{\uline{AUTO~FLT}~RUDDER~TRIM~SYS(1)(2)~FAULT), FCOM~PRO.AEP.AUTO~FLT}


\section{Rudder travel limiter faults}

Loss of rudder limit functionality from a single FAC leads to loss of redundancy
only.

The effect of complete loss of rudder limiter functionality depends on when the
failure occurred. In general, the rudder should be used with caution when above
160kt. An attempt should be made to recover the limiter by resetting the
\ac{FAC}s. If unsuccessful, full rudder travel authority may or may not be
recovered at slat extension. If a landing must be made with the rudder travel
limiter frozen in the high speed regime, max crosswind is reduced to 15kt and
differential braking may be required on the landing roll (do not arm autobrake).

\multicite{\uline{AUTO~FLT}~RUD~TRV~LIM(1)(2)(SYS), FCOM~PRO.AEP.AUTO~FLT}


\section{FCU faults}

Loss of a single channel will result in the spare channel automatically taking
over. All that is required is a cross check of the baro refs.

Loss of both channels leads to loss of all \ac{FCU} and \ac{EFIS} panels. The
autopilots, flight directors\footnote{Flight Directors will pop up to provide
guidance in the event of a go-around.} and autothrust are lost and parameters
that are normally controlled by the panels revert to sensible default values. If
the weather radar image remains displayed, disregard it since the scale will be
incorrect. Since it will only be possible to set the \ac{QNH} on the standby
altimeters, the \ac{MDA} should not be set in the \ac{MCDU}; instead the \ac{PM}
should make standard callouts from the standby altimeter.

\multicite{\uline{AUTO FLT}~FCU~1(2)(1+2)~FAULT, FCOM~PRO.AEP.AUTO~FLT}

\chapter{Hydraulics}

\section{Green + yellow systems low pressure}

It may be possible to recover the yellow system using the yellow electrical
pump. The \ac{PTU} will need to be turned off in this case, as the yellow
electric pump lacks sufficient capacity to pressurise the green system through
the \ac{PTU}. Systems lost because of low air pressure in the reservoir will be
recoverable at lower altitudes. Systems lost due to reservoir overheats may be
usable for the approach once they have cooled down.

Roll control is available from ailerons and spoiler 3. Pitch control is
available from the elevators, but the \ac{THS} is frozen. Yaw damping is
lost. Slats are available, but slow. Flaps are frozen. Control law reverts to
Alternate Law without stability protections. The autopilots are lost.

The gear must be gravity extended, but due to the frozen \ac{THS} this must be
delayed until \V{APP} is achieved in \ac{CONF} 3. Furthermore, transition to
\ac{CONF} 3 must be achieved with the method described in Section
\ref{sec-stuck-slats/flaps} due to the flaps being frozen. Transition to Direct
Law on gear deployment adds to the fun, especially as pitch trim is unavailable.

Cat 2 and 3 capability is lost. The landing will be \ac{CONF} 3, most probably
with only the slats deployed; the flare attitude will be abnormal. There will
only be one spoiler (\#3), no reversers, accumulator only braking and no nose
wheel steering. Hence required landing distances almost triple.

The go around, in some ways, is exceptionally straightforward. The gear cannot
be raised and the configuration should be maintained. Due to the frozen
stabiliser, a speed close to \V{APP} should be flown.\footnote{Airbus originally
suggested flying \V{FE} $-$ 10kt; I assume this was found to cause handling
difficulties due to pitch trim not being available.} As long as the flaps are
frozen at zero, the slats can be retracted for a subsequent diversion; fuel flow
will be approximately 3 times normal due to the extended gear. If flaps are not
at zero, fuel flow will be up to 4 times normal.

A paper summary is available in section \ac{HYD} of the \ac{QRH}, and this
should be applied once all \ac{ECAM} actions are completed.

\multicite{\uline{HYD}~G~+~Y~SYS~LO~PR, QRH~AEP.HYD, FCOM~PRO.AEP.HYD}


\section{Blue + yellow systems low pressure}

It may be possible to recover the yellow system using the yellow electrical pump
or the blue system using the \ac{RAT}. Systems lost because of low air pressure
in the reservoir will be recoverable at lower altitudes. Systems lost due to
reservoir overheats may be usable for the approach once they have cooled down.

Roll control is provided by ailerons and spoiler 5, pitch control by the
\ac{THS} and left elevator. Slats and flaps are available at reduced rate. The
autopilots are lost but Normal Law is retained. Speedbrake is unavailable.

Cat 2 and 3 capability is lost. Landing distances are increased due to loss of
spoilers 2, 3 and 4 and loss of \#2 reverser. Approach configuration is normal
apart from slow flaps and slats and gravity gear extension (protects green
system). Nose wheel steering is lost.

Gear cannot be retracted on go-around. Fuel burn for a subsequent diversion will
be significantly greater (approx factor 3 times normal); see Section
\ref{sec-flight-with-gear-extended} for further details.

A paper summary is available in the \ac{HYD} section of the \ac{QRH}, and this
should be applied once all \ac{ECAM} actions are completed.

\multicite{\uline{HYD}~B~+~Y~SYS~LO~PR, QRH~AEP.HYD, FCOM~PRO.AEP.HYD}


\section{Green + blue systems low pressure}

If the blue system has been lost due to the loss of its electrical pump, it may
be recovered by deploying the \ac{RAT}. Systems lost because of low air pressure
in the reservoir will be recoverable at lower altitudes. Systems lost due to
reservoir overheats may be usable for the approach once they have cooled down.

Roll control is provided by spoilers 2 and 4 only. Use of speedbrake would
therefore lead to loss of roll control. Pitch control is available from the
starboard elevator; the \ac{THS} remains available. Due to the limited control
surfaces available, the aircraft will be slightly sluggish. The slats are
frozen, but flaps are available. Control law reverts to Alternate Law without
stability protections. The autopilots are lost. The approach will be flown with
the autothrust off.

Due to the frozen slats, configuration changes must be carried out using the
method described in Section \ref{sec-stuck-slats/flaps}. The gear must be
gravity extended; to benefit from the improved elevator response available in
Direct Law, this is done at 200kt. Manual trim will be available.

Cat 2 and 3 capability is lost. The landing will be \ac{CONF} 3. Two spoilers
per wing are available, reverser 2 is available, alternate braking is available
and nose wheel steering is available. Landing distances approximately double.

Go around is straightforward -- the gear cannot be retracted and the flap
configuration should be maintained. Simply select \V{FE}~$-$~10kt. For
diversion, the flaps can be retracted. With the gear remaining down, fuel burn
will increase by a factor of approximately 3 if the slats are at zero or up to
approximately 3\textonehalf\ if they are extended.

A paper summary is available in the ac{HYD} section of the \ac{QRH}, and this
should be applied once all \ac{ECAM} actions are completed.

\multicite{\uline{HYD}~G~+~B~SYS~LO~PR, QRH~AEP.HYD, FCOM~PRO.AEP.HYD}


\section{Green system low pressure}

The major lost systems are normal landing gear operation (gravity extension is
available but gear retraction is not) and the normal brake system, including the
autobrake (alternate braking is available). Landing distance will be increased
due to loss of two spoilers per wing and reverser 1. Flap and slat deployment
will be slow.

\multicite{\uline{HYD}~G~SYS~LO~PR, FCOM~PRO.AEP.HYD}


\section{Yellow system low pressure}

It may be possible to recover the yellow system by using the yellow electric pump.

Two spoilers per wing and reverser 2 are lost, so landing distance will increase
slightly. Nose wheel steering is lost. Flap deployment will be slow. As the
alternate braking system is only available through the brake accumulator, ensure
there is sufficient pressure when the parking brake is set.

\multicite{\uline{HYD}~Y~SYS~LO~PR, FCOM~PRO.AEP.HYD}



\section{Blue system low pressure}

One spoiler per wing will be lost but this has negligible effect on landing
distance. Slats will be slow to deploy. Deployment of the \ac{RAT} is not
recommended unless another system is lost. If the system is lost due to low
reservoir level, emergency generation capability is lost.

\multicite{\uline{HYD}~B~SYS~LO~PR, FCOM~PRO.AEP.HYD}


\section{Engine driven pump low pressure}

Turn off the affected pump. The \ac{PTU} will pressurise the affected system.

\multicite{\uline{HYD}~G(Y)~ENG~1(2)~PUMP~LO~PR, FCOM~PRO.AEP.HYD}


\section{Electric pump low pressure or overheat}

Turn off the affected pump. In the case of an overheat, the pump may be
re-engaged for the approach providing the relevant \ac{FAULT} light on the
overhead panel has extinguished.

\multicite{\uline{HYD}~Y(B)~ELEC~PUMP~LO~PR(OVHT), FCOM~PRO.AEP.HYD}


\section{Low reservoir air pressure}

Loss of air pressure to a hydraulic reservoir may lead to pump cavitation and
hence fluctuating pressures. If this occurs, turn off the affected pump, and if
applicable, turn off the \ac{PTU}. Cavitation reduces with altitude, so it may
be possible to reinstate the system during the descent.

\multicite{\uline{HYD}~G(Y)(B)~RSVR~LO~AIR~PR, FCOM~PRO.AEP.HYD}


\section{Reservoir overheat}

Turn off all affected pumps and if applicable, turn off the \ac{PTU}. The system
should be reinstated for the approach if it has cooled sufficiently. This is
indicated by the \ac{FAULT} light going out on the overhead panel.

\multicite{\uline{HYD}~G(Y)(B)~RSVR~OVHT, FCOM~PRO.AEP.HYD}


\section{Low reservoir fluid level}

Turn off all affected pumps and, if applicable, turn off the \ac{PTU}. The
affected system is not recoverable. In the case of low reservoir level in the
yellow system, it is possible that the fluid from the brake accumulator may also
be lost. This usually occurs within 10 minutes of the initial warning. Without
the brake accumulator, the parking brake is not available, so chock the aircraft
before shutting down engine 1.

\multicite{\uline{HYD}~G(Y)(B)~RSVR~LO~LVL, FCOM~PRO.AEP.HYD}


\section{PTU fault}

In flight this indicates that either the green or yellow system is low on fluid
and has low system pressure. The \ac{PTU} must be turned off to prevent
overheating the supplying system.

\multicite{\uline{HYD}~PTU~FAULT, FCOM~PRO.AEP.HYD}


\section{RAT fault}

Indicates that either the \ac{RAT} is not fully stowed, pressure is present in
the \ac{RAT} stowing actuator or that the \ac{RAT} pump is not available. No
action is required.

\multicite{\uline{HYD}~RAT~FAULT, FCOM~PRO.AEP.HYD}

\end{document}
