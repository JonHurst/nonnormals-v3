\documentclass[a5paper,11pt,twoside]{book}
\usepackage[utf8]{inputenc}
\usepackage[T1]{fontenc}
\usepackage{lmodern}
\usepackage{textcomp}
\usepackage{microtype}
\usepackage[left=0.75in, bottom=1in]{geometry}
\usepackage{color}
\usepackage[color]{changebar}
\cbcolor{blue}
\usepackage{array}
\usepackage[pdfauthor={Jon Hurst},hidelinks]{hyperref}
\usepackage[normalem]{ulem}
\title{A320 Family Non-Normal-Notes\\~\\\large{Version 2.6}}
\author{Jon Hurst}
\date{}


\newcommand{\ac}[1]{{\scshape\MakeLowercase{#1}}}
\newcommand{\ecam}[2]{{\ac{\uline{#1} #2}}}
\newcommand{\cphrase}[1]{\ac{#1}}
\newcommand{\inlcite}[1]{{\ac{#1}}}
\newcommand{\multicite}[1]{%
  \nopagebreak
  \noindent{{\color{blue}\footnotesize[ \inlcite{#1} ]}}
}
\newcommand{\V}[1]{V\textsubscript{#1}}
\newcommand{\strong}[1]{\textbf{#1}}

\begin{document}
\frontmatter
\maketitle

\ifdefined\HCode
\else
\tableofcontents
\fi

\mainmatter
\chapter{Operating techniques}
%%%%%%%%%%%%%%%%
%labels to be moved once section is processed
\label{sec-loss-of-braking}
\label{sec-all-engine-failure}
\label{sec-gravity-fuel-feeding}
%%%%%%%%%%%%%%%%
\section{Rejected Takeoff}
The decision to reject rests solely with \ac{CM}1. This decision is communicated
with the words ``Stop'' or “Go''. ``Stop'' implies that \ac{CM}1 is taking
control of the aircraft.

Below 100kt the \ac{RTO} is relatively risk free and a decision to stop should
be made for any \ac{ECAM} and most other problems.

Above 100kt the \ac{RTO} may be hazardous and stopping should only be considered
for loss of engine thrust, any fire warning, any uninhibited
\ac{ECAM}\footnote{There are five uninhibited amber \ac{ECAM} cautions that
require a high speed \ac{RTO}. Only two uninhibited \ac{ECAM}s are not on this
list: \ecam{ENG}{1(2) THR LEVER DISAGREE} if the \ac{FADEC} automatically
selects idle thrust and \ecam{FWS}{FWC 1+2 FAULT}. The first of these should
never happen due to \ac{FADEC} logic. The second generates a message on the
\ac{EWD} but no master caution (it is the computers that generate master
cautions that have failed). You could therefore modify this rule to: stop for
any \ac{ECAM} warning or caution except the caution-like \ecam{FWS}{FWC 1+2
  FAULT}.} or anything which indicates the aircraft will be unsafe or unable to
fly.

If a stop is required, \ac{CM}1 calls ``Stop'' while simultaneously bringing the
thrust levers to idle, then to max reverse.

If the stop was commenced below 72kt the ground spoilers will not automatically
deploy and the autobrake will therefore not engage. Monitor automatic braking,
and if there is any doubt, apply manual braking as required. If normal braking
fails, announce ``Loss of braking'' and proceed with the loss of braking memory
items (see Section \ref{sec-loss-of-braking}).

\looseness=1
If the reason for the stop was an engine fire on the upwind side, consider
turning the aircraft to keep the fire away from the fuselage.

If there is any chance of requiring evacuation, bring the aircraft to a complete
halt, stow the reversers, apply the parking brake, and order ``Attention, crew
at stations'' on the \ac{PA}.

If evacuation will definitely not be required, once the aircraft's safety is
assured the \ac{RTO} can be discontinued and the runway cleared. In this case
make a \ac{PA} of ``Cabin crew, normal operations''.

\looseness=1
During this initial phase, \ac{CM2} confirms reverser deployment (``Reverse
green''), confirms deceleration (``Decel''), cancels any audio warnings, informs
\ac{ATC} and announces ``70 knots'' when appropriate. \ac{CM2} then locates the
emergency evacuation checklist.

Once the aircraft has stopped, \ac{CM1} takes the radios and asks \ac{CM2} to
carry out any required \ac{ECAM} actions. Whilst the \ac{ECAM} actions are being
completed, \ac{CM1} will build up a decision as to whether to evacuate. If an
evacuation is required, see Section \ref{sec-evacuation}. Otherwise order
``Cabin crew, normal operations''.

If the aircraft has come to a complete halt using autobrake \cphrase{MAX}, the brakes
can be released by disarming the spoilers.

\looseness=1
If, following an \ac{RTO}, a new takeoff is to be attempted, reset both
\ac{FD}s, set the \ac{FCU}, then restart \ac{SOP}s from the After Start
checklist. Carefully consider brake temperatures; temperature indications
continue to climb for some time after a significant braking event.

\multicite{EOMB~3.10, FCTM~PRO.AEP.MISC}


\section{Asymmetric takeoff}

Apply rudder conventionally to maintain runway track. At V\textsubscript{r}
rotate at a slightly reduced rate towards an initial pitch target of
12\textonehalf\textdegree\, then target speed \V{2} to \V{2}+15kt. Bank angle
should be limited to 15\textdegree{ }when more than 3kt below manoeuvring speed
for the current configuration\footnote{This is a conservative rule of thumb. If
the \ac{FMGC} has correctly identified an engine out condition, the \ac{FD/AP}
will automatically limit bank angle according to a less conservative algorithm
(see \inlcite{FCOM~SYS.22.20.60.40})}.

When the ground to flight mode transition is complete\footnote{Introducing
\ac{TOGA} during the ground to flight mode transition (commences as the pitch
increases through 8\textdegree, complete after 5 seconds) results in a pitch up
moment at a time where the effect of stick pitch control is not wholly
predictable: the stick will need to be moved forward of neutral to counteract
the introduced pitch moment and then returned to neutral as flight mode blends
in. A slight pause before selecting \ac{TOGA} results in much more normal and
predictable handling.}, select \ac{TOGA}\footnote{\ac{FLX} may be used but this
tends to allow speed to decay unless pitch is reduced}. Adjust and trim rudder
to maintain $\beta$ target; this will result in a small side-slip angle towards
the failed engine. Engage the autopilot once gear is up and rudder is trimmed.

%%  and request ``pull heading''. If the \ac{EOSID}
%% follows the track of the cleared \ac{SID}, \ac{NAV} may be used, but this is
%% very rare with easyJet \ac{EOSID}s.

\section{Low level failure handling}
\label{sec-failures-after-v1}

Handling of failures that occur on the takeoff roll or at very low level is
primarily a test of triage skills. Airbus provides support in three ways:

\begin{itemize}
\item \ac{ECAM} flight phase inhibitions filter out less serious failures until
  1500ft \ac{AAL} is attained on climb out or speed is below 80kt on the landing
  roll.
\item \ac{TOGA} thrust is made available for an extra 5 minutes (giving a total
  of 10 minutes) in emergency situations, which allows more time before
  acceleration and cleanup is required.
\item A recommendation is published at the start of the \ac{QRH}, that, apart
  from cancelling audio warnings, ``no action will be taken'' until an
  appropriate flight path is established and the aircraft is at least 400ft
  \ac{AGL}\footnote{\ldots although it does go on to water this statement down
  by saying that, for some unspecified emergency situations, the 400ft part of
  the recommendation may be disregarded}.
\end{itemize}

Whilst below 400ft, then, the focus should be on flying and monitoring, with
heightened awareness of the possibility of missing essential normal actions,
such as calling rotate or raising the gear due to the distraction of the
failure. It may help \ac{PF} assimilate the challenges of the flying task if
\ac{PM} states the title of the first displayed \ac{ECAM} procedure, but no
action to diagnose or contain the failure should be taken. A very quick and well
timed ``Mayday, Mayday, Mayday, standby'' and ``Attention Crew at Stations''
from \ac{PM} may also be useful to forestall external interruptions. When and if
possible, the autopilot should be engaged to reduce workload.

An extremely useful tool for dealing with low level failures is the \ac{EOSID},
as described in Section \ref{sec-eosid}. While the name suggests that this is
only to be used with engine failures, in reality it is a pre-planned safe flight
path that may be flown whenever there is doubt that available aircraft
performance is sufficient to fly the cleared \ac{SID} or go-around. Since both
pilots should already be aware of the details of the \ac{EOSID}, \ac{PM} can
simply declare their intention to fly it to generate a shared mental model.
Note that flying an \ac{EOSID} is higher workload than flying the planned
\ac{SID} in \ac{NAV} or flying visually; it is an option, not a requirement.

Another useful tool, often used in association with the \ac{EOSID}, is the
concept of ``high priority tasks''. These are defined as:

\begin{itemize}
\item For engine failure, the master switch of the affected engine has been
  turned off.

\item For engine fires, \emph{either} one squib has been fired and the fire
  warning has extinguished \emph{or} both squibs have been fired.
\end{itemize}

These definitions help with the triage process, allowing standardised
interleaving of the diagnosis and containment of the failure with the flying and
monitoring tasks. In particular, the level acceleration phase of the \ac{EOSID}
is generally delayed until these tasks are completed, with the phrase ``engine
is secure'' a de facto standard call to indicate that \ac{PF} might like to
consider interrupting the containment process\footnote{The phrases ``Stop
\ac{ECAM}'' and ``Continue \ac{ECAM}'' are standard for interrupting
containment.}.  Unfortunately, they are only defined for engine failures and
fires, so with other failures, the crew will have to make a judgement call as to
what constitutes ``high priority''.

\multicite{FCTM~PRO.AEP.ENG}

\section{EOSID}
\label{sec-eosid}

Before the divergence point (the last common point between the \ac{SID} and the
\ac{EOSID}), if the aircraft detects a loss of thrust the \ac{EOSID} will be
displayed as a temporary flight plan. In this case the temporary flight plan can
be inserted and \ac{NAV} mode used. Otherwise it will be necessary to pull
heading and manually follow either the yellow line or bring up a pre-prepared
secondary flight plan and follow the white line.

If beyond the divergence point, pull heading and make an \emph{immediate} turn
the shortest way onto the \ac{EOSID}. Airbus specifically recommends against
this (\inlcite{FCOM~AS.22.20.60}), but easyJet states it as policy
(\inlcite{EOMB 4.4.4}).

Electing to fly the \ac{EOSID} implies a level acceleration segment:

\begin{itemize}
\item Initially fly a \ac{TOGA} climb at the higher of \V{2} or current speed,
  up to a limit of \V{2}+15kt. If a \ac{FLEX} takeoff was carried out, a
  \ac{FLEX} climb is permissible. This climb is continued until all high
  priority tasks are complete (see Section \ref{sec-failures-after-v1}) and the aircraft is
  above single engine acceleration altitude (usually 1000ft \ac{QFE}, but may be
  higher if so specified by the take-off performance calculation). If the
  \ac{FMGS} has detected the engine out condition, the automatic mode change
  from \ac{SRS} to \ac{CLB} will be inhibited; if not, intervention with
  selected modes will be required to prevent untimely acceleration.

\item The next segment is a \ac{TOGA} level acceleration and clean up, either to
  \ac{Conf} 1 and S speed if an immediate \ac{VMC} return is desired or to
  \ac{Conf} 0 and green dot. Again \ac{FLEX} may be used if a \ac{FLEX} takeoff
  was carried out. Level acceleration is usually achieved by pushing \ac{V/S};
  if the \ac{FMGS} has detected the engine out condition, all preselected speeds
  entered in the \ac{MCDU} will have been deleted, so the managed target speed
  should automatically move to 250kt. The phrases ``Stop \ac{ECAM}'' and
  ``Continue \ac{ECAM}'' can be used to interrupt \ac{ECAM} procedures in order
  to initiate this segment.

\item The final segment is a \ac{MCT} climb segment to \ac{MSA}, either at S
  speed if in \ac{Conf} 1 or at green dot speed if in \ac{Conf} 0. This is
  usually achieved in open climb; if the \ac{FMGS} has detected the engine out
  condition, the managed target speed becomes dependent on flight phase, and in
  this case should automatically select green dot.
\end{itemize}

\ac{TOGA} may be used for a maximum of 10 minutes.

If an \ac{EOSID} is annotated as ``\cphrase{STD}'', then acceleration to green
dot should be completed prior to commencing the first turn. If
``\cphrase{NON-STD}'', the turn takes priority.

\multicite{EOMB~4.4.4, FCOM~DSC.22\_20.60.40}

\chapter{Miscellaneous}

\section{Emergency descent \emph{(memory item)}}
\label{sec-emer-descent}

If an emergency descent is required, the Captain should consider taking control
if not already \ac{PF}. \ac{PF} initiates the memory items by announcing
``Emergency Descent.''

Don oxygen masks and establish communication.

\ac{PF} then flies the emergency descent.  Descent with autopilot and autothrust
engaged is preferred. The configuration is thrust idle, full speed brake and
maximum appropriate speed, taking into account possible structural
damage.\footnote{According to Airbus, structural damage may be suspected if
there has been a ``loud bang'' or there is a high cabin vertical speed. When
limiting descent speed due to suspected structural damage, it is \ac{IAS} rather
than Mach that is relevant.} Target altitude is \ac{FL100} or \ac{MORA} if this
is higher. If speed is low, allow speed to increase before deploying full
speedbrake to prevent activation of the angle of attack protection. Landing gear
may be used, but speed must be below \V{LO} when it is extended and remain below
\V{LE}. If on an airway, consider turning 90\textdegree{ }to the left.

\ac{PM}'s only memory action is to turn the seatbelt signs on; their primary
task is to ensure that \ac{PF} has promptly and correctly initiated the descent.

Once the memory actions are complete and the aircraft is descending, \ac{PF}
should call for the Emergency Descent Checklist (``My radios, Emergency Descent
Checklist''). This will lead \ac{PF} to finesse the speed and altitude targets and
inform \ac{ATC} of the descent; \ac{PM} to set continuous ignition on the engines and set
7700 on the transponder. Both pilots then set their oxygen flows to the \ac{N}
position\footnote{There may be insufficient oxygen to cover the entire emergency
descent profile if the oxygen masks are left set to 100\%.} and, if cabin
altitude will exceed 14,000ft, \ac{PM} deploys the cabin oxygen masks. On easyJet
aircraft, the \ac{CIDS}/\ac{PRAM} will automatically play a suitable \ac{PA}, so it is not
necessary for the flight crew to carry out the \ecam{}{EMER DESCENT (PA)} action.

Once level, restore the aircraft to a normal configuration. When safe to do so,
advise cabin crew and passengers that it is safe to remove their masks. To
deactivate the mask microphone and switch off the oxygen flow, close the oxygen
mask stowage compartment and press the ``\cphrase{PRESS TO RESET}'' oxygen control slide.

\multicite{EOMB~3.80.2, QRH~AEP.MISC, FCOM~AEP.MISC, FCTM~AEP.MISC}


\section{Windshear \emph{(memory item)}}

\subsection{Takeoff roll}

Windshear encountered on the takeoff roll is \emph{only} detectable by
significant airspeed fluctuations. It is possible that these fluctuations may
cause \V{1} to occur significantly later in the takeoff roll then it should. In
this case it falls to the Captain to make an assessment of whether sufficient
runway remains to reject the takeoff, or whether getting airborne would be the
better option. If the takeoff is to be continued, call ``Windshear, \ac{TOGA''}
and apply \ac{TOGA} power.  Rotate at \V{r} or with sufficient runway
remaining\footnote{``Sufficient runway remaining'' is actually Boeing advice –
Airbus offers no guidance for the case where there is insufficient runway
available to stop nor to rotate at normal speeds.} and follow \ac{SRS}
orders. \ac{SRS} will maintain a minimum rate of climb, even if airspeed must be
sacrificed.

\subsection{Reactive}

The reactive windshear detection system is a function of the \ac{FAC}s. It only
operates when below 1300ft \ac{RA} with at least \ac{CONF} 1 selected. In the
takeoff phase, it is inhibited until 3 seconds after lift off and in the landing
phase it is inhibited below 50ft \ac{RA}.

A warning is indicated by a red \cphrase{WINDSHEAR} flag on the \ac{PFD} and a
``Windshear, Windshear, Windshear'' aural warning. Call ``Windshear, \ac{TOGA}''
and apply \ac{TOGA} power.

The autopilot can fly the escape manoeuvre as long as the required \ac{AOA} is
less than $\alpha$\textsubscript{prot}. If the autopilot is not engaged, follow
the \ac{SRS} orders on the \ac{FD}s. If the \ac{FD}s are not available,
initially pitch up to 17.5\textdegree, then increase as required.

Do not change configuration until out of the windshear. Once clear of the
windshear, clean up the aircraft: leveraging the go-around procedure is useful
for this.

In severe windshear, it is possible that Alpha Floor protection will
activate. As \ac{TOGA} will already be selected, this will have no immediate
effect. Once clear of the windshear, however, \ac{TOGA} lock will be
active. This, combined with the unusual aircraft configuration, leads to a
significant threat of overspeed. The most natural way to disengage \ac{TOGA}
lock is to disengage the autothrust using the instinctive disconnect \ac{PB} on
the thrust levers then use manual thrust until the situation has sufficiently
stabilised to re-engage the autothrust.

\multicite{FCOM~PRO.AEP.SURV}

\subsection{Predictive}

Below 2300ft \ac{AGL}, the weather radar scans a 5nm radius 60\textdegree\ arc
ahead of the aircraft for returns indicating potential windshear.

Alerts are categorised as \strong{advisory}, \strong{caution} or
\strong{warning}, in increasing order of severity. Severity is determined by
range, position and phase of flight. Alerts are only provided when between 50ft
and 1500ft, or on the ground when below 100kt.

All types of alert produce an indication of windshear position on the \ac{ND},
provided that the \ac{ND} range is set to 10nm. A message on the \ac{ND}
instructs the crew to change range to 10nm if not already set. A
\strong{caution} additionally gives an amber \cphrase{W/S AHEAD} message on
both \ac{PFD}s and an aural ``Monitor Radar Display'' warning. A
\strong{warning} additionally gives a red \cphrase{W/S AHEAD} message on the
\ac{PFD}s and either a ``Windshear Ahead, Windshear Ahead'' or ``Go Around,
Windshear Ahead'' aural message.

If a \strong{warning} occurs during the takeoff roll, reject the takeoff. If it
occurs during initial climb, call ``Windshear, \ac{TOGA}'', apply \ac{TOGA}
thrust and follow \ac{SRS} orders. If it occurs during approach, fly a normal
go-around. Configuration may be changed as long as the windshear is not entered.

If a \strong{caution} occurs during approach, use \ac{CONF} 3 to optimise
go-around climb gradient and consider increasing \V{APP}; up to a maximum of
\V{LS}+15 may be used.

If positive verification is made that no hazard exists and providing that the
reactive windshear is serviceable the crew may downgrade the \strong{warning} to
a \strong{caution.}

\multicite{FCTM PR.NP.SP.10.10}

\section{Unreliable airspeed \emph{(memory item)}}
\label{sec-unreliable-airspeed}

Unreliable airspeed indications may result from radome damage and/or
unserviceable probes or ports. Altitude indications may also be erroneous if
static probes are affected.

The \ac{FMGC}s normally reject erroneous \ac{ADR} data by isolating a single
source that has significant differences to the other two sources. It is possible
that a single remaining good source may be rejected if the other two sources are
erroneous in a sufficiently similar way. In this case, it falls to the pilots to
identify and turn off the erroneous sources to recover good data.

The first problem is recognition of a failure, since the aircraft systems may be
unable to warn of a problem. The primary method of doing this is correlation of
aircraft attitude and thrust to displayed performance. Correlation of radio
altimeter and \ac{GPIRS} derived data (available on \cphrase{GPS MONITOR} page)
may also aid identification. The stall warning (available in alternate or direct
law) is based on alpha probes, so will likely be valid. Other clues may include
fluctuations in readings, abnormal behaviour of the automatics, high speed
buffet or low aerodynamic noise.

If the aircraft flight path is in doubt, disconnect the automatics and fly the
following short term attitude and thrust settings to initiate a climb:

\bigskip
\begin{tabular}{|l|c|c|}
  \hline
  \textbf{Condition} & \textbf{Thrust} & \textbf{Pitch}\\\hline
  Below Thrust Reduction Altitude & \ac{TOGA} & 15\textdegree \\\hline
  Below \ac{FL}100 & \ac{CLB} & 10\textdegree \\\hline
  Above \ac{FL}100 & \ac{CLB} & 5\textdegree \\\hline
\end{tabular}
\bigskip

If configured \ac{CONF} Full, select \ac{CONF} 3, otherwise flap/slat
configuration should be maintained. The gear and speedbrake should be
retracted. If there is any doubt over the validity of altitude information, the
\ac{FPV} must be disregarded. If altitude information is definitely good, the
\ac{FPV} may be used.

It is important to understand that at this stage, while the pilot has identified
that airspeed is unreliable, the aircraft systems have not. Thus flight envelope
protections based on airspeed data from unreliable \ac{ADR}s may activate. This
may lead to pitch inputs from the flight computers that cannot be overridden
with the sidesticks. In this case, immediately switch off any two \ac{ADR}s;
this causes the flight computers to revert to Alternate Law with no protections,
and thus allows control of the aircraft to be regained.

Once the flight path is under control and a safe altitude is attained, the
aircraft should be transitioned into level flight. Refer to the \ac{QRH}
\ecam{NAV}{Unreliable Speed Indication} procedure to extract a ballpark thrust
setting, a reference attitude and a reference speed for the current
configuration, bearing in mind that an auto-retraction of the flap may have
occurred. Set the ballpark thrust setting and adjust pitch attitude to fly
level; if barometric altitude data is considered accurate use the \ac{VSI},
otherwise fly a constant \ac{GPS} altitude. The thrust should then be adjusted
until level flight is achieved with the reference attitude. Note that in the
radome damage case, the required N1 may be as much as 5\% greater than the
ballpark figure. Once stable, the speed will be equal to the reference speed.

If there is insufficient data available to fly level (e.g. \ac{GPS} data
unavailable and barometric data unreliable), fly the reference attitude with the
ballpark thrust setting. This will give approximately level flight at
approximately reference speed.

With the speed now known, the \ac{ADR}s can be checked to see if any are giving
accurate data. If at least one \ac{ADR} is reliable, turn off the faulty
\ac{ADR}s. \ac{GPS} and \ac{IRS} ground speeds may also be used for an
approximate cross check.

If all \ac{ADR}s are considered unreliable, turn off any two of them; one is
kept on to provide stall warning from the alpha probes. More recent aircraft
have backup speed/altitude scales based on \ac{AOA} probes and \ac{GPS}
altitudes which are activated when below \ac{FL}250 by turning off the third
\ac{ADR}. The \ac{QRH} \ecam{NAV}{ALL ADR OFF} procedure describes the use of
these scales, but it boils down to fly the green on the speed scale and
anticipate slightly reduced accuracy from the altitude scale.  For aircraft
without this functionality, tables are provided in the \ac{QRH}
\ecam{NAV}{Unreliable Speed Indication} procedure to enable all phases of flight
to be flown using just pitch and thrust settings. Acceleration and clean up are
carried out in level flight. \ac{CONF} 1 can be selected as soon as climb thrust
is selected, \ac{CONF} 0 once the appropriate S speed pitch attitude from the
table on the first page of the procedure is reached. Configuration for approach
is also carried out in level flight, stabilising in each configuration using the
technique described above. The approach is flown in \ac{CONF} 3 at an attitude
that should result in \V{LS}+10 when flying a 3\textdegree{ }glide. Landing
distance will be increased.

\multicite{QRH~AEP.NAV, FCOM~PRO.AEP.NAV, FCTM~PRO.AEP.NAV}

\section{Incapacitation}

Take control, using the stick priority button if necessary. Contact cabin crew
\ac{ASAP}. They should strap the incapacitated pilot to his seat, move the seat
back, then recline it. If there are two cabin crew available, the body can be
moved. Medical help should be sought from passengers, and the presence of any
type rated company pilots on board ascertained.

\multicite{FCTM~PRO.AEP.MISC}

\section{Forced Landing (inc. Ditching)}

There are two scenarios where off field forced landing or ditching would be
considered: either you have insufficient energy to reach a suitable airfield
(e.g actual or impending fuel exhaustion, catastrophic failure of both engines),
or you have insufficient time to do so (e.g. uncontained fire).

Ditching and off field forced landing without power are discussed in Section
\ref{sec-all-engine-failure}. Support for such landings is provided by the
\ecam{ENG}{ALL ENGINES FAILURE} \ac{ECAM} and the \ecam{ENG}{ALL ENG FAIL}
\ac{QRH} procedure.

Support for ditching and forced landing with power is provided by the \ac{QRH}
\ecam{MISC}{Ditching} and \ac{QRH} \ecam{MISC}{Forced Landing} procedures
respectively. Necessarily implicit in these checklists is the assumption that
the aircraft is fully serviceable, which is unlikely to be the case. There is
also an assumption that plenty of time is available for extensive preparation of
cabin and cockpit. It is highly likely that these checklists will need adapting
to the situation.

The fundamentals are the same with or without power. Ditching will be gear up
with a target pitch attitude of 11\textdegree{ }and minimal vertical speed, and
should be made parallel to the swell unless there are strong crosswinds, in
which case an into wind landing is preferred. Forced landing will be gear down
with the spoilers armed. The aircraft should be depressurised for the landing,
with the Ditching \ac{PB} pushed for the ditching case; in the without power
cases this is achieved using \ac{RAM AIR}, and this is the same for with power
forced landing. The ``with power'' ditching achieves depressurisation by turning
off all bleeds, which provides a slightly more watertight hull.

The main difference between the with and without power cases is that max
available slats and flaps are used in the former case, wheras \ac{CONF} 2 is
mandated for the latter. Approach speeds must also be high enough to prevent
\ac{RAT} stall (i.e. >140kt) if it is being relied upon. The combination of
these factors means that much lower landing speeds can be achieved if power is
available.

\multicite{QRH~AEP.MISC}

\section{Evacuation}
\label{sec-evacuation}
Evacuation should be carried out in accordance with the emergency evacuation
checklist. The easyJet procedure is for \ac{CM}1 to call for the checklist and
then send a Mayday message to \ac{ATC} before commencing the checklist.

The first two items confirm the \ac{RTO} actions of stopping the aircraft,
setting the parking brake and alerting the cabin crew. The next item confirms
\ac{ATC} has been alerted.

The next four items prepare the aircraft for evacuation. If manual cabin
pressure has been used, \ac{CM}2 checks cabin diff is zero, and if necessary
manually opens the outflow valve. \ac{CM}2 then shuts the engines down with
their master switches, and pushes all the fire buttons (including the
\ac{APU}). Confirmation is \emph{not} required before carrying out these
actions. In response to the next checklist item, ``Agents'', \ac{CM}1 decides if
any extinguishing agents should be discharged and instructs \ac{CM}2 to
discharge them as required. Engine agent 2 will not be available. Agents should
only be discharged if there are positive signs of fire.

Finally, order the evacuation. This is primarily done with the \ac{PA}
``Evacuate, unfasten your seat belts and get out'', with the evacuation alarm
being triggered as a backup.

\multicite{EOMB~3.80.1, FCOM~PRO.AER.MISC, FCTM~PRO.AER.MISC}


\section{Overweight landing}

A landing can be made at any weight, providing sufficient landing distance is
available. In general, automatic landings are only certified up to \ac{MLW}, but
the \ac{FCOM} specifies that, for the A319 only, autoland is available up to
69000kg in case of emergency.

The preferred landing configuration is \ac{CONF}~Full, but lower settings may be
used if required by \ac{QRH} or \ac{ECAM} procedures. \ac{QRH}
\ecam{MISC}{Overweight Landing} also specifies \ac{CONF}~3 if the aircraft
weight exceeds the \ac{CONF}~3 go around limit; this will only ever be a factor
for airfields with elevations above 1000ft.

Packs should be turned off to provide additional go around thrust.

If planned landing configuration is less than \ac{Conf}~Full, use \ac{Conf}~1+F
for go-around.

It is possible that S speed will be higher than \V{FE~next} for \ac{CONF}~2. In this
case, a speed below \V{FE~next} should be selected until \ac{CONF}~2 is achieved,
then managed speed can be re-engaged.

In the final stages of the approach, reduce speed to achieve \V{LS} at runway
threshold. Land as smoothly as possible, and apply max reverse as soon as the
main gear touches down. Maximum braking can be used after nosewheel
touchdown.

After landing, switch on the brake fans and monitor brake temperatures
carefully. If temperatures exceed 800\textdegree C, tyre deflation may occur.

\multicite{QRH~AER.MISC, FCOM~PRO.AER.MISC, FCTM~PRO.AER.MISC}

\section{Engine failure in cruise}

Engine out ceiling is highly dependent on weight; \ac{ISA} deviation also has a
modest effect. It will generally lie between \ac{FL}180 and \ac{FL}350.

The first action will be to select both thrust levers to \ac{MCT} so as to allow
the autothrust its full engine out range. If the \ac{N}1 gauges indicate a
thrust margin exists, then the aircraft is below engine out ceiling; descent may
be appropriate to increase the available thrust margin, but there is no
immediate threat. If, however, the \ac{N}1 gauges indicate that the autothrust
is commanding \ac{MCT}, and the speed is still decaying, then the aircraft is
above engine out ceiling and prompt execution of a drift down procedure is
required.

Drift down with autopilot engaged in \ac{OP} \ac{DES} is preferred. Engagement
of this vertical mode normally results in the autothrust commanding idle thrust,
which is not what is desired. Thus, having set the thrust lever to \ac{MCT}, the
autothrust is disconnected. The \ac{PROG} page provides a \ac{REC MAX EO} flight
level to use as an altitude target. If the speed decay is modest, it may be
possible to alert \ac{ATC} before initiating the descent, but in-service events
have shown that speed decay is often very rapid, requiring descent initiation to
be prioritised.

Once drift down has been initiated, a decision needs to be made about speed. If
obstacles are a concern, the lowest drift down rate and highest ceiling are
achieved at green dot. Airbus refers to drifting down at green dot as ``Obstacle
strategy''. Flying at green dot reduces the chance of the \ac{FADEC}s
automatically relighting the failed engine as the engine will be windmilling
more slowly. Therefore, if obstacles are not a concern, M.78/300kt is flown, a
speed that will always fall within the stabilized windmill engine relight
envelope; Airbus refers to this as ``Standard Strategy''.

If obstacles remain a problem, \ac{MCT} and green dot speed can be maintained to
give a shallow climbing profile. Once obstacles are no longer a problem, descend
to \ac{LRC} ceiling (use \ac{V/S} if <500 fpm descent rate), engage the autothrust
and continue at \ac{LRC} speed.

\multicite{FCTM~PRO.AEP.ENG.EFDC}
\section{Single engine circling}

It may not be possible to fly level in the standard circling configuration of
\ac{CONF} 3, gear down. This can be ascertained by checking the table in the
\ac{QRH}\ \ecam{misc}{One Engine Inoperative -- Circling Approach} procedure. If
affected, plan a \ac{conf} 3 landing and delay gear extension until level flight
is no longer required; anticipate a \cphrase{L/G NOT DOWN} \ac{ecam} warning
below 750ft (which can be silenced with the \ac{EMER CANC} pb) and a
\ac{GPWS} ``Too Low Gear'' aural alert below 500ft \ac{RA}.

\multicite{QRH~AEP.MISC}

\section{Bomb on board}

The primary aim is to get the aircraft on the ground and evacuated \ac{ASAP}.

The secondary aim is to prevent detonation of the device. This is achieved by
preventing further increases in cabin altitude through the use of manual
pressure control and by avoiding sharp manoeuvres and turbulence.

The tertiary aim is to minimise the effect of any explosion. This is achieved by
reducing the diff to 1 psi. The method is to set cabin vertical speed to zero
using manual pressurisation control, then descend to an altitude 2500ft above
cabin altitude. As further descent is required, cabin vertical speed should be
adjusted to maintain the 1 psi diff for as long as possible. Automatic pressure
control is then reinstated on approach. Low speeds reduce the damage from an
explosion but increase the risk of a timed explosion occurring whilst airborne;
a compromise needs to be found. The aircraft should be configured for landing as
early as possible to avoid an explosion damaging landing systems.

In the cabin, procedures are laid down for assessing the risks of moving the
device and for moving the device to the \ac{LRBL} at door 2R.

\multicite{QRH~AER.80, FCOM~PRO.AER.MISC}

\section{Stall recovery \emph{(memory item)}}

Aerofoil stall is always and only an angle of attack issue. It is not possible
to directly prove an unstalled condition from attitude and airspeed data. The
flight recorders from the December 2014 Air Asia accident recorded an angle of
attack of 40\textdegree\ (i.e. around 25\textdegree\ greater than critical
angle) with both pitch and roll zero and speeds up to 160kt. Importantly, it is
perfectly possible to be fully stalled in the short term unreliable airspeed
configurations described in Section
\ref{sec-unreliable-airspeed}. Identification of a fully stalled condition is
thus largely dependent on identifying a high and uncontrollable descent rate
that does not correlate with normal flight path expectations for the attitude
and thrust applied.

To recover from a fully stalled condition, the angle of attack of the aerofoils
must be reduced to below critical. The generic stall recovery is therefore
simply to pitch the nose down sufficiently to break the stall and level the
wings. In normal operations, the velocity vector of the aircraft is around
3\textdegree{ }below the centerline of the aircraft (i.e. an attitude of around
3\textdegree{ }is required to fly level). In a stalled condition, the velocity
vector may be 40\textdegree{ }or more below the centerline of the
aircraft. Thus the amount of pitch down required to recover a fully stalled
aircraft can be 30\textdegree{ }or more.

In two recent Airbus accidents involving stalls, the lack of physical cross
coupling of sidesticks was a major factor. If one pilot elects to hold full back
sidestick, the aircraft cannot be recovered by the other pilot unless the
takeover button is used. With all the alarms, it would be easy to miss ``Dual
Input'' warnings, so \emph{always} press the takeover button.

The aircraft's thrust vector helps to accelerate the aircraft during the
recovery, and increasing speed along the aircraft's centerline acts to reduce
the stalled angle of attack. Thus, while thrust is not a primary means of
recovery, it does help. Unfortunately, Airbus have determined that due to the
pitch couple associated with underslung engines, there may be insufficient
longitudinal control authority to pitch the aircraft sufficiently to recover
from a stall if \ac{TOGA} is selected. It may therefore be necessary to
initially \emph{reduce} thrust to allow the primary recovery technique to be
applied; this is extremely counterintuitive.

Once there are no longer any indications of the stall, smoothly recover from the
dive, adjust thrust, check speedbrakes retracted and if appropriate (clean and
below 20,000ft) deploy the slats by selecting \ac{CONF}~1. The load factor
associated with an overly aggressive pull out can induce a secondary stall; on
the flip side, once reattachment of the airflow occurs, drag rapidly diminishes
and exceedance of high speed airframe limitations becomes a threat. A balance
needs to be found.

If a stall warner sounds on takeoff it is likely to be spurious since you are
almost certainly in normal law. The procedure in this case is essentially to
initially assume unreliable airspeed and fly \ac{TOGA}, 15\textdegree , wings level
until it can be confirmed that the warning is spurious.

A stall warning may occur at high altitude to indicate that the aircraft is
reaching $\mathrm{\alpha_{buffet}}$. In this case simply reduce the back
pressure on the sidestick and/or reduce bank angle.

\multicite{FCOM~PRO.AER.MISC}


\section{Computer reset}

Abnormal computer behaviour can often be stopped by interrupting the power
supply of the affected computer. This can be done either with cockpit controls
or with circuit breakers. The general procedure is to interrupt the power
supply, wait 3 seconds (5 seconds if a \ac{CB} was used), restore the power,
then wait another three seconds for the reset to complete. \inlcite{QRH
  AER.RESET} details the specific procedures for a variety of systems.

On the ground, almost all computers can be reset. \ac{MOC} can usually supply a
reset procedure if nothing applicable is available in the \ac{QRH}. The
exceptions are the \ac{ECU} and \ac{EIU} while the associated engine is running
and the \ac{BSCU} when the aircraft is not stopped.

In flight, only the computers listed in the \ac{QRH} should be considered for
reset.

\multicite{QRH~AER.SYSTEM~RESET}

\section{Landing distance calculations}

Many failures result in a longer than normal landing distance. The \ac{QRH}
inflight performance section has tables for calculating \V{APP} and Reference
Landing Distances for single failures. These reflect the performance achievable
in a typical operational landing without margin. easyJet requires a factor of
1.15 to be applied to these distances.

The \ac{EFB} module provides both factored and unfactored landing distances, and
also can calculate for multiple failures.

The safety factor may be disregarded in exceptional circumstances.

\multicite{QRH~IFP, FCOM~PER.LDG, EOMB~4.14.2}

\section{Abnormal V Alpha Prot}

If two or more angle of attack vanes become frozen at the same angle during
climb, a Mach number will eventually be reached such that the erroneous angle of
attack data indicates an incipient stall. When this happens, Normal Law high
angle of attack protection will activate. The flight computers' attempt to
reduce angle of attack will not, however, be registered by the frozen vanes,
leading to a continuous nose down pitch rate which cannot be overridden with
sidestick inputs.

Indications of this condition are available from the
$\alpha$\textsubscript{prot} and $\alpha$\textsubscript{max} strips. If the
$\alpha$\textsubscript{max} strip (solid red) completely hides the
$\alpha$\textsubscript{prot} strip (black and amber) or the
$\alpha$\textsubscript{prot} strip moves rapidly by more than 30kt during flight
manoeuvres with \ac{AP} on and speed brakes retracted, frozen angle of attack
vanes should be suspected.

The solution is to force the flight computers into Alternate Law where the
protection does not apply. This is most conveniently done by turning off any two
\ac{ADR}s. Once in Alternate Law, the stall warning strip (red and black)
becomes available. Since stall warning data also comes from the angle of attack
vanes, erroneous presentation is likely.

\section{Overspeed Recovery}

In general the response to an overspeed should be to deploy the speedbrake and
monitor the thrust reduction actioned by the autothrust. Disconnection of the
autopilot will not normally be required. If autothrust is not in use, the thrust
levers will need to be manually retarded.

It is possible that the autopilot will automatically disengage and high speed
protection will activate, resulting in an automatic pitch up. In this case,
smoothly adjust pitch attitude as required.

At high altitude, there is a threat of over-correction caused by the lethargic
response of the speedbrake when commanded to stow. In the worst case, a descent
may be required to recover speed. This threat can be mitigated by promptly
cancelling the speedbrake as soon as the overspeed condition ceases.

\multicite{FCTM~PRO.AER.MISC}

\section{Volcanic Ash Encounter}

Volcanic ash clouds are usually extensive, so a 180\textdegree\ turn will
achieve the quickest exit.

Air quality may be affected, so crew oxygen masks should be donned with 100\%
oxygen to exclude fumes. Passenger oxygen may also need to be deployed.

Probes may become blocked with ash, so be prepared to carry out the unreliable
speed procedure.

Disconnect the autothrust to prevent excessive thrust variations.

To minimise the impact on the engines, if conditions permit thrust should be
reduced. Turn on all anti-ice and set pack flow to high in order to increase
bleed demand and thus increase engine stall margin. Wing anti-ice will need to
be turned off again before attempting relight in case of flameout.

If engine \ac{EGT} limits are exceeded, consider a precautionary engine shutdown
with restart once clear of volcanic ash. Engine acceleration may be very slow
during restart. Since compressor and turbine blades may have been eroded, avoid
sudden thrust changes.

Damage to the windshield may necessitate an autoland or landing with a sliding
window open.

\multicite{QRH~AEP.MISC, FCOM~PRO.AEP.MISC, FCTM~PRO.AEP.MISC}

\chapter{Air con and pressurisation}

\section{Cabin overpressure}

There is no \ac{ECAM} in the case of total loss of pressure control leading to an
overpressure, so apply the \ac{QRH} procedure. The basic procedure is to reduce air
inflow by turning off one of the packs and put the avionics ventilation system
in its smoke removal configuration so that it dumps cabin air overboard. The
$\Delta$P is monitored, and the remaining pack is turned off if it exceeds 9
psi. 10 minutes before landing, both packs are turned off and remain off, and
the avionics ventilation is returned to its normal configuration.

\multicite{QRH~AEP.CAB~PR, FCOM~PRO.AEP.CAB~PR}


\section{Excess cabin altitude}

An \ac{ECAM} warning of excess (>9550ft) cabin altitude should be relied upon, even
if not backed up by other indications.

The initial response should be to protect yourself by getting an oxygen mask
on. Initiate a descent; if above \ac{FL}160, this should be in accordance with
the Emergency Descent procedure (see Section \ref{sec-emer-descent}). Once the
descent is established and all relevant checklists are complete, check the
position of the outflow valve and, if it is not fully closed, use manual control
to close it.

\multicite{\uline{CAB PR}~EXCESS~CAB~ALT, FCOM~PRO.AEP.CAB~PR}


\section{Landing Elevation Fault}

If the landing field elevation is not available from the \ac{FMGS}, the landing
elevation must be manually selected. This is done by pulling out and turning the
\ac{LDG} \ac{ELEV} knob. The scale on the knob is only a rough indication; use the \ac{LDG}
\ac{ELEV} displayed on either the \ac{CRUISE} page or the \ac{CAB} \ac{PRESS} \ac{SD} page instead.

\multicite{\uline{CAB PR}~LDG~ELEV~FAULT, FCOM~PRO.AEP.CAB~PR}


\section{Pack fault}
\label{sec-pack-fault}

The \ecam{air}{PACK FAULT} \ac{ECAM} indicates that the pack flow control valve
position disagrees with the selected position or that the pack valve has closed
due to either compressor outlet overheat or pack outlet overheat.

The affected pack should be turned off.

A possible reason for this failure is loss of both channels of an Air
Conditioning System Controller (\ac{ACSC}). If this occurs, the associated hot
air trimming will also be lost (cockpit for \ac{ACSC} 1, cabin for \ac{ACSC} 2).

If there are simultaneous faults with both packs, ram air must be used. This
will necessitate depressurisation of the aircraft, so a descent to \ac{FL}100 (or \ac{MEA}
if higher) is required. If a \ac{PACK} button \ac{FAULT} light subsequently extinguishes,
an attempt should be made to reinstate that pack.

\multicite{\uline{AIR}~PACK~1(2)(1+2)~FAULT, FCOM~PRO.AEP.AIR}


\section{Pack overheat}

The associated pack flow control valve closes automatically in the event of a
pack overheating (outlet temp~>260\textdegree C or outlet temp >230\textdegree
C four times in one flight). The remaining pack will automatically go to high
flow, and is capable of supplying all of the air conditioning requirement. This
system's automatic response is backed up by turning off the pack. The \ac{FAULT}
light in the \ac{PACK} button remains illuminated whilst the overheat condition
exists. The pack can be turned back on once it has cooled.

\multicite{\uline{AIR}~PACK~1(2)~OVHT, FCOM~PRO.AEP.AIR}


\section{Pack off}

A warning is generated if a functional pack is selected off in a phase of flight
when it would be expected to be on. This is usually the result of neglecting to
re-instate the packs after a packs off takeoff. Unless there is a reason not to,
turn the affected pack(s) on.

\multicite{\uline{AIR}~PACK~1(2)~OFF, FCOM~PRO.AEP.AIR}


\section{Pack regulator faults}

A regulator fault is defined as a failure of one of four devices: the bypass
valve, the ram air inlet, the compressor outlet temperature sensor or the flow
control valve. The \ac{ECAM} bleed page can be used to determine which device is at
fault.

Regardless of the device at fault, the ramification is the same; the pack will
continue to operate but there may be a degradation in temperature regulation. If
temperatures become uncomfortable, consideration should be given to turning off
the affected pack.

\multicite{\uline{AIR}~PACK~1(2)~REGUL~FAULT, FCOM~PRO.AEP.AIR}


\section{ACSC single lane failure}

Each \ac{ACSC} has two fully redundant ``lanes'', so loss of a single ``lane'' results in
loss of redundancy only.

\multicite{\uline{AIR}~COND~CTL~1(2)~A(B)~FAULT, FCOM~PRO.AEP.AIR}


\section{Duct overheat}

A duct overheat is defined as a duct reaching 88\textdegree C or a duct reaching
80\textdegree C four times in one flight. If this occurs, the hot air pressure
regulating valve and trim air valves close automatically and the \ac{FAULT} light
illuminates in the \ac{HOT} \ac{AIR} button. This light will extinguish when the
temperature drops to 70\textdegree C.

Once the duct has cooled, an attempt can be made to recover the hot air system
by cycling the \ac{HOT} \ac{AIR} button.

If recovery is not possible, basic temperature regulation will continue to be
provided by the packs.

\multicite{\uline{COND}~FWD~CAB/AFT~CAB/CKPT~DUCT~OVHT, FCOM~PRO.AEP.COND}


\section{Hot air fault}

If the hot air pressure regulating valve is not in its commanded position, the
effects will depend on its actual position.

If it is closed when commanded open, the packs will provide basic temperature
regulation.

More serious is if it has been commanded closed in response to a duct overheat
and it remains open. Manual control may be effective, but if it is not the
only option is to turn off both packs and proceed as per Section
\ref{sec-pack-fault}.

\multicite{\uline{COND}~HOT~AIR~FAULT, FCOM~PRO.AEP.COND}


\section{Trim air faults}

Either a fault with one of the trim air valves or an overpressure downstream of
the hot air valve. An associated message indicates which condition exists.

Failure of a trim valve leads to loss of optimised temperature regulation for
the corresponding zone; basic temperature regulation is still available.

The \ac{TRIM} \ac{AIR} \ac{HIGH} \ac{PR} message may be disregarded if triggered when all the trim
air valves are closed. This occurs during the first 30 seconds after the packs
are selected on and in flight if all zone heating demands are
fulfilled.\footnote{The \ac{FCOM} is not very informative regarding response to
overpressure when this does not apply. However the \ac{MEL} operating procedures for
dispatch with this condition indicate that turning the \ac{HOT} \ac{AIR} pb-sw off is
probably a good idea.}

\multicite{\uline{COND}~TRIM~AIR~SYS~FAULT, FCOM~PRO.AEP.COND}


\section{Cabin fan faults}

If both cabin fans fail, their flow should be replaced by increasing the pack
flow to \ac{HI}.

\multicite{\uline{COND}~L~+~R~CAB~FAN~FAULT, FCOM~PRO.AEP.COND}


\section{Lavatory and galley fan faults}

The cabin zone temperature sensors are normally ventilated by air extracted by
these fans. Loss of the fans therefore leads to loss of accurate zone
temperature indication.

On older aircraft, temperature control reverts to maintenance of a fixed cabin
zone inlet duct temperature of 15\textdegree C.

On newer aircraft the temperature controls for the cabin revert to controlling
temperature in the ducts. If \ac{ACSC} 2 has also failed, the duct temperatures are
maintained at the same level as the cockpit duct temperature, and may therefore
be controlled with the cockpit temperature selector.

\multicite{\uline{COND}~LAV~+~GALLEY~FAN~FAULT, FCOM~PRO.AEP.COND}



\section{Pressure controller faults}

Loss of a single cabin pressure controller leads to loss of redundancy only.

If both pressure controllers are lost, use manual control. The outflow valve
reacts slowly in manual mode, and it may be 10 seconds before positive control
of the outflow valve can be verified. It may also react too slowly to prevent a
temporary depressurisation.

To activate manual pressurisation control, press the \ac{MODE SEL} button. This
allows the \ac{MAN V/S CTL} toggle switch to directly control the outflow
valve. Moving the toggle to \ac{DN} closes the outflow valve causing the cabin
altitude to descend, whilst moving the toggle to \ac{UP} opens the outflow valve
causing the cabin altitude to climb. The target climb and descent rates are
500fpm and 300fpm, these being displayed on the status page for easy reference.

A table of \ac{FL} versus \ac{CAB ALT TGT} is also provided on the status page;
no guidance is given for the interpretation of this table. The final action of
the procedure is to fully open the outflow valve when reaching 2500ft \ac{AGL}
in preparation for an unpressurised landing, so to avoid large pressurisation
changes during this action, the final cabin altitude target needs to be
aerodrome elevation plus 2500ft. This gives an indication of how \ac{CAB ALT
  TGT} should be interpreted: it is the lowest cabin altitude that still results
in a safe $\Delta$P at a given \ac{FL}. A cabin altitude greater then \ac{CAB
  ALT TGT} is always acceptable\footnote{A reasonable maximum cabin altitude is
8800ft, which is when the \ac{CAB ALTITUDE} advisory triggers.} and, moreover,
for the final stages of the approach, it is necessary. The method is therefore
to avoid cabin altitudes below \ac{CAB ALT TGT} for your current \ac{FL} while
ensuring that a cabin altitude of aerodrome elevation plus 2500ft will be
achieved by the time you need to fully open the outflow valve.

Ensure cabin diff pressure is zero before attempting to open the doors.

\multicite{\uline{CAB~PR}~SYS~1(2)(1+2)~FAULT, FCOM~PRO.AEP.CAB~PR}


\section{Low diff pressure}

High rates of descent may lead to the aircraft descending through the cabin
altitude when more than 3000ft above the landing altitude. An \ac{ECAM} warning
indicates that this situation is projected to occur within the next
1\textonehalf minutes. If the rate of descent of the aircraft is not reduced,
the pressure controllers will have to resort to high rates of change of cabin
altitude, which may cause passenger discomfort. The aircraft's vertical speed
should be reduced unless there is a pressing reason not to.

\multicite{\uline{CAB~PR}~LO~DIFF~PR, FCOM~PRO.AEP.CAB~PR}


\section{Outflow valve closed on ground}

If the outflow valve fails to automatically open on the ground, manual control
should be attempted. If that doesn't work, depressurise the aircraft by turning
off both packs.

\multicite{\uline{CAB~PR}~OFV~NOT~OPEN, FCOM~PRO.AEP.CAB~PR}



\section{Open safety valve}

There are safety valves for both cabin overpressure and negative differential
pressure; the associated \ac{ECAM} message does not distinguish between the two.

If diff pressure is above 8psi, it is the overpressure valve that has
opened. Attempt manual pressurisation control and if that fails, reduce aircraft
altitude.

If diff pressure is below zero, it is the negative differential valve. Reduce
aircraft vertical speed or expect high cabin rates.

\multicite{\uline{CAB~PR}~SAFETY~VALVE~OPEN, FCOM~PRO.AEP.CAB~PR}

\chapter{Avionics Ventilation}

\section{Blower fault}

Defined as low blowing pressure or duct overheat. Unless there is a \ac{DC} \ac{ESS} Bus
fault, the blower fan should be set to \ac{OVRD}. This puts the avionics ventilation
into closed configuration and adds cooling air from the air conditioning
system.%<!-- {TODO:investigate --> <!-- involvement of DC ESS BUS fault} -->

\multicite{\uline{VENT}~BLOWER~FAULT, FCOM~PRO.AEP.VENT}


\section{Extract fault}

Defined as low extract pressure. The extract fan should be put in \ac{OVRD}. This
puts the avionics ventilation into closed configuration and adds cooling air
from the air conditioning system.

\multicite{\uline{VENT}~EXTRACT~FAULT, FCOM~PRO.AEP.VENT}


\section{Skin valve fault}

Defined as one of three faults: the inlet valve is not fully closed in flight;
the extract valve is fully open in flight; or the extract valve did not
automatically close on application of take-off power. The \ac{ECAM} \ac{Cab
  Press} page will differentiate.

If the fault is with the inlet valve, no action is required since it
incorporates a non-return valve.

If the extract valve is affected, the system should be put into smoke
configuration; this sends additional close signals to the extract valve. If the
extract valve still remains open, the \ac{ECAM} directs the crew to depressurise the
aircraft. The rationale for this seemingly extreme reaction to a relatively
minor issue is that the \ac{ECAM} can only really occur immediately after the
take-off inhibit ceases at 1500ft \ac{AAL}. The extract valve is normally held closed
by the pressurisation and its motor is not sufficiently powerful to overcome
this. Thus the extract valve can only be open in flight if it never closed. With
the extract valve open, it will likely not be possible to complete the flight
since the additional hole will make it impossible to properly pressurise the
aircraft at cruise altitude, and the pressurised air rushing through the open
outflow valve will cause it to be unpleasantly noisy in the cockpit. This makes
depressurising the aircraft and returning for engineering attention the obvious
solution.\footnote{The \ac{ECAM} procedure associated with this failure is due to be
modified in an upcoming \ac{FWC} update.}

\multicite{\uline{VENT}~SKIN~VALVE~FAULT, FCOM~PRO.AEP.VENT}


\section{Avionics ventilation system fault}

Defined as either a valve not in its commanded position or the Avionics
Equipment Ventilation Controller (\ac{AEVC}) being either unpowered or failing its
power-up test. The system will automatically default to a safe configuration
similar to smoke configuration. No crew action is required.

\multicite{\uline{VENT}~AVNCS~SYS~FAULT, FCOM~PRO.AEP.VENT}

\chapter{Electrical}

\section{Emergency configuration}

Attempt to restore normal power by recycling the main generators. If that fails,
try again after splitting the systems with the \ac{BUS TIE} button.

If normal power cannot be restored, ensure that the emergency generator is on
line (deploy the \ac{RAT} manually if required) and maintain speed >140kt to avoid
\ac{RAT} stall. Cycling \ac{FAC} 1 will recover rudder trim. Once 45 seconds have elapsed
and when below \ac{FL}250, the \ac{APU} can be started.

So much equipment is lost in the emergency configuration that \ac{QRH}
\ecam{ELEC}{elec emer config summary} provides a table of \emph{surviving}
equipment. Notable losses are:

\begin{itemize}
\item All the fuel pumps, requiring Gravity Fuel Feeding procedures (see Section
  \ref{sec-gravity-fuel-feeding}) and making center tank fuel unusable.

\item The anti-skid, three fifths of the spoilers and the reversers. Combined
  with the higher landing speeds required to prevent \ac{RAT} stall this results
  in significantly increased landing distances.

\item Alternate Law with reduced protections. Mechanical yaw becoming Alternate
  Law yaw with \ac{FAC}1 reset. Anticipate Direct Law at gear extension.

\item Anti-icing for probes supplying the standby instruments.

\item Nose Wheel Steering.
\end{itemize}

The \ac{QRH} \ecam{ELEC}{elec emer config summary} should be applied once
\ac{ECAM} actions are complete.

\multicite{\uline{ELEC}~EMER~CONFIG, QRH~AEP.ELEC, FCOM~PRO.AEP.ELEC}


\section{Battery only}

Power is available for approximately 30 mins.\footnote{This information was
part of Airbus \ac{CBT} training. There is no figure available in the \ac{FCOM}.}
QRH AEP.ELEC provides details of remaining equipment. This is very similar to
the emergency electrical configuration (see <xref>) with the additional loss of
\ac{FAC}1 and \ac{FMGC}1. An attempt should be made to bring the emergency generator on
line by ensuring speed is >140kt and deploying the \ac{RAT} with the \ac{EMER} \ac{ELEC} \ac{PWR}
\ac{MAN} \ac{ON} button.

\multicite{\uline{ELEC}~ESS~BUSES~ON~BAT, QRH~AEP.ELEC, FCOM~PRO.AEP.ELEC}


\section{IDG low oil pressure/ high oil temperature}

The \ac{IDG} should be disconnected. Assuming the associated engine is running, press
the \ac{IDG} button until the \ac{GEN} \ac{FAULT} light comes on. Do not press the button for
more than 3 seconds.

The \ac{APU} generator should be used if available.

\multicite{\uline{ELEC}~IDG~1(2)~OIL~LO~PR/OVHT, FCOM~PRO.AEP.ELEC}


\section{Generator fault}

Try to reset the generator by turning it off, then after a short pause, turning
it on again. If unsuccessful, turn it back off.

If an engine driven generator cannot be recovered, the \ac{APU} generator should be
used if available.

Single generator operation leads to shedding of the galley. Loss of an engine
driven generator leads to loss of \ac{CAT} \ac{III} \ac{DUAL} capability.

\multicite{\uline{ELEC}(APU)~GEN~(1)(2)~FAULT, FCOM~PRO.AEP.ELEC}


\section{Battery fault}

The affected battery contactor opens automatically. \ac{APU} battery start is
unavailable with a single battery.

\multicite{\uline{ELEC}~BAT~1(2)~FAULT, FCOM~PRO.AEP.ELEC}


\section{AC Bus 1 fault}

Some or all of the equipment on \ac{AC} bus 1 becomes unavailable, including \ac{TR}1. \ac{DC}
Bus 1 is powered from \ac{DC} Bus 2 via the battery bus. Power must be re-routed to
the Essential \ac{AC} bus via \ac{AC} bus 2. This is automatic on some aircraft. Manual
re-routing is achieved with the \ac{AC} \ac{ESS} \ac{FEED} button. Once Essential \ac{AC} is
powered, the Essential \ac{TR} powers the \ac{DC} Essential bus.

Notable lost equipment includes the blue hydraulic system and associated
services (including spoiler 3), radio altimeter 1 (and hence Cat \ac{III}
capability), half the fuel pumps, the nose wheel steering, the avionics blower
fan and p1 windshield heat.

\multicite{\uline{ELEC}~AC~BUS~1~FAULT, FCOM~PRO.AEP.ELEC}


\section{AC Bus 2 fault}

Some or all of the equipment on \ac{AC} bus 2 becomes unavailable, including \ac{TR}2. \ac{DC}
bus 2 is powered from \ac{DC} bus 1 via the battery bus. The majority of this
equipment has a redundant backup, the loss of the \ac{FO}’s \ac{PFD} and \ac{ND} and a
downgrade to Cat I being the major issue. Landing distances are unchanged.

\multicite{\uline{ELEC}~AC~BUS~2~FAULT, FCOM~PRO.AEP.ELEC}


\section{AC Ess Bus fault}

It may be possible to recover the bus by transferring its power source to \ac{AC} \ac{BUS}
2 with the \ac{AC} \ac{ESS} \ac{FEED} button. If this is unsuccessful, some or all of the
equipment on the \ac{AC} \ac{ESS} bus will be lost. The majority of this equipment has a
redundant backup, with the loss of the Captain’s \ac{PFD} and \ac{ND} and a downgrade to
Cat I being the major issues. Landing distances are unchanged.

It is worth noting that loss of \ac{AC} Ess Bus implies loss of Passenger Oxygen
masks. Where appropriate, this loss of redundancy can be mitigated by flying at
a level where descent to a safe altitude can be achieved without masks. The main
form of guidance on altitude hypoxia comes in the form of “Time of useful
consciousness” tables. Working on the principal that if you remain conscious you
definitely remain alive, 25,000ft would seem to be a good compromise. This gives
you 2 to 3 minutes of useful consciousness to dive to 18,000ft, where you would
then have 30 minutes to clear any terrain.\footnote{These tables are obviously
not designed to be used in this way – the exposure to hypoxia in the descent
will likely impact the \ac{TOUC} at 18,000ft, and we are really more concerned with
survivability than useful consciousness – but they can at least give a feeling
for the parameters involved.}

\multicite{\uline{ELEC}~AC~ESS~BUS~FAULT, FCOM~PRO.AEP.ELEC}



\section{AC Essential Shed Bus lost}

Some or all of the equipment on the \ac{AC} \ac{ESS} \ac{SHED} bus is lost. The major issue is
the loss of the passenger oxygen masks. Landing distances are unchanged.

\multicite{\uline{ELEC}~AC~ESS~BUS~SHED, FCOM~PRO.AEP.ELEC}


\section{DC Bus 1 fault}

Some or all of the equipment on \ac{DC} Bus 1 is lost. Most of the equipment loss
causes loss of redundancy only. Landing distances are unchanged.

\multicite{\uline{ELEC}~DC~BUS~1~FAULT, FCOM~PRO.AEP.ELEC}


\section{DC Bus 2 fault}

Some or all of the equipment on \ac{DC} Bus 2 is lost. The F/O’s static probe sensor
is lost, so \ac{ADR}3 should be selected on the F/O’s side. \ac{FCU}2 is lost, so check
that the baro ref on the \ac{FCU} and \ac{PFD} agree. Landing distance increases by up to
35\% due to the loss of 3 ground spoilers per side and one reverser. Autobrake
is also unavailable. Due to the loss of \ac{SFCC}2, the slats and flaps will be slow
and the engines will remain in approach idle. \ac{FAC}2 is lost, so the
characteristic speeds on both \ac{PFD}s are provided by \ac{FAC}1. F/O window heat, wipers
and rain repellent are lost.

The other lost systems either have redundant backups or are non-essential. It
should be noted that the only flight computers remaining are \ac{ELAC} 1, \ac{SEC} 1 and
\ac{FAC} 1.

\multicite{\uline{ELEC}~DC~BUS~2~FAULT, FCOM~PRO.AEP.ELEC}


\section{DC Essential Bus fault}

The major headache associated with \ac{DC} Essential Bus failure is significant loss
of communications systems. This is exacerbated by a design flaw which, at time
of writing, affects \ac{MSN}s 2037–2402 and \ac{MSN}s 2471–3122. By design, \ac{ACP}1 and \ac{ACP}2
are lost, along with \ac{VHF}1. This allows two-way communication to be recovered by
one pilot using \ac{ACP}3 (selected via the \ac{AUDIO} \ac{SWTG} rotary selector) with \ac{VHF}2 or
\ac{VHF}3. Since speaker 1 is also lost, having P2 handle the radios with speaker 2
at high volume is the only method of both pilots having awareness of \ac{ATC}
communications. On the \ac{MSN}s detailed above, however, the audio cards connecting
cockpit mikes and headsets are <emphasis>all</emphasis> powered from the \ac{DC}
Essential Bus. It may be possible to receive transmissions with a combination of
\ac{VHF}2/3, \ac{ACP}3 on \ac{FO} and speaker 2 on, but transmission is limited to morse code
on the transmit switch\footnote{I am only inferring this solution from the list
of equipment lost; there is no specific documentation to indicate that it will
work.}.

\ac{FCU}1 is lost, so the baro refs should be checked. The \ac{GPWS} is lost and should be
turned off.

Landing distances are increased due to the loss of reverser 2 and the loss of
the blue hydraulic system (and hence spoiler 3). Wing anti-ice is also lost, so
landing distances will also increase significantly if ice is accreted and
increased approach speeds are required.

Slats and flaps are slow due to the loss of \ac{SFCC}1. This also leads to the
engines reverting to approach idle.

Landing capability is Cat 2 due to the loss of the auto-thrust. The \ac{ECAM} status
page incorrectly reports Cat 3 single.

The \ac{FCOM} lists additional systems beyond those listed on the \ac{STATUS} page that
are lost with \ac{DC} Essential Bus failure. Of note among these is loss of \ac{HP} fuel
shutoff valves. This requires that the engines are shut down with the Engine
Fire pushbutton switches.

\multicite{\uline{ELEC}~DC~ESS~BUS~FAULT, FCOM~PRO.AEP.ELEC}


\section{DC Essential shed}

The only major issue is the loss of wing anti-ice. Therefore, avoid icing
conditions, and apply landing distance procedure if ice accretes.

\multicite{\uline{ELEC}~DC~ESS~BUS~SHED, FCOM~PRO.AEP.ELEC}


\section{Loss of DC Bus 1 and DC Bus 2}

Some or all of the systems supplied by \ac{DC} Bus 1 and \ac{DC} Bus 2 are lost.

The implications are the same as for <xref linkend="dc-emerg-conf"/> except that
the \ac{RAT} will not need to be deployed since \ac{ESS} \ac{TR} will be supplied from \ac{AC} 1
instead of the emergency generator.

\multicite{\uline{ELEC}~DC~BUS~1+2 FAULT, FCOM~PRO.AEP.ELEC}


\section{Generator overload}

Shed some load by switching off the galleys.

\multicite{\uline{ELEC}~GEN~1(2)~OVERLOAD, \uline{ELEC}~APU~GEN~OVERLOAD,\\
  FCOM~PRO.AEP.ELEC}


\section{Loss of TRs}

No systems are lost as a result of failure of a single \ac{TR}. If the fault is with
\ac{TR}1 or \ac{TR}2, \ac{DC} \ac{ESS} will be supplied by the \ac{ESS} \ac{TR} via \ac{AC} \ac{ESS}; in this case only
Cat 3 single will be available.

If \ac{TR}1 and \ac{TR}2 are both lost then \ac{DC} Bus 1, \ac{DC} Bus 2 and the \ac{DC} Battery Bus will
also be lost. \ac{DC} \ac{ESS} will remain powered by the \ac{ESS} \ac{TR}. The \ac{FCOM} is not very
forthcoming regarding this failure; there is only a description in the \ac{DSC}
section and nothing specific in the \ac{PRO} section. The situation is, however, very
similar to the “\ac{ECAM} complete” phase of <xref linkend="dc-emerg-conf"/>; the
only difference is that \ac{AC} 1 rather than the emergency generator is providing
the power to the \ac{ESS} \ac{TR}, and hence you don’t need to worry about \ac{RAT} deployment
or \ac{RAT} stall.

\multicite{\uline{ELEC}~TR~1(2), \uline{ELEC}~ESS~TR~FAULT, FCOM~PRO.AEP.ELEC}


\section{Battery bus fault}

Some or all of the equipment on the Battery bus is lost. The only major items
lost are \ac{APU} fire detection and \ac{APU} battery start.

\multicite{\uline{ELEC}~DC~BAT~BUS~FAULT, FCOM~PRO.AEP.ELEC}


\section{DC Emergency configuration}

Defined as the loss of \ac{DC} \ac{BUSSES} 1 + 2, \ac{DC} \ac{ESS} \ac{BUS} and \ac{DC} \ac{BAT} \ac{BUS}. Recovery
assumes that the \ac{DC} \ac{ESS} \ac{BUS} can be fully restored by deploying the \ac{RAT} with the
\ac{EMER} \ac{ELEC} \ac{PWR} button.

Equipment powered or controlled through \ac{DC} \ac{BUS} 1, \ac{DC} \ac{BUS} 2 and the \ac{BAT} \ac{BUS} is
lost. Loss of the equipment associated with the \ac{BAT} \ac{BUS} is fairly benign: mainly
\ac{APU} Battery Start is unavailable due to loss of the \ac{APU} \ac{ECB} and \ac{APU} fire
detection is inop. A lot of equipment is lost with the loss of \ac{DC} \ac{BUS} 1+2, but
it is worth remembering that all three busses are also lost in Emergency
Electrical Config, so you will have, at minimum, all the equipment detailed in
QRH~24.1. The main items of note are:

\begin{itemize}

\item A long runway is required. Minimum \V{APP} is 140kt to prevent \ac{RAT}
  stall. Antiskid, reversers and 60\% of the spoilers are lost. Loss of
  nosewheel steering adds to the difficulty. Braking is from the \ac{ABCU}, so only
  manual braking is available.\footnote{easyJet aircraft automatically modulate
  to 1000psi, but the sim may not.}

  \item Flight computer redundancy is significantly reduced, with only \ac{ELAC} 1,
    \ac{SEC} 1 and \ac{FAC} 1 available. This is, however, sufficient to keep Normal Law,
    so a \ac{CONF} \ac{FULL} landing and, indeed, \ac{CAT} 3 \ac{SINGLE} autoland (using \ac{AP}1 and
    \ac{FMGS}1) are available. The latter may be useful since wipers and window heat
    are lost. Only \ac{FCU} 1 is available.

  \item Deployment of lift devices is slow due do loss of \ac{SFCC} 2, but they are
    all available. Normal gear operation is available through \ac{LGCIU} 1.

  \item Pressurised fuel is available from the \#1 wing tank pumps, but center
    tank fuel is unusable.

  \item Redundancy in the pressurisation system is seriously compromised. Manual
    pressure control and \ac{CPC} 2 are lost, so you are reliant on \ac{CPC} 1 for
    control. Pack 2, \ac{BMC} 2 and cross bleed control are all lost, so you are
    reliant on \ac{ENG} 1 Bleed and Pack 1 for supply. Ram Air remains available, so
    if required a slow depressurisation through turning off Pack 1 followed by a
    depressurised landing with Ram Air is achievable.

  \item Communications are limited to \ac{VHF} 1, controlled by \ac{RMP} 1.

  \item Redundancy in the fire detection and suppression systems is
    compromised. The engines each retain one detection loop and one fire
    bottle. \ac{APU} fire detection and cargo fire extinguishing are lost.

  \item Heating for all static ports is lost, so be alert for unreliable
    airspeed and altitude.
\end{itemize}

\multicite{\uline{ELEC}~DC~EMER~CONFIG, FCOM~PRO.AEP.ELEC}


\section{Static inverter fault}

Normal operations are not affected.

\multicite{\uline{ELEC}~STAT~INV~FAULT, FCOM~PRO.AEP.ELEC}


\section{Generator 1 line off}

Pressing the \ac{GEN} 1 \ac{LINE} button on the emergency electrical panel has much the
same effect as pressing the \ac{GEN} 1 button on the main electrical panel, with the
difference that \ac{GEN} 1 continues to supply its associated fuel pumps. It is
primarily used for the smoke drill. If it’s not meant to be off, turn it on.

\multicite{\uline{ELEC}~EMER~GEN~1~LINE~OFF, FCOM~PRO.AEP.ELEC}


\section{Tripped circuit breakers}

It is generally not recommended to reset circuit breakers in flight. It is,
however, acceptable to attempt a single reset if it is judged necessary for the
safe continuation of the flight.

On the ground, any circuit breakers other than those for the fuel pumps may be
reset as long as the action is coordinated with \ac{MOC}.

The \ac{ECAM} warning will be triggered if a green circuit breaker trips.

\multicite{\uline{C/B}~TRIPPED, FCOM~PRO.AEP.ELEC}


\end{document}
